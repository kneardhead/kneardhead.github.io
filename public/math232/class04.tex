\documentclass[letter]{article}
\usepackage[monocolor]{ahsansabit}

\title{Honors Multivariable Calculus : : Class 04}
\author{Ahmed Saad Sabit, Rice University}
\date{\today}

\begin{document}\maketitle
Forward direction was proved in last class. 
\[
\lim_{\vec{x} \to \vec{a}}  f(\vec{x}) = \vec{L}
\]  if and only if $\forall $ sequence $\{x_k\}$ with $\vec{x}_k \to \vec{a}$ (but not equal) we have $\{f(\vec{x})\} \to \vec{L}$. 

\pf{
Let's prove the statement in reverse. Let's assume the limit is not $\vec{L}$. Suppose \[
\lim_{\vec{x} \to \vec{a}}  f(\vec{x}) \neq  \vec{L}
\] 
We will put our limit definition to work. Then $\forall \epsilon > 0$ such that $\forall \delta > 0$.

There exists $\vec{x}$ not equal ot $\vec{a}$, such that $|\vec{x} - \vec{a}| < \delta$ but $|f(\vec{x}) \to  \vec{L} \ge \epsilon$

There is an $\epsilon$ around $\vec{L}$, that no matter what $\delta$ (which is around $\vec{x}$), there is an $\vec{x}$ is within this periphery of $\delta$, but doing the mapping the $f(\vec{x})$ ends up being outside of the $\epsilon$ ball.

How to get a sequence out of this? $\epsilon$ is fixed but $\delta$ is anything above $0$. 

For any $k \in \mathbb{Z}^{+}$, letting $\delta$ as $\frac{1}{k}$, we can find $\vec{x}_k$ satisfying, 
\[
|\vec{x} - \vec{a}| < \frac{1}{k}
\] 
But $f(\vec{x}_k) - \vec{L}$ is still outside of $\epsilon$ ball. ($|f(\vec{x}_k) - \vec{L}| \ge \epsilon$).

Now we have sequence $\{\vec{x}_k\}$ which converges to $\vec{a}$ and we have the sequence $f(\vec{x}_k)$ of this thing which never gets within the $\epsilon$ of $\vec{L}$, so it doesn't converge. 
}

Ruden Analysis for more, Davidson and Donnsig. 

How can we describe $f(x) = \sqrt{x} $ if we want to solve for $\lim_{x \to 0} f(x)$. Domain in multivariable can look weird,
\begin{figure}[ht]
    \centering
    \incfig{domain-problem-in-multivariable-calculus}
    \caption{domain problem in multivariable calculus}
    \label{fig:domain-problem-in-multivariable-calculus}
\end{figure} 

First we need to define what points are legitimate to take a limit off? This is the valid inputs that we can give our function to throw out $f(\vec{x})$. 

Another examples, let's say our domain is $\mathbb{Z}$ in $\mathbb{R}$. Does it make sense to take a limit? Taking a limit towards $2$ doesn't make sense because though $2 $ is in the domain it is we are not really intreested in $x=2$. 

\df{
If $D \subset \mathbb{R}^{n}$, we say that $\vec{a} \in \mathbb{R}^{n}$ is a limit point of $D$ if you can go arbitrarily close to $\vec{a}$ without equalling $\vec{a}$ itself. Hence, if $\forall \epsilon > 0$ that $\exists \vec{x} \in  D$ without $\vec{x} \neq \vec{a}$ but $|\vec{x}-\vec{a}| < \epsilon$.

Or you can say, for $\forall \epsilon>0$, $B_\epsilon(\vec{a})$ contains a point in $D$ that is not $\vec{a}$ itself. \[
	B_\epsilon(\vec{a}) \cap (D \smallsetminus \{\vec{a}\})  \neq \phi
\] 
}

Talk about limits if the domain is $\mathbb{Q}$ in $\mathbb{R}$. What does it mean to take a limit in such case?

\df{
Now \[
f: D \subset \mathbb{R}^{n} \to  \mathbb{R}^{m}
\] 
and $\vec{a}$ is a limit point in $D$, then,
\[
	\lim_{\vec{x} \to \vec{a}} f(\vec{x}) = \vec{L}
\] 
and $\vec{x} \subset D$, everything else is the same as the definition we did before. We are interested in the intersection of the ball and the domain. 
}
\end{document}
