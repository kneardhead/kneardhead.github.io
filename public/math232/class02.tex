\documentclass[letterpaper]{article}
\usepackage[icmcb]{ahsansabit}

\title{Honors Multi Variable Calculus Class 02} 
\author{Ahmed Saad Sabit}
\date{\today} 

\begin{document}
\maketitle

\nt{Because I have not made an example environment hence I will keep them in the problem environment.
}
\thm{This} 
\df{This} 
\pr{this}
\solu{This}

\section{Definition of the Limit $\epsilon-\delta$ rule}
We had come across definition of limits pretty late. I remember reading this on the Complex Analysis by Serge Lang. We will use the formal definition of the limit. 

\df{Formal definition of limits goes like $\lim_{x \to a} f(x) = L$ for $\forall \epsilon > 0$ we have $\exists \delta > 0$ such that if $|x-a| < \delta$ where $x \neq a$, then $|f(x) - L| < \epsilon$. 
	\[
	\lim_{x \to a} f(x) = L ; \forall \epsilon > 0, \exists \delta > 0, 
	\] 
}

We are currently talking about choosing the appropriate $\delta$ or $\epsilon$ to solve the problem.

\pr{$f: \mathbb{R} \to \mathbb{R}$, the Dirichlet goes like, 
\[
f(x) = 1
\] If $x$ is rational. 
\[
f(x) = 0
\] If $x$ is irrational. Now what is the $\lim_{x \to 0} f(x)$?
}

\solu{There is no limit. We can prove $f(x)$ is not $1$ by saying the definition of limit is not working here. Negation would go like there exists $\epsilon > 0$ we have $\forall  \delta > 0$, there exists $x$ such that $|x-a| < \delta$ but $|f(x) - L | \geq \epsilon$. 

The professor goes on to choose $\epsilon = \frac{1}{2}$, no matter what we take $\delta$ to be, as long as $\delta > 0$, but we are trying to show that there is an $x$ in this $\delta$ range where $f(x)$ is too far from $L$.

We can find some $x$ which is irrational between $0$ and $\delta$. So $|x-0|< \delta$ but $f(x) = 0$ so $|f(x) - 1 | > \epsilon$. 
}

You can prove that two different limits at the same point. Interesting!


\pr{
\[
\lim_{x \to 1} x^2+2x = 3
\] 	
}
\pf{
Let $\epsilon > 0$, we choose $\delta$? 

We have to do some scratch work before we decide what we want to be $\delta$.

We want to get $|x^2+2x - 3 |< \epsilon$. Where $|x-1|$ is small enough. We need to figure out what small enough is. We can do this,
\[
|x^2 - 1 + 2x -2| = |\left(x-1\right)\left(x+1\right)+ 2(x-1)| \leq |(x-1)(x+1)| + |2(x-1)|
\]
So both terms being $\frac{\epsilon}{2}$ gives us the total $\epsilon$.Let's try something like both two terms of the last equation to be $< \frac{\epsilon}{2}$. This condition can be filled by,
\[
2(x-1) < \frac{\epsilon}{2}
\] For which we have $x-1 < \frac{\epsilon}{4}$. 

For the first term we now just consider $|x-1| < 1$, then $|x+1| < 3$, and from here, 
\[
|(x-1)(x+1)| < 3 |x-1|
\] And this will be $< \frac{\epsilon}{2}$ if $|x-1| < \frac{\epsilon}{6}$. Now put all the three bounds for $\epsilon$ together, then, choosing $\delta = \min(1,\frac{\epsilon}{6})$. 

We are flexible, we can just come up with any delta that works and that is enough. 

\textbf{So starting formally,} $\epsilon>0$, we choose $\delta = \min\left(\frac{\epsilon}{6},1\right)$, then if $|x-1| < \delta$, then we know, 
\[
|x-1| < 1, 0<x<2, |x+1| < 3
\] 
So, $|(x-1)(x+1)|< \frac{\epsilon}{6} \cdot 3 = \frac{\epsilon}{2}$.
Also $|2(x-1)| = 2 |x-1| \leq 2 \cdot \frac{\epsilon}{6}   \frac{\epsilon}{3} < \frac{\epsilon}{2}$.

Hence, \[|x^2+2x-3| \leq |(x-1)(x+1)| + |2 (x-1)| < \frac{\epsilon}{2} + \frac{\epsilon}{2} = \epsilon\]
}

So, how does this extend to multivariable? 

\section{For multi variables?}
What should happen for multi-variable function? 
\[
f: \mathbb{R}^2 \to \mathbb{R}
\] 
Or what should mean,
\[
\lim_{(x,y) \to (1,4)} f(x,y) = 5
\] 
Intuitively, as $(x,y)$ goes closer to $(1,4)$, the output gets closer and closer to $5$. The same definition holds, but we just want the distance of $(x,y)$ from the limit instead of $|x-a|$ (where $a$ is limit) to be lower than a bound $\delta$. 

You can still use something like, 
\[
\lim_{\vec{x} \to \vec{a}} f(\vec{x}) = \vec{L}
\] 
It's still exactly the same thing, still! 

\df{
\[
\lim_{\vec{x} \to \vec{a}} f(\vec{x}) = \vec{L}
\] 
Definition for multiple variables is that
\[
\forall \epsilon > 0, \exists \delta > 0
\] such that if
\[
| \vec{x} - \vec{a} | < \delta 
\] then,
\[
| f(\vec{x}) - \vec{L} | < \epsilon
\] 
}

\begin{figure}[ht]
    \centering
    \incfig{for-the-2-dimensional-case-the-above-figure-is-something-like}
    \caption{For the 2 Dimensional case the above figure is something like}
    \label{fig:for-the-2-dimensional-case-the-above-figure-is-something-like}
\end{figure}

\pr{
$$
\lim_{x,y \to 1,2} x+y = 3	
$$
}
\pf{
Let $\epsilon > 0$, we pick a $\delta$. Scratch work,
\[
|x+y -3| < \epsilon
\] 
So, 
\[
|x-1 + y -2| \leq |x-1| + |y-2|
\] 
Set each terms to be lower than $\frac{\epsilon}{2}$. Then, \[
|x-1 + y -2| \leq \epsilon
\] 
We want the distance to be lower than $\delta$, hence,
\[
\sqrt{(x-1)^2 + \left(y-2\right)^2} < \delta
\] 
But using flexibly,
\[
|x-1| < \delta
\] And
\[
|y-2| < \delta
\] 
We want both of these terms to be lower than $\frac{\epsilon}{2}$, so just consider $\delta$ to be lower than $\frac{\epsilon}{2}$. Yay! 

\textbf{Formally}, setting $|(x,y) - (1,2)| < \delta = \frac{\epsilon}{2}$, we have set this. Hence both being smaller than 
\[
|x+y - 3| < \epsilon
\] 
}

\begin{figure}[ht]
    \centering
    \incfig{for-the-last-problem-the-adjoined-diagram}
    \caption{For the last problem the adjoined diagram}
    \label{fig:for-the-last-problem-the-adjoined-diagram}
\end{figure}
\end{document}
