\documentclass[a4paper]{article}
\usepackage[monocolor]{ahsansabit}

\title{Computational Complex Analsys}
\date{\today}
\author{Ahmed Saad Sabit}
\begin{document}
\maketitle

\df{Definitions on Hyperbolic Trigonometry
\[
e^{z} = \sum_{n=0}^{\infty} \frac{z^{n}}{n!}
\] And we proved that this had the usual properties that solve $ke^{z+w} = e^{z} e^{w}$ webreak them into even and odd, 
\[
= \sum_{n=0}^{\infty} \frac{z^{2n}}{(2n)!} +
\sum_{n=0}^{\infty} \frac{z^{2n+1}}{(2n+1)!}
\] 
\[
= \cosh(z) + \sinh(z)
\] 

We take the same thing and replace $z$ with $-z$, and nothing happens in the first term because there is a positive power,
\[ e^{-z}
= \sum_{n=0}^{\infty} \frac{z^{2n}}{(2n)!} -
\sum_{n=0}^{\infty} \frac{z^{2n+1}}{(2n+1)!} =  \cosh(z) - \sinh(z)
\] 
}

Now using them we have, 
\[
\cosh(z) = \frac{e^{z} + e^{-z}}{2}	
\] 
\[
\sinh(z)  = \frac{e^{z} - e^{-z}}{2}
\] 
\[
\cosh(z) \sinh(z) = \frac{e^{2z} - e^{-2z}}{4} = \frac{\sinh\left(2z\right)}{2}
\] 
\[
\sinh(2z) = 2 \cosh(z) \sinh (z)
\] 

Calc 102 review, 
\[
\cos \theta = 1 - \frac{\theta^2}{2!} + \frac{\theta^{4}}{4!} - \cdots
\] 
\[
\sin \theta = \theta - \frac{\theta^3}{3!} + \frac{\theta^{5}}{5!} - \cdots
\] 

Now we are going to put $iz$, hence,
\[
	e^{iz} = \sum_{n=0}^{\infty} \frac{i^{n} z^{n}}{n!} = \sum \frac{i^{2n}} {\left(2n\right)!} 
z^{2n}
+ 
\sum \frac{i^{2n+1}}{\left(2n+1\right)!} z^{2n+1} 
\] From our Calc 102 review now we have,
\[
e^{iz}= \cos \left(z\right) + i \sin (z)
\]

\nt{
In the book we produce the exponential functions from the $\sin z$ and $\cos z$, but here we take the reverse approach because we consider that $e^{iz}$ is more fundamental.
}

This gives us, 
\[
\cos z = \frac{e^{iz} + e^{-iz}}{2}
\] 
\[
\sin z = \frac{e^{iz} - e^{-iz}}{2}
\] 

Let's do a plot of $e^{i\theta}$.

\begin{figure}[ht]
    \centering
    \incfig{plot-of-complex-power}
    \caption{Plot of complex power}
    \label{fig:plot-of-complex-power}
\end{figure}

\begin{align*}
z &= x + i y \\
&= r \cos \theta + i r \sin \theta \\
&=r \left(\cos \theta + i \sin \theta\right) = 
	re^{i \theta}
\end{align*}

This gives, \[
	\left(r_1 e^{i \theta_1}\right) \left(r_2 e^{i \theta_2}\right) = r_1 r_2 e^{i \left(\theta_1 + \theta_2\right)}
\] 

\[
z = |z| e^{i \left(\arg z\right)}
\] 
The ambiguity is that the argument can be $+2 \pi $. 

If we multiply all points in $\mathbb{C} without \{0\}$ by $r e^{i \theta}$, then the result is, 
\emph{multiply by the modulus $r$ and add $\theta$ to the argument.} Exponentiating is rotation.

\pr{
Page 20: Exercise 01. 

\[
	\left(\sqrt{3}  + i\right)^{7}
\] 
}
\solu{
We can do it in one go (without doing this 7 times). We are looking at $e^{i \frac{\pi}{6}} = \frac{\sqrt{3} }{2} + i \frac{1}{2}$. 
\[
	\left(\sqrt{3}  + i\right)^{7}
= \left(2 e^{i \frac{\pi}{6}}\right)^{7} = 2^{7} e^{i \frac{7\pi}{6}}
\] 
I see that,
\[
\frac{7\pi}{6} = \frac{\pi + 6\pi }{6} = \pi + \frac{1}{6}\pi 
\] 
So what we get is, 
\[
= - 2^{6} \left(\sqrt{3}  + i\right)
\] 
}

\pr{
Find all the $n$-th root of a complex number. $w$ is unknown. $n \in \mathbb{Z}$.} 
\solu{
\[
z = w^{n}
\] We want all possible values of $w$. $z$ is given. First, we may as well assume the modulus of $w$ is $1$. So, find the $n$-th root of $1$. 

We need to find $z$ such that, $z^{n} =  1$. 
\[
z^{n} = 1 = e^{2\pi i} = e^{4 \pi i} = \cdots
\] 
So, we have, 
\[
	z = 1 , e^{2 \frac{i\pi}{n}} , e^{4 \frac{i\pi}{n}} , \ldots
\] 



Observation, consider the polynomial of degree $n$, 
\[
z^{n} - 1 / z^{2 \pi i k \frac{1}{n}} - 1
\] 

Illustration,
\[
\frac{z^{4} - 1}{z - 1}
\] These have no remainder as exact divisions.
\[
z^{n} - 1 = \prod_{k=0}^{n-1} \left(z - e^{2 \pi i \frac{k}{n}}\right) 
\]
Wait bro what is happening $\ldots$.

\[
z^{4} - 1 = \left(z-1\right)\left(z - e^{\pi \frac{i}{4}}\right)
\left(z - e^{2 \pi \frac{i}{4}}\right) \left(z - e^{3 \pi \frac{i}{4}}\right)
\]
}
\end{document}
