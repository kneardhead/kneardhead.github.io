\documentclass[a4paper]{article}

\usepackage[utf8]{inputenc}
\usepackage[T1]{fontenc}
\usepackage{textcomp}
\usepackage[dutch]{babel}
\usepackage{amsmath, amssymb}


% figure support
\usepackage{import}
\usepackage{xifthen}
\pdfminorversion=7
\usepackage{pdfpages}
\usepackage{transparent}
\newcommand{\incfig}[1]{%
	\def\svgwidth{\columnwidth}
	\import{./figures/}{#1.pdf_tex}
}
\setlength\parindent{0em}
\pdfsuppresswarningpagegroup=1

\title{Honors Linear Algebra 354}
\author{Ahmed Saad Sabit} 
\date{09 januari 2024 - \today}
\begin{document}
\maketitle
\section{Office Hours} 
Tuesday, \emph{12:30 - 2:30}. Wednesday \emph{9:30 - 12:30} and  \emph{2:30 - 3:30}.

\section{Vector Spaces}
\section{1.1 $\mathbb{R}^{n}$ and $\mathbb{C}^{n}$.}
We can have a field $\mathbb{F}$ which can be either $\mathbb{R}$ or $\mathbb{C}$. 

Let's talk a bout a List. A list can be,
\[
	(f_1, f_2, \ldots f_n)
\] 
These are ordered elements and the length of the list is $n$. But, it's no more a list if there are infinite elements. 

$\mathbb{F}^{n}$ is the set of all lists of length $n$ positive integer with elements of $\mathbb{F}$. 
\[
	\left(x_1, x_2, x_3, \ldots x_{n} \right)
\] 

$\mathbb{R}^{2}$ can be drawn like in the cartesian plane, but it's difficult to draw $\mathbb{C}^{2}$. It is an ordered pair of complex numbers. These are vector spaces. 
\[
	\left(x_1, \ldots x_{n}\right) + 
	\left(y_1, \ldots y_{n}\right) = 
	\left(x_1 + y_1, \ldots, x_{n} + y_{n}\right)
\] 
We can multiply scalar times vectors, note that $a$ is any number. It can be real or complex.  
\[
a \in \mathbb{F} 
\] For vectors $v,u$
\[
a \left(v + u\right) = av + au
\] 
\[
	\left(a+b\right) v = av + bv
\] 
We define origin, 
\[
O = 	\left(0,0,0,0, \ldots, 0\right)
\] 
We have a zero vector such that, 
\[
\vec{v} = \vec{v} + \vec{0}
\] 

\section{Definition of Vector Space}
Definition: There are objects called $V$ vectors. There are scalars $\mathbb{F}$. Then we have operations which have the properties of either $\mathbb{R}^{n}$ or $\mathbb{C}^{n}$ made with no notation other than $V$. 

We need addition, $\vec{v} + \vec{w}$. It's still vector. It's commutative $\vec{v} + \vec{w} = \vec{w} + \vec{v}$, its associative, $\vec{v} + \left(\vec{w} + \vec{z}\right) = \left(\vec{v} + \vec{w}\right) + \vec{z}$. 

for all $0$ we have $\vec{v} + 0 = v$. The scalar multiples, $a \in \mathbb{F}$ and $\vec{v} \in V$, 
\[
a \vec{v} \in V
\]

We can have
\[
0 v = 0
\] 
Proof: \[
	\left(0 + 0\right) v = 0v + 0v = 0v
\] 
From here, subtracting one of the $0v$ from one side,
\[
0 = 0 v
\] 

A vector space with scalars $\mathbb{R}$ is called a real vector space. A vector space with scalars $\mathbb{C}$ is a complex vector space.

On $P$-16, in Sheldon Axler, $V$ denotes (upper case V) always denotes a vector space over $\mathbb{F}$. 
And scalars can be complex or real, it's not always just simple $1,2,3, \ldots n$

\section{Subspaces} 
Can a vector space be consist of just one element? Yes, element zero $\{0\}$ all by itself. 

But can it be empty? Because it doesn't have the inverses. It just says it must have at least the origin. 
Definition: If $V$ is a vector space. A subset $U$ of $V$ is said to be a subspace of $V$ if it satisfies all the axioms of a vector space. Using structure of $V$ itself.

\begin{figure}[ht]
    \centering
    \incfig{vector-space-and-subspace-diagram}
    \caption{Vector space and subspace diagram, here Red is not a subspace because it doesn't have the origin.}
    \label{fig:vector-space-and-subspace-diagram}
\end{figure}

$\vec{v_0}$ is a vector and the line that connects it can be a subspace. Any single line that goes through the origin in a $\mathbb{R}^{2}$ space is a subspace. Food for thought, can we prove or disapprove that for a space $\mathbb{R}^{n}$, $\mathbb{R}^{n-1}$ is a subspace.

\newpage
\section{Chapter 2: Finite Dimensional Vector Spaces}
There is a name of this book. It was the textbook of the professor.

Let's review our standing assumptions, 
\begin{itemize}
	\item $\mathbb{F}$, V, here there is a field and $V$ is a vector space over it. 
	\item 
\end{itemize}

Learning objectives are,
\begin{itemize}
	\item Span
	\item Linear Dependence
	\item Bases
	\item Dimension
\end{itemize}

2A span and linear independence. 

Definition: A linear combination of a list $\vec{v_1}, \ldots , \vec{v_n}$ of vectors is any vector of the form,
\[
a \vec{v_1} + \ldots + a_n \vec{v}_n
\] 
Here $a_i$ is a scalar such that $a \in \mathbb{F}$. The set of all linear combination of the list $\vec{v}_1, \ldots, \vec{v}_n$ is a subspace of $V$.

This subset is a subspace of $V$, and called a span. 

If span $\{\vec{v_1}, \ldots \vec{v}_n\}= V$, then we can say that $\vec{v}_1, \ldots \vec{v}_n$ span $V$. 

For $\mathbb{R}^{3}$, $\text{span}\{(1,0,0),(0,1,0) \}$ is the  $x-y$ plane. Until $z\neq 0$, we will have $x,y$ plane to be the span for anything, because no matter how you make the linear combination, you cannot have any vector that points towards $z$ to make a $z$ span.

Definion: Linear Independence : 
Linear independent if $a_1 \vec{v}_1 + a_2 \vec{v}_2 + \ldots + a_n \vec{v}_n = 0$, can only and only be possible if you can use $a_1 = a_2 = a_3 = \ldots = a_n = 0$.

A list of one vector is linearly independent if and only if that vector is not $0$. A list of two vectors is linearly independent if and only if neither are the scalar multiple of the other. 
\[
a_1 \vec{v}_1 + a_2 \vec{v}_2 = 0
\] 

\end{document}
