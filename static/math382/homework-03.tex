\documentclass[letter]{article}
\usepackage[monocolor]{ahsansabit}

\title{Computational Complex Analysis : : Homework 03}
\author{Ahmed Saad Sabit, Rice University}
\date{\today}

\begin{document}
	\maketitle
\section{Problem}
The first few derivatives of the function are, evaluated at $z=0$,
\[
f^{(1)} = \frac{d}{dz} \frac{1}{(1-z)^3} = \frac{3}{(1-z)^{4}}  = 3 = \frac{3!}{2!}
\] 
\[
f^{(2)}=\frac{d}{dz} \frac{3}{(1-z)^{4}} = \frac{12}{(1-z)^{5}} = 12 = \frac{4!}{2!}
\] 
\[
f^{(3)}=\frac{d}{dz} \frac{12}{(1-z)^{5}} = \frac{60}{(1-z)^{6}} = \frac{5!}{2!}
\] 
We can see the pattern, so using the McLaurin Series,
\[
f (z) = \sum_{n=0}^{\infty} \frac{f^{(n)}(0)}{n!} z^{n}
\] 
This gives us, 
\[
	\boxed{	\frac{1}{(1-z)^3} = \sum_{n=0}^{\infty} \frac{(2+n)!}{2!\,n!}}
\]  

\section{Problem}
Like the previous problem on the paper I computed the $a_n$ each, 
\[
	f^{(1)} = \frac{6z}{(3-z)^3}
\] 
\[
f^{(2)} = \frac{6(3+2z)}{(3-z)^{4}}
\] 
\[
f^{(3)} = \frac{36(z+3)}{(3-z)^{5}}
\] Using a calculator,
\[
a_0, a_1, a_2, a_3, \ldots = 0 , 0 , \frac{2}{9} , \frac{4}{9}, \frac{8}{9}, \frac{160}{81}, \frac{400}{81}\ldots
\] 
We get, 
\[
\boxed{
	\left(\frac{z}{3-z}\right)^2 = 
	\frac{1}{9}z^2 + \frac{2}{27}z^3 + \frac{1}{27}z^{4} + \frac{4}{243} z^{5} + \frac{5}{729}z^{6} + \cdots
}
\] 

\section{Problem}
We take repeated derivatives and find out $a_n$ each one by one, doing the calculation like above with the help of a calculator, 
\[
\text{series}(a_n) = 0 ,1 ,2,2,0,-4,-8 ,\ldots
\] 
From that we get, 
\[
\boxed{
\sin(z) e^{z} = z + z^2 + \frac{1}{3}z^2 - \frac{1}{30}z^{5} - \frac{1}{90}z^{6} + \ldots
}
\] 

\section{Problem}
Each derivative considered $z=0$ for $e^{z} + e^{\omega z } + e^{\omega^2 z} / 3$ gives
\[
f^{(1)} = \frac{1+\omega + \omega^2}{3}
\] 
\[
f^{(2)} = \frac{1 + \omega^2 + \omega^{4}}{3}
\] 
\[
f^{(3)} = \frac{1 + \omega^{3} + \omega^{6}}{3}
\] 
\[
f^{(4)} = \frac{1+\omega^{4}+\omega^{8}}{3}
\] 
Here $\omega = e^{\frac{2\pi i }{3}} $ so 
\[
 \omega^{(3k)} = e^{2\pi k i } = 1
\]
And noticing, 
\[
	(1+\omega+\omega^2)^2 =\omega^{4} + 2\omega^3 + 3\omega^2 + 2\omega + 1 = 0  
\]We get something interesting,
\[
\omega^{4} + 3\omega^2 + 1 + 2\omega^3 + 2 \omega = 0 
\]Breaking down the $3\omega^2$ and using $\omega^3 =1$\[
\omega^{4} + \omega^2 + 1 + 2 + 2 \omega^2 + 2 \omega = 0 
\] We get, 
\[
\omega^{4} + \omega^2 + 1 = 0
\] It can be shown that further terms are also 0. From here we have all the rest of the terms to be $0$. Hence the maclaurin series, 
\[
\frac{e^{z} + e^{\omega z} + e^{\omega^2 z}}{3} = \frac{1}{0!} = \boxed{
1
} 
\] 

\section{Problem}
Let's define $q = z - \pi i $. 
\[
	\sum_{n=0}^{\infty} (-1)^{n} \frac{(z-\pi i )^{n}}{n!}=
	\sum_{n=0}^{\infty} \left( \frac{(-1)^{n}}{n!}\right) q^{n}
\] 
This is quite similar to the maclaurin series of $e^{x}$, and considering $x= -p$
\[
\sum_{n=0}^{\infty} \frac{x^{n}}{n!}
=\sum_{n=0}^{\infty} (-1)^{n}\frac{p^{n}}{n!}
\]
So what we have up there is simply,
\[
	\sum_{n=0}^{\infty} \left( \frac{(-1)^{n}}{n!}\right) q^{n} = e^{-q} = 
	e^{\pi i - z} = \boxed{
	-e^{-z}
	}
\] 
\end{document}
