\documentclass[letter]{article}
\usepackage[monocolor]{ahsansabit}

\title{Computational Complex Analysis : : Class 05}
\author{Ahmed Saad Sabit, Rice University}
\date{\today}

\begin{document}\maketitle
	If the derivative exists for a point $z_0$, then we have \[
	\frac{\partial f}{\partial x} = \frac{1}{i} \frac{\partial f}{\partial y}
	\] 
	Suppose $f=f(z)$ has continuous partial derivatives with respect to $x$ and $y$. Now we want to show $f'$ exists. We know that $f$ is differentiable in real variable sense. 
	\[
	f(z+h) = f(x+h_1,y+h_2) = f(x,y) + \frac{\partial f}{\partial x} (x,y)h_1 + \frac{\partial f}{\partial y} h_2 + \text{Smaller Terms}
	\] 
	Here $h = h_1 + ih_2$ Using this we can come to the proof, 
	\[
		\lim_{h \to 0} \frac{f(z+h) - f(z)}{h} 	
	\] This thing exists. There is a proof for this. TODO. 

\textbf{Terminology} 
\begin{table}[htpb]
	\centering
	\label{tab:label}
	\begin{tabular}{c|c}
	\textbf{Terminology} 
	\textbf{Frank Jones} & \textbf{Everyone except Frank Jones and his followers} \\ \hline
	Cauchy-Riemann Equation & Cauchy-Riemann Equations
	\end{tabular}
\end{table}

Cauchy Riemann equation in polar form is 
\[
\frac{\partial f}{\partial r} = \frac{1}{i r} \frac{ \partial f}{\partial \theta}
\]
\pr{
Show
\[
\frac{\mathrm{d} \log(z)}{\mathrm{d} z} = \frac{1}{z}
\] 
$z \in  \mathbb{Z} $
}


\end{document}
