\documentclass[letter]{article}
\usepackage[monocolor]{ahsansabit}

\title{Computational Complex Analysis : : Class 04}
\author{Ahmed Saad Sabit, Rice University}
\date{\today}

\begin{document}
\maketitle 

Homework handout is being analyzed. 

\section{Complex Logarithms}
Let $z\neq 0$ be arbitrary, then using polar representation: 
\[
z = r e^{i \theta}
\] 
Now $r = |z|$ and $\theta = \arg(z)$ where the argument can't be an actual function because it has infinitely many outputs.

Warning, never write the $\ln(z)$ where $z \in \mathbb{C}$. We will use log, $\log z = \log r + \log e ^{i\theta}$. We get $\ln r + i \theta$.

\df{
For any $z\neq 0$ complex number, the log of $z$ is defined to be,\[
\log z = \ln |z| + i \arg(z)
\]
Hence turns out, $\log z$ and $\arg z$ has the same sort of ambiguity. 
\[
\arg(z + 2 \pi ) = \arg (z)
\] \[
\log(z + 2\pi ) = \log(z)
\] 
}

We are computing $e^{\log z }$ which is, 
\[
e^{\log z } = e^{\ln |z|} e^{i \arg (z)} = |z| e^{i \arg(z)} = z
\]
That is $z$ no matter of the choice of $\arg z$. 

\[
\log (zw) = \ln |zw| + i \arg(zw) = \ln |z| + \ln|w| + i (\arg z + \arg w) + 2 \pi i k
\] \[
= \log | z| + \log| w| + 2 \pi i k 
\] 

\section{Differentiation} 
This is where complex analysis becomes very different from real analysis. 

Math 101: A function $f(t)$ is differentiable at $t_0$ if $\lim_{h \to 0} \frac{f(t_0 + h) - f(t_0)}{h}$ is $f'(t_0)$. Here is math 382, 

	\df{
	A function $f = f(z_0 )$ is differentiable at $z_0$ if, 
	\[
	\lim_{h \to 0} \frac{f(z_0 + h) - f(z_0)}{h} = f'(z_0)
	\]
	Here, $h \in \mathbb{C}$ and $h \neq  0 $. However this is a huge definition. 
	}

Examples, let's say $f(z) = z^2$. Then having, 
\[
f'(z_0) = \lim_{h \to 0} \frac{f(z_0 + h) - f(z_0)}{h} = \lim_{h \to 0} \frac{(z_0 + h)^2 - z_0^2}{h} = \lim_{h \to 0} \frac{2 z_0 h + h^2}{h} = \lim_{h \to 0}  2z_0 + h = 2 z_0 
\] 		

	\thm{
		If $f$ is a differentiable function at $z_0$, then $f$ is continuous at $z_0$. 
	}
	\pf{
		\begin{align*}
			f'(z_0) &= \lim_{h \to 0} \frac{f(z_0+h)-f(z_0)}{h} \\
			&= \lim_{h \to 0} \frac{1}{h} \left(f(z_0+h)-f(z_0)\right) 
		.\end{align*}
		For this $\frac{1}{h}$ as it blows up, we must  have the parenthesis tend to zero otherwise we don't have a limit.
	}

\df{
Product Rule. 
\[
	(fg)' = fg' + f'g
\] Given the function $\frac{1}{f}$ is also differentiable at $z_0$. 
}

Let's take $f(z) = Re(z)$. It's not differentiable at $0$. 
\[
	(f(h) - f(0)) \frac{1}{h} = \frac{Re(h)}{h} = \frac{x}{x+iy}
\] This has no limit. 

The complex input function can be written as $f(z) = f(x+iy) = f*(x,y)$. 

Thus if $h \in \mathbb{R}$ then, $\frac{f*(x+h,y)}{h} $ is simply $\frac{\partial f*}{\partial x}$. 
For $h \in  \mathbb{C} = i l$ we get $\frac{f*(x,y+l)}{il}$ is giving us $\frac{\partial f*}{\partial y}$. 
\[
\frac{\partial f*}{\partial x} = \frac{1}{i} \frac{\partial f*}{\partial y}
\] This is the Cauchy-Riemann Equation. 
\end{document}
