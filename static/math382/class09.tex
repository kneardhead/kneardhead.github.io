\documentclass[letter]{article}
\usepackage[monocolor]{ahsansabit}

\title{Computational Complex Analysis : : Class 09}
\author{Ahmed Saad Sabit, Rice University}
\date{\today}

\begin{document}
\maketitle 

Open connected set $D \in \mathbb{C}$ continuous $f$ defined on $D$. Then these two statement are equivalent: 
\begin{itemize}
	\item The line integral $\int_\gamma f \ \mathrm{d} z$ are equivalent only on $z_0 \neq z_1$ independent of path. 
	\item $f$ has an anti derivative $F$ defined $D$. $F$ is differentiable on $D$ and $F' = f$. Proof that  existence of an anti derivative is given.
\end{itemize}
\[
\int_\gamma f \mathrm{d} z ; \int_\gamma F' \mathrm{d} z 
\] 
Here $\mathrm{d} z = \gamma' (t) \mathrm{d} t$ 
\[
= \int_a^b F'(\gamma(t)) \gamma'(t) \mathrm{d} t
\]
\[
= \int_a^{b} \frac{\mathrm{d} }{\mathrm{d} t}\left(F(\gamma(t)) \right)\mathrm{d} t
\]
ETC
\[
	F(\gamma(t))_{t = a}^{t = b} = F(z_1) - F(z_0)
\] 
Converse, we assume the integral of $f$ and 
\[
\int_\gamma f(z) \mathrm{d} z
\] is an independent path. We want to produce an anti-derivative $F$. Clever way
define $F(z) = \int_w^{z} f \mathrm{d} z$, this makes sense because any path can do. Try to analyze $F(z+h) - F(z)$. Which is a straight line. 
\[
	F(z+h) = \int_{\text{that path}} f \mathrm{d} z = \int_w^{z} f \mathrm{d} z + \int_z^{z+h} f \mathrm{d} z = F(z) + \int_z^{z+h} f \mathrm{d} z
\] Parametrize $z + th$ 
\[
F(z+h) - F(z) = \int_0^1 f(z+th) h \mathrm{d} t
\]
I am kind of out of clue what is Prof Frank doing at this point. If anything, I want some rest and my head is really heavy right now. 
\[
	\frac{F(h+z) + F(z)}{h} = \int_0^{1} f(z+th) \mathrm{d} t 
\] 
The average of $f(z+th)$ at $0 \le t \le 1$. \[
F'(z) = f(z)
\]

\section{Green's Theorem}
Let there be a region, and we have a function $f(x,y)$ defined for that region. The region is $D$. And $\frac{\partial f}{\partial x}, \frac{\partial f}{\partial y}$ are continuous. Greens theorem says,
\[
	\iint \frac{\partial f}{\partial x} \mathrm{d} x \mathrm{d} y = \int_{\partial D} f \mathrm{d} y
\] 
$\partial D$ is the boundary of $D$.
\begin{figure}[ht]
    \centering
    \incfig{greens-theorem}
    \caption{Greens theorem}
    \label{fig:greens-theorem}
\end{figure}
Likewise 
\[
	\iint_D \frac{\partial g}{\partial y} \mathrm{d} x \mathrm{d} y = - \int_{\partial D} -g \mathrm{d} x
\]
Now for an experiment. 

\[
	\iint_D  = \int_{\partial D} f \mathrm{d} z = \int_{\partial D} f \mathrm{d} x + 
i \int_{\partial D} f \mathrm{d} y
\]
$z = x + iy$ so $\mathrm{d} z = \mathrm{d} x + i \mathrm{d} y$
Let's try this now
\[
	\int_{\partial D} f \mathrm{d} z =  \iint_D -\frac{\partial f}{\partial y} \mathrm{d} x \mathrm{d} y + 
	i \iint_D \frac{\partial f}{\partial x} \mathrm{d} x \mathrm{d} y
\]
\[
= \iint _D \left(i \frac{\partial f}{\partial x} - \frac{\partial f}{\partial y}\right) \mathrm{d} x \mathrm{d} y = i \iint_D \left(\frac{\partial f}{\partial x} - \frac{1}{i} \frac{\partial f}{\partial y}\right) \mathrm{d} x \mathrm{d} y
\]
If $f$ is put to Cauchy-Riemann equation then we get, 
\[
0
\]
If $f$ is differentiable at every point then the cauchy riemann equation states that $\frac{\partial f}{\partial x} - \frac{1}{i} \frac{\partial f}{\partial y} = 0$. Therefore the line integral 
$\int_{\partial D} f \mathrm{d} z = 0$. So the integral $\int f \mathrm{d} z$ is 0 on closed paths. Therefore it's independent of the path. Therefore $f$ has an anti-derivative.

Riemann Cauchy equation needs a derivation by me.

\begin{figure}[ht]
    \centering
    \incfig{derivative-of-derivative-might-not-always-exist.}
    \caption{Derivative of Derivative might not always exist.}
    \label{fig:derivative-of-derivative-might-not-always-exist.}
\end{figure}
Back to $f'(z)$, suppose $f$ is differentiable at $z$ and $f'(z) \neq  0$ then let's try to see about directional derivative $f$ at $z$. 

Consider $z$ and $z+h$ that is a straight line. We are looking at 
$\frac{f(z+th)}{\mathrm{d} t}$ at $t = 0$. 
\[
\lim_{t \to 0} \frac{f(z+th)-f(z)}{t } \approx \frac{f'(z) th}{t}
\] So the complex projection of the derivative along $h$ that can be thought of as a vector is a directional derivative (along $\theta$), $= f'(z) e^{i \theta}$. So rotating the complex number by an angle $\theta$.

Complex differentials preserve angles at points where $f'(z) \neq  0$. Lengths can vary but angles are preserved, hence conformal.  
\end{document}
