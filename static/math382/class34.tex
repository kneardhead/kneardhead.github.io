\documentclass[letter]{article}
\usepackage[monocolor]{ahsansabit}

\title{Computational Complex Analysis : : Class 34}
\author{Ahmed Saad Sabit, Rice University}
\date{\today}

\begin{document}
\maketitle
Liouville's Theorem  is that if $f$ is holomorphic on $\mathbb{C}$ and $|f(z)| \le \text{const}$ then, $f$ is constant. If $u$ is harmonic on $\mathbb{C}$ and $|u(z)| \le  \text{ const } $ the $u$ is constant. Proof is use a harmonic conjugate $v$ and $u+iv$ is holomorphic. Take $f = e^{u + iv}$ and $|f| = e ^{u}$. 

MVP of Harmonic Functions. If $u$ is harmonic, then $u(z_0) = \text{ average of $u$ on circles centered at $z_0$ }$. 

Maximum Principle: if $u$ is harmonic on an open connected set, it cannot have local maximum without being constant. Proof is going to be $u \le M$ where open set and $u = M$ at some point.

Corollary, suppose $D$ is bounded open set and $u$ is continuous $D$ and $\partial D$ and harmonic in $D$. Then $u$ attains its maximum and min value on $D$. 

A solution, if it exists, is unique. Existence, a given function on a body, we want to find harmonic function on $D$ which takes the given value on the body: ``Dirichlet Problem". 




\end{document}
