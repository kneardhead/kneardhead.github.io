\documentclass[letter]{article}
\usepackage[monocolor]{ahsansabit}

\title{Computational Complex Analysis : : Class 21}
\author{Ahmed Saad Sabit, Rice University}
\date{\today}

\begin{document}
\maketitle

We are going to look at $z^{\alpha}$.
Both quantity are $\mathbb{C}$. 
\[
z^{\alpha}=
\text{exp}(\alpha \log z)
\] 
\nt{
\[
= \exp \left(
\alpha \left(
\ln |z| + i \arg z 
\right)
\right) = \exp \left(\alpha \ln |z| \right) \exp \left(i \alpha \arg z\right)
\] 
}
$z \neq 0$, it's ambiguous. Since $\log z$ is
ambiguous. So also we could say $z^{\alpha}$ is 
exponential of $\alpha \log z$ but we can add
\[
z^{\alpha} = 
\exp
\left(
\alpha 
\left(\log z + 2 \pi i n\right)
\right)
\] 
\[
z^{\alpha} = 
\exp
\left(\alpha \log z + 2 \pi i \alpha n\right)
\] 
If $\alpha $ is an integer we have it defined. $z^{\alpha}$ equals as it should. 


Let's work on how about $\alpha = 0$. How about $1 ^{\alpha}$, 
\[
1^{\alpha} = \exp \left(\alpha i \arg 1\right) = \exp (\alpha i \left(2 \pi n \right) )
\]
It has infinitely many values. $\alpha$ is complex.


\section*{Derivative}
What about
\[
\frac{\mathrm{d} }{\mathrm{d}  z} \left(z^{\alpha}\right) = 
\frac{\mathrm{d} }{\mathrm{d} z} \exp\left(\alpha \log z\right) = 
\exp \left(\alpha \log z\right) \frac{\alpha}{z}
\]
\[
= \frac{\alpha}{z}z^{\alpha} = \alpha z^{\alpha - 1}
\] 
So it holds. For another one, 
\[
\frac{\mathrm{d} }{\mathrm{d} \alpha} = z^{\alpha} \left(\log z\right)
\]
For fun try $i^{i}$. 

\section*{Integral} 
New computation of an integral. We want to go from $0$ to $\infty.$
 \[
\int_0^{\infty} \frac{x^{\alpha - 1}}{x + 1}
\] 
Set $\alpha \in \mathbb{R}$. 
The condition on $\alpha$ to make the integral finite 
as $x\to 0$ and $x\to \infty$. 
\[
1 > \alpha > 0 
\] 
Just confirms the integral will exist. Now I am going to look at the plane. 
We are interested in the origin, of course, but we want to go to $+ \mathbb{R}$. 
$z$ has singularity at $z = -1$. 
\begin{figure}[ht]
    \centering
    \incfig{integral-around-for-class-21}
    \caption{Integral around for Class 21}
    \label{fig:integral-around-for-class-21}
\end{figure}
We want to use $f(z) = {z^{\alpha - 1} \over z + 1}$

\[
\int_C \frac{z^{\alpha - 1}}{z+1} \mathrm{d} z = 
2 \pi i 
\left(\text{Residue at -1}\right) = 
2 \pi i  { 
	\left(-1\right)^{\alpha - 1}\over 
	1
}
\] 
Then integrand numerator equals 
\[
\exp \left(\left( \alpha - 1 \right)  \log z \right)
= \exp 
\left( 
\left(\alpha - 1\right) i \arg (-1 ) 
\right)
= 
\exp \left(
	\left(\alpha - 1\right)
\right) i \pi 
\]
The details are
\[
\int _0^{\infty} \frac{x^{\alpha -1} }{x + 1} \mathrm{d} x + \int_\infty^0 \frac{x^{\alpha - 1}}{x+1} e^{2 \pi i \left(\alpha - 1\right)}
\] 
Beware about going around the circle changes the argument by $2\pi $
\[
\int{ x^{\alpha - 1} \over x + 1} \mathrm{d}  x = 
{ 
2\pi i e ^{ \pi i (\alpha - 1) } \over 1 - e^{2 \pi i (\alpha - 1)} } 
\]

\[
= \pi / \sin \pi \alpha
\] 
Here the $\epsilon$ is a radius that will go to zero around $0$.

\section*{Handy methods to extend this example} 
Here's one. Take the formula and change dummy variable, 
\[
\boxed{
\int_0 ^{ \infty} \frac{x^{\alpha - 1}}{x + 1} \mathrm{d}  x = \pi / \sin \pi \alpha
}
\] 
Using $x = y^3$, 
\[
\pi / \sin \alpha \pi = 
\int_0^{\infty} \frac{y ^{3 \alpha - 3} }{ y^3 + 1} 3y^2 \mathrm{d} y 
= 
\int_0^{\infty} 
{y ^{3 \alpha -1} \mathrm{d} y \over y^3 + 1 } 
\] 
\[
\frac{\pi}{\sqrt{3} / 2} = 3 \int_0^{\infty} \mathrm{d} y \frac{1}{y^3 + 1} = \frac{2 \pi }{3 \sqrt{3} }
\]

Differentiate the equation 
\[
f;dsakjl;lkjfdsa;lkjfdsa;lkjfdsa;lkjfdsa;lkjfdsaf;lkjdsa;lkjfdsafdsa;lkj;lkjfdsalkjfdsalkjfdsafdsa;lkj
\]
\[
\int_0^{\infty} \frac{x^{\alpha - 1} \ln x}{x + 1} \mathrm{d}  x = 
\pi \frac{\mathrm{d} }{\mathrm{d}  \alpha} \csc \pi \alpha = - \pi ^2 \csc \pi \alpha \cot \pi \alpha
\]
So for $\alpha = \frac{1}{2}$, 
\[
\int_0^{\infty} \frac{x^{- \frac{1}{ 2}} \ln x}{x + 1} = 0
\]
For $x=e^{y}$, then $x$ goes from $0$ to $\infty$ so $y$ goes from $-\infty$ to $\infty$. 

\[
	\int_{-\infty}^{\infty} e^{y} \mathrm{d} y \frac{e^{(\alpha - 1) y }}{e^{ y } + 1} = \frac{\pi}{\sin \pi \alpha}
\] 


\[
	\int_{-\infty}^{\infty} \mathrm{d} y \frac{e^{\alpha y  }}{e^{ y } + 1} = \frac{\pi}{\sin \pi \alpha}
\] 






































\end{document}
