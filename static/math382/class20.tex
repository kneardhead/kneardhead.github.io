\documentclass[letter]{article}
\usepackage[monocolor]{ahsansabit}

\title{Computational Complex Analysis : : Class 20}
\author{Ahmed Saad Sabit, Rice University}
\date{\today}

\begin{document}
\maketitle

We are going to exploit 
\[
z = r e^{i \theta}
\]
\[
\log z = \ln r + i \theta
\] 
\[
\log z = \ln |z| + i \arg z
\]
Today we will look at 
\[
0 < \arg < 2 \pi 
\] 
The function 
\[
f(z) \log z
\]
Residue theorem 
\[\frac{1}{2\pi i }
\int_C f(z) \log z \mathrm{d} z = \text{ sum of the residues of integrand }f(z)\log z \text{ inside the path }
\]
\[
	\frac{1}{2 \pi i } \int_\epsilon^R f(x) \ln x \mathrm{d} x +
	\frac{1}{2 \pi i}\int_{-C} \cdots + 
	\frac{1}{2\pi i } \int_R^{\epsilon} f(x) (\ln x + 2 \pi i ) \mathrm{d}  x
\]
Taking the limit 
\[
0 = \frac{1}{2 \pi i } \int_0^{\infty} f(x) \ln x - \frac{1}{2 \pi i } \int_0^\infty f(x) 
\left(\ln x + 2 \pi i \right)\mathrm{d} x = \text{ sum of residues }
\]

\[
\boxed{
\int_0^{\infty} f(x) \mathrm{d} x = - \sum_{}^{} f(z) \log z \text{ residues}
}
\]

\section*{Examples}
\[
\int_0^{\infty} \frac{\mathrm{d} x}{ x^3+1}
\] 
so this is going to be 
\[
- \sum_{n}^{} \text{residues of } \frac{\log z}{z^3+1}
\] 
Singularities happen at $z^3 = -1$ 
\[
z = e^{ \pi i \over 3}, e^{ \pi i }, e^{5 \pi i \over 3}
\]
\[
\text{res}\left({\log z \over z^3+1 } \right) = \frac{\log z}{3 z^2} = {z \log z \over 3 z^3} = - {z \log z \over 3}
\] 
Look at 
\[
z^{\alpha}
\] 


\end{document}
