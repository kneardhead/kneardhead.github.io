\documentclass[letter]{article}
\usepackage[monocolor]{ahsansabit}

\title{Computational Complex Analysis : : Class 23}
\author{Ahmed Saad Sabit, Rice University}
\date{\today}

\begin{document}
\maketitle
\section*{Slight improvement to Rouche's Theorem}
Two holomorphic functions $f,g$ around a curve over the domain $D$. $D \cup C$. 
\[
|g(z)| \le |f(z)| \text{ on } C
\] 
and 
$f+g$ is never $0$ on $C$. Then, $f+g$ and $f$ have same number of $0$ in $D$. Proof repeated, 
want to show, 
\[
\int_C 
\frac{(f+g)'}{f+g} - f'/f = 0
\] 
This is re-written as
\[
\int_C \frac{g'f - f' g}{(f+g) f} = \int \frac{(\frac{g}{f}) ' }{1 + \frac{g}{f}}
\]
$h = g / f$, so 
\[
\int_C \frac{h'}{ 1 + h} 
\] 
\[
= \int_C \frac{d}{dz} \left(\log(1+h)\right)\mathrm{d} z = 0
\] 
Before the values of $1+h$ couldn't hit the boundary but now they can hit the boundary. But they cannot go around the origin to change the determination of the argument. 

\section*{Examples}
$3e^{z} - z$ on $|z| \le 1$ how many $0$ are there? On the boundary $C$, the 
$|z|  = 1$, and 
$|3 e^{z } | = 3 e^{x} \pm 3e^{-1} = \frac{3}{e}> 1$ there are no zeroes because $e^{z}$ has no zeros. 

Another example. 
\[
z^{4} - 5z + 1
\] How many zeroes are there in the annulus $1 \le |z| \le  2$. Let's see at $|z| = 2$. So $|z|^{4} = 16$ and $|-5z + 1| \le 11$, and there looks like 4 zeros. In the smaller $|z| = 1$, 
$5z$ dominates, and there's 1 zero. 

Different proof. 
\[
\frac{1}{2 \pi i } \int_C \frac{f' + t g'}{f+ tg} \mathrm{d} z = N(t)
\] 

\df{
Suppose $f$ is a function defined on some set in  $\mathbb{R}^2$ with values in some set in $\mathbb{R}^2$. We say that $f$ is an \emph{open function} if for every open set $G$ where $f$ is defined 
$f$ the image of $f(G)$ is also open. $f(x) = e^{x}$. An open interval in input maps to an open interval in output.    
}
$f(x) = x^2 $ is not open. 

\thm{
	Every non constant 
holomorphic function is an open function.   
}
\pf{
Let $z_0$ be fixed. 
Then let $w_0 = f(z_0)$. 
$f(z_0) - w_0$ is $0$ at $z_0$. 
Let's call the order of zero as $N \ge 1$. 
i.e. 
\[
f(z) - f(z_0) = (z-z_0)^{N} g(z)
\] 
Where $g(z_0) \neq  0$. There is no sequence $z_k \to z_0$  for which $f(z_k) = w_0$. Therefore there should be a circle centered on $z_0$ on inside which $f = w_0 $ only at $z_0.$  
}
Problem for self, does an open interval can map to something closed? 


\end{document}
