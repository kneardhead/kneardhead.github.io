\documentclass[letter]{article}
\usepackage[monocolor]{ahsansabit}

\title{Computational Complex Analysis : : Class 17}
\author{Ahmed Saad Sabit, Rice University}
\date{\today}
\renewcommand{\frac}{\dfrac} 
\begin{document}
\maketitle

\section*{Residue Theorem} 
Apply Cauchy's theorem if $f$ has poles as shown. Let there $z_1, z_2, z_3$ each are there. We apply the cauchy's theorem on a region that doesn't have the neighborhood around the singularities. 

\[
	\int_{\text{region}} f(z) \mathrm{d} x \mathrm{d} y= \frac{1}{2\pi i } \int_{\text{boundary}} f \mathrm{d} z
\]
For the region with holes being analytic
\[
	\frac{1}{2\pi i} \int_{\text{boundary}} f(z)\mathrm{d} z  = 0
\] 
\[
	\int_{\text{boundary of } D} f(z)\mathrm{d} z + \sum_{\text{CW circles}} \int f(z)\mathrm{d} z = 0
\]
\[
	\int_{\partial D} f(z) \mathrm{d} z = \sum_{}^{} \int_{CCW} f(z)\mathrm{d} z
\]
\[
	\frac{1}{2 \pi i} \int_{\partial D} f(z)\mathrm{d} z = 
	\sum_{z \in  D}^{} \text{Res}(f,z)
\] 

\section*{Example} 
\[
	\int_{- \infty}^{\infty} \frac{\mathrm{d} x}{x^2 + 1} = 
\] Calc 2, what do we do? Take $\arctan x$ from $-\infty$ to $\infty$. We get
\[
\frac{\pi}{2} - \left(- \frac{\pi}{2}\right) =  \pi 
\] 
Let's do this using the residue theorem. And then we can move on to the residue theorem $x^2 $ to $x^{4}$. 

So I define $f(z) = \frac{1}{z^2 + 1}$. It has poles at $i$ and $-i$. 
\[
\text{Res}(f, i)= \frac{1}{2z} (z = i) = \frac{1}{2i}
\]
\[
\text{Res}(f, -i)= \frac{1}{2z} (z = -i) = -\frac{1}{2i}
\]
As 
\[
\text{Res} \frac{a(z)}{b(z), z_0} = \frac{a(z_0)}{b'(z_0)}
\]

Now we have to find a region.  The region will be semicircle in complex plane with one of the poles contained. Radius of this semicircle sunset be 
\[
	\frac{1}{2 \pi i } \int_{\partial D_R} \frac{1}{z^2 + 1}\mathrm{d} z = \text{Res}(f,i) = \frac{1}{2i}
\] 
\[
	\int_{\partial D_R} \frac{\mathrm{d} z}{z^2+1} = \pi 
\]
We can estimate the bounds
\[
| \int_{\cap } \int f(z) \mathrm{d} z | \le  \int_{\cap } |f(z) \mid \mathrm{d} z| = \int \frac{1}{|z^2+1|}|\mathrm{d} z | \le \int_0^{\pi } \frac{1}{R^2 - 1} R \mathrm{d} \theta
\] 

\section*{Example} 
Getting even more serious
\[
	\int_{-\infty}^{\infty} \frac{\mathrm{d} x}{x^{4} + 1}
\] Choose the holomorphic function 
\[
f(z) = \frac{1}{z^{4} + 1}
\] 
Find residues, 
\[
z^{4} = -1 = e^{\pi i } = e^{3 \pi i} = e^{5 \pi i } = e^{7 \pi i}
\] 
\[
z = e^{\frac{\pi i }{4}} , e^{3 \pi \frac{i}{4}}, e^{\frac{5 \pi i }{4}}, e^{\frac{7 \pi i}{4}}
\] 
Now let's find the residues for each 
\[
\text{Res} = \frac{1}{4z^3} = \frac{z}{4 z^{4}} = - \frac{z}{4}
\] 
As $z^{4} = -1$. 

Now, choose the path. 

\begin{figure}[ht]
    \centering
    \incfig{residue-path-of-a-semicircle}
    \caption{Residue path of a semicircle}
    \label{fig:residue-path-of-a-semicircle}
\end{figure}

Now apply the residue theorem 
\[
	\frac{1}{2 \pi i } \int_{-R}^{R} + \int_{\cap } f(z) \mathrm{d} z = \sum \text{residues of 2 points inside the region} = \frac{- e^{\pi i \frac{1}{4}}}{4} - \frac{e^{\frac{3 \pi i }{4}}}{4} 
\]
Computation 
\[
\frac{\pi}{\sqrt{2} }
\]
So the integral is done. 

\[
f(z) = \frac{1}{z^{4} + 1} = \frac{\pi}{\sqrt{2} }
\] 

\[
	\int_{-\infty}^{\infty} \frac{\mathrm{d} x}{x^{6}+1}
\] 
The poles occur at 
\[
z^{6} = -1 = e^{\pi i } = e^{3 \pi i } = e^{5 \pi i } = e^{7 \pi i } = e^{9 \pi i } = e^{11 \pi i }
\] 
The roots are gotten by simply taking the $6$ th root.
We choose the path semicircle that is going to have 3 poles in the upper plane. These 3 poles are the upper plane existing roots of the number.

\[
\int \frac{\mathrm{d} z}{z^{6} +1} = \sum_{}^{} \text{Residues at 3 points}
\] 
\[
	\frac{1}{2 \pi i } \int_{-\infty}^{\infty} \frac{\mathrm{d} x}{x^{6} + 1}
\]
Residues
\[
\text{Res}(f, \cdot) = \frac{1}{6 z^{9}} = \frac{z}{6 z^{6}} = - \frac{z}{6}
\]
\[
	\frac{1}{2 \pi i} \int_{-\infty}^{\infty} \frac{\mathrm{d} x}{x^{6}+1} =
\frac{
-e^{\pi \frac{i}{6}} - i - e^{5 \pi \frac{i}{6}}
}{6} = - \frac{i}{3}
\] 
So we get
\[
	\int_{-\infty}^{\infty} \frac{\mathrm{d} x}{x^{6} +1} = \frac{2 \pi }{3}
\] 
\end{document}
