\documentclass[letter]{article}
\usepackage[monocolor]{ahsansabit}

\title{Computational Complex Analysis : : Class 18}
\author{Ahmed Saad Sabit, Rice University}
\date{\today}

\begin{document}
\maketitle
\[
\int_0^{\infty} \frac{\mathrm{d} x}{x^{4}+x^2+1}
\] 

\[
	\int_{-\infty}^{\infty} \frac{\cos a x}{x^2 + 1} \, \mathrm{d} x
\]

\[
\int_0^{\infty} \frac{\mathrm{d} x}{x^3 - i }
\] 
Residue theorem says, 
\[
\frac{1}{2 \pi i} \oint f(z) \mathrm{d} z = 
\sum_{}^{} \text{Residue inside } C
\] 

\section*{Illustration of a Property} 
One of the properties of residues,
\[
\text{Res}\left(fg' , z_0\right) = - \text{Res}(f'g, z_0)
\] 
I want to calculate the residue of 
\[
\text{Res} \left(\csc^3(z) , 0\right)
\]
We know that 
\[
\text{Res}\left(\csc (z) , 0 \right) = 1
\]

You get a homework assignment to find, 
\[
\text{Res} \left(\csc^{n}(z) , 0\right)
\] Here $n$ is odd otherwise we have an even function.

We will show how to go from first power to the third power. Similar technique for the homework problem. 
\[
\text{Res}\left(\csc^3\right) = \text{Res} \left(\csc \cdot \csc ^2\right)
\] We can leave out the origin, and $\csc $ is a derivative of something.
\[
\tan ' x = \sec ^2  x \quad \text{and} \quad \cot ' x = - \csc ^2 x 
\]
We can find, using the given mentioned formula, 
\[
 = \text{Res} \left(\csc \cdot \left(- \cot ' \right)\right) = \text{Res} \left(\csc ' \cot \right)
\]
\[
= \text{Res} \left(- \csc \cot \cot \right)
\]
\[
= \text{Res}\left(- \csc \cot ^2\right)
\]
Y`all knew the rule $\sec ^2 = 1 + \tan ^2 $
\[
= \text{Res}\left(- \csc ^3 + \csc \right) = - \text{Res}\left(\csc ^3\right) + \text{Res} \left(\csc\right)
\]
Left hand and right hand written together,
\[
\text{Res}\left(\csc ^3\right) = 
- \text{Res}\left(\csc ^3\right) + \text{Res} \left(\csc\right)
\]

We get 
\[
	2 \text{Res}\left(\csc ^3\right) = \text{Res}(\csc) = 1
\] 
Hence we get,
\[
\boxed{
\text{Res} \left(\csc ^3\right) = \frac{1}{2}
}
\] 
\section*{Calculations} 
\[
\int_0^{\infty} \frac{\mathrm{d} x}{x^{4}+x^2+1}
\] 

\[
f(z) = \frac{1}{z^{4} + z^2 + 1}
\] 
Our integral goes from two sides of infinity. That integral is bound from $0$ to $\infty$, so we just need half of the infinite as that's an even function
\[
	= \frac{1}{2}\int_{-\infty}^{\infty} \frac{\mathrm{d} x}{x^{4} + x^2+ 1}
\]
There are some poles in the semi-circle region we considered.
\[
\frac{1}{2 \pi i} \int_C f(z) \mathrm{d} z = \sum \text{Res}\left(f(z), \text{upper half plane}\right)
\]
An extra step, multiply both sides of the function with $z^2-1$
\[
f(z) = \frac{z^2 - 1}{z^{6} - 1}
\]
So the roots are 
\[
z^{6} = 1 = e^{2 n \pi i }
\] 
\[
z = e^{\frac{2n \pi i}{6}}
\] $n$ here goes from $0,1,2,3,4,5$. 
Testing over \verb| a \over b| 
\[
	{a \over b}
\] 
The roots in the upper half plane are
\[
e^{2 \pi i \frac{1}{6}} = e^{ 2 \pi i / 6}, e^{ 4 \pi i / 6}
\]
"How did we get the $e ^{3 \pi i \frac{1}{6}}$? Prof: By mistake". 

Now the integral over the ``path" (not ``region") is 
\[
\frac{1}{2 \pi i } \int_C f(z) \mathrm{d} z = \text{Res}\left(f , e^{ \pi i \frac{1}{3}}\right) + Res ( e^{ 2 \pi i  / 3})
\]

Now we need to check what $f$ is like over $R$ for $R\to \infty$, we need to have $f$ tends to zero hence the limit gives me a basically $0$.

\[
\text{Res} \left(\frac{z^2 - 1}{z^{6} - 1}, z_0\right) = \frac{z_0^2 - 1}{ 6 z_0 ^{5}} = \frac{z_0^3 - z_0}{6}
\] 
The $z_0$ are $e ^{ 2 \pi i \over 6}, e^{4 \pi i \over 6 }$

\[
	\frac{1}{2}\int_{-\infty}^{\infty} f(x) \mathrm{d} x= \frac{1}{2}2 \pi i 
	\left(
		{e^{\pi i } - e^{ \pi i \over 3}  
			\over
		6} 
		+
		{e^{2 \pi i} - e^{2 \pi i \over 3} 
			\over 
		6} 
	\right)= 
	\boxed{{
	\pi \over 2 \sqrt{3}  }
	}
\]

\section*{Another Computation} 
\[
\boxed{
\int_{-\infty}^{\infty} \frac{\cos a x}{ x^2 + 1} \mathrm{d}  x}
\]
Here's a function of $z$ and here 
\[
f(z) = \frac{\cos az}{z^2 + 1}
\] 
We cannot use $a = i \alpha$ because that blows up. Hence we change it a bit, 
\[
f(z) = {
e ^{i a z} \over z^2 + 1}
\]
We have to pick the real portion for $\cos a z$. Here $a \in \mathbb{R}$. Investigate what happens when for the semicircle $R\to \infty$. Now
\[|
e^{i a z}| = e^{
\text{Re} \left(i a z\right)} = e^{\text{Re} \left(i a x - ay\right)} 
 = 
 e^{ - a y }
\]
Now we are required to say $a$ is $a >0$ otherwise we have something blowing up. Hence 
\[
a\ge 0
\] 
Back to the function now, 
\[
\left| f(z) \right| = 
\left| 
\frac{e^{i a z}}{z^2 + 1 }
\right| 
\]
Now we need, where the singularities happen? Well at $z = i$ for upper half plane. 
\[
\text{Res} \left(\frac{e^{i a z}}{z^2 + 1}, i\right) ={  e^{-a} \over 2 i }
\] 
For $R \to  \infty$, the residue theorem
\[
\frac{1}{2 \pi i } \int_{\mathbb{R}} {e^{i a x} \over x^2  +1} \mathrm{d} x ={ e^{-a} \over 2i}
\] 
\[
\boxed{
	\int_{-\infty}^{\infty} \frac{e^{i a x}}{x^2 + 1} = \pi e^{-a} \qquad (a\ge 0)
}
\] 
\[
R \quad \mathbf R \quad \pmb R
\] 
Question: Is it practical to solve for definite integrals given we knew how the function behaved at finite $\pmb R$?
\end{document}
