\documentclass[letter]{article}
\usepackage[monocolor]{ahsansabit}

\title{Computational Complex Analysis}
\author{Ahmed Saad Sabit, Rice University}
\date{\today}

\begin{document}
\maketitle

I entered the class a little late because I couldn't find where the room was. By the time I was there, the professor had shown \[
	\overline{zw} = \overline{z} \, \overline{w}
	\] 
I am re-reading the materials a few days later and I will keep the theoretical notes here. In the rare case that anyone is reading this instead, I think you should just read the book because what I type here is going to be quite hazy. I will try to improve quality of my handouts after I write some bad ones. One way this handout might be very useful to you is when I add my ideas why we are doing something (for example like why do we need to prove $z_2 = 0$ if $z_1 z_2 = 0$ given $ z_1 \neq  0$, isn't it just common sense?)

\section{General Definitions}
\df{
Let's define a cute number called $i$ which has this specific property, 
\[
i^2 = -1
\] Which conversely means, \[
\sqrt{-1}  = i
\] 
}
Let's play around with this number for a bit, 
\begin{align*}
	i &= i \\
	i^2 &= -1 \\
	i^3 &= -i \\
	i^{4} &= 1 \\
	i^{5} &= i \\
	i^{6} &= -1 \\
	i^{7} &= -i \\
	i ^{8} &= 1 
.\end{align*}
There is a repeating pattern here. 

\df{
Complex numbers to me is basically a pair of numbers that are $(x,y): x,y \in \mathbb{R}$ and they follow a nice rule of addition and multiplication. 
\[
	(x_1, y_1) + (x_2, y_2) = (x_1 + x_2, y_1 + y_2)
\] 
\[
	(x_1,y_1) * (x_2, y_2) = (x_1 x_2 - y_1 y_2, x_1 y_2 + x_2 y_1)		
\] 
A complex number can be written in the form $x+ iy$ where $i^2 = -1$. These numbers are member of the $\mathbb{C}$ complex numbers. 
}

\pf{
The addition is trivial, for multiplication we will use the complex number form, $z = x + iy$ and $i^2 = -1$ to get,
\[
	\left(x_1 + i y_1\right)\left(x_2 + i y_2\right) = x_1 x_2 + i x_1 y_2 + i x_2 y_1 + i^2 y_1 y_2		
\] 
This gives us, 
\[
\boxed{
	\left(x_1 + i y_1\right)\left(x_2 + i y_2\right) =
x_1 x_2 - y_1 y_2 + i(x_1 y_2 + x_2 y_1)
}
\] 
}

An interesting thing to note is that setting $y_i$ to $0$ makes the operations simply $\mathbb{R}$ like. A logical way to think about this new type of $\mathbb{C}$ number is that this should have rules for $\sqrt{x  + iy} $ and like $\left(x+iy\right)^2$. We can find each operations on complex numbers one by one, plus we can even have operations that are unique to complex numbers.

Before getting in too theoretical let's try to do a cool problem using Binomial Theorem

\pr{
Compute \[
	(1 - i)^{4}
\] 
}
\solu{Now I can simply multiply the term to itself for 4 times, or I can act pro and use binomial theorem. Anything works to be honest.

Using the following,
	\[(a+b)^{n} = \frac{n!}{j! (n-j)!} a^{j} b^{n-j}\]
	We can get, 
	\[
		(1 + b)^{4} = \frac{4!}{0! 4!} b^{4} 
		+ \frac{4!}{1! 3!} b^3 
		+ \frac{4!}{2! 2!} b^2
		+ \frac{4!}{3! 1!} b
		+ \frac{4!}{4! 0!} 
	\]
	\[
	= b^{4} + 4 b^3 + 6 b^2 + 4 b  + 1
	\] If we have $b = -i$, then, 
	\[
	 = 1 - 4(-i) + 6(i)^2 +4(-i) + 1 = -4
	\]
	We got $(1-i)^{4} = -4$. 
}

Now of course we have the additive inverse (fun fact in Russian this is called Prativa-Palozhni Elemi-niyum) which is $z + (-z) = 0$. But do we have multiplicative inverse? 

It should follow the rule $z z^{-1} = 1$. Turns out the next definition can help us, 
\df{
If $z = x + iy$ is a complex number, we call the \textbf{Complex Conjugate }of $z$ to be,
\[
z^{*} = x - iy
\] 
We just reverse the sign of $iy$ term. 
}

This comes in tremendously helpful while finding the inverse of $z$.

\pr{
Find the inverse of $z = x + iy$
}
\solu{
\[
z = x + iy = \left(x  +iy\right) \frac{x - iy}{x - iy} = \frac{x^2 + y^2}{x - iy}  
\] 
If we define $\sqrt{x^2 + y^2}$ to be the \textbf{Modulus of} $z$, then we can easily tell,
\[
z = \frac{|z|^2}{z^{*}}
\] Here $|z|$ is a purely real number. So the nice thing is, having $z^{*}$ down in the numerator is a successful complex number, which being upside down is meaning that we have an inverse. 
\[
z \cdot \frac{z^{*}}{|z|^2} = 1
\] This is simply, 
\[
	(x + iy) \cdot 
	\left(\frac{x}{x^2 + y^2} - i \frac{y}{x^2 + y^2}\right) = 1
\] Hence the inverse is, 
\[
\frac{x}{x^2 + y^2} - i \frac{y}{x^2 + y^2} = z^{-1}
\] }


\pr{
Prove that $\mathbb{C}$ is a field.
}
\solu{
Given if $\overline{z}$ exists, and none here is a zero, we can have a field. We need a multiplicative inverse.
\[
{z} = \frac{\overline{z}}{ z \overline{z}} = \frac{\overline{z}}{|z|^2}
\] 
And because it's already known that $-z$ exists such that $z + (-z) = 0$, hence what we have here is a field.
}

The interesting thing is, our basic ideas of numbers that if $ab = 0$ and $a \neq 0$, then $b$ must be zero, such ideas needs to be verified. Let's go through the next problem to verify this for complex numbers, 
\pr{
Prove $z_2 = 0$ if $z_1 \neq 0$ while $z_1 z_2 = 0$
}
\solu{
\[
z_2 = z_2 \cdot 1 = z_2 \left(z_1 z_1^{-1}\right) = (z_1 z_2) z_1^{-1} = 0 \cdot z_1^{-1} = 0
\] 
If $z_1$ not zero, then neither is $z_1^{-1}$. So from here the proof is complete.
}


Now using the complex calculation, we find out that,
\[
|z+w|^2 \le |z|^2 + |w|^2 + 2 \Re(z \overline{w})
\]
This relation above can be found out by using $z = x + iy$ and $w = x' + i y'$. 
For a complex number $z = x + iy$ the $\Re(z) = x$. We can use triangle inequality,
\[
|z+w|^2 \le |z|^2 + 2 |z \overline{w} | + |w|^2
\] 
From this we can show, 
\[
|z+w|^2 \le (|z| + |w| )^2
\] 
A useful proof that comes out handy while doing this calculation,
\[
| z w | ^2 = (zw) ( \overline{zw}) = zw \overline{z} \, \overline{w} = |z|^2 |w|^2
\] 
Talked about the Polar Format of complex number. Here it's mentionable that $\theta$ is not unique because $\theta$ is same as $\theta + 2\pi $. $|z|$ is well defined, $\theta$ is argument.

Using the definition of exponentials in series, \[
	e^{x} = 1 + x + \frac{x^2}{2!} + \frac{x^3}{3!} + \cdots = \sum_{n=1}^{\infty} \frac{x^{n}}{n!}
\] 
This can be helpful to prove the following problem,
\pr{
Prove that \[
e^{z+w} = e^{z} e^{w}
\] 
}
\solu{
Because we will use the binomial theorem, let me state this before, 
\[
	(a+b)^{n} = \frac{n!}{j! (n-j)!} a^{j} b^{n-j}
\] 
Now, starting with the series, 
\begin{align*} 
	e^{z+w} &= \sum_{n=0}^{\infty} \frac{(z+w)^{n}}{n!} \\
	&= \sum_{n=0}^{\infty} \frac{1}{n!} \left( \sum_{j=0}^{n} \frac{n!}{j! (n-j)! } z^{j} w^{n-j}\right)  \\
	&= \sum_{n=0}^{\infty} \sum_{j=0}^{n} \frac{z^{j} w^{n-j}}{j! (n-j)!} \\
\end{align*}
It must have something to do with expanding $(z+w)^{n}$ using the binomial theorem. If you look at the equation you might feel stuck but now we are going to use Cauchy's Product of Infinite series to the rescue, 
\[
	\left(\sum_{i=0}^{\infty} a_i\right)
	\left(\sum_{j=0}^{\infty} b_j\right) =
	\sum_{k = 0}^{\infty} \left(\sum_{l=0}^{k} a_l b_{(k-l)}\right)
\] 
Using this we can get, 
\[
= \left(\sum_{j = 0}^{\infty} z^{j}/j!\right) \left(\sum_{n=0}^{\infty} w^{n}/n!\right)
= \boxed{
	e^{z} e^{w}
}
\] 
}
\end{document}
