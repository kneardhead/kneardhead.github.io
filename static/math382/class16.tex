\documentclass[letter]{article}
\usepackage[monocolor]{ahsansabit}

\title{Computational Complex Analysis : : Class 16}
\author{Ahmed Saad Sabit, Rice University}
\date{\today}

\begin{document}
\maketitle

\[
f(z ) = 
\ldots 
+ 
\frac{c_{-2} }{(z-z_0)^2} + \frac{c_{-1}}{z-z_0} + c_0 + c_1 (z-z_0)
\] 
\[
\text{Res}(f,z_0)
\] also it equals $\frac{1}{2\pi i} \int_{C} f(z) \mathrm{d} z$. Also it equals the unique $a \in  \mathbb{C}$ such that $f(z) - \frac{a}{z-z_0}$. 

\[
\text{Res} \left(
	(z-z_0)^{n}, z_0 
\right) = 1
\]
If $n = -1$ otherwise $0$. 
\[
C_1 \text{Res} \left(e^{\frac{1}{z}} , 0\right) = 1
\]
Simple pole at $z_0$ 
\[
f(z) = \frac{c_a}{z-z_0} + c_0 + c_1(z-z_0) + \cdots
\]

Suppose our function can be written as a quotient: 
\[
f(z) = \frac{a(z)}{b(z)} 
\] And $a,b$ is holomorphic near $z_0$ and $b$ has a simple zero at $z_0$. \[
b'(z_0) \neq  0 
\] 
\[
\text{Res}(f(z), z_0) = \lim_{z \to z_0} \frac{(z-z_0) a(z)}{b(z)} = \lim_{z \to z_0} \frac{a(z)}{\frac{b(z)-b(z_0)}{z-z_0}} = \frac{a(z_0)}{b'(z_0)}
\] 

\end{document}
