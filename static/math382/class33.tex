\documentclass[letter]{article}
\usepackage[monocolor]{ahsansabit}

\title{}
\author{Ahmed Saad Sabit, Rice University}
\date{\today}

\begin{document}
\maketitle
The corollary is if also $f$ is conformal transformation of $D$ with conformal inverse $f^{-1}$, then $f(z) = az$ and $|a| = 1$.
Schwarz Lemma is that $|f(z)| \le 1$ and $f(0) = 0$ then $|f(z)| \le |z|$ and $|f'(0)| \le  1$.

Apply Schwarz lemma to $f$ and to $f^{-1}$. 
\[
|f(z)| = |z| 
\]
\[
|(f^{-1})' (0) | \le 1
\] 
\[
	(f f')'(0) = 1
\] 
\[
\therefore |f'(0)| = 1
\] 
\[
\therefore f(z) = az
\]

Theorem we tried to see in Friday, a conformal transformation of $D$ onto $D$ have the form, 
\[
f(z) = \omega \phi_a(z)
\] 
\[
	\phi_a (z) = \frac{z-a}{1- \overline{a} z}
\]
\[
|w| = 1 \quad |a| < 1
\]
Proof is that suppose $f$ is a conformal transformation of $D$ then $f(0) = a$. 

\df{
A $C^{2}$ function on $\mathbb{C}$ is said harmonic if it satisfies the Laplace's Equation. 
\[
	u_{x x } + u_{y y } = 0
\]
For us, the prime examples are real parts and imaginary parts of Holomorphic functions. 
\[
u_x = v_y
\] 
\[
u_y = - v_x
\] 

}
\end{document}
