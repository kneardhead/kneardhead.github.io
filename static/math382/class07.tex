\documentclass[letter]{article}
\usepackage[monocolor]{ahsansabit}

\title{Computational Complex Analysis : : Class 07}
\author{Ahmed Saad Sabit, Rice University}
\date{\today}

\begin{document}
	\maketitle
	 Let's start with a power series 
	 \[
	 f(z) = \sum_{n=0}^{\infty} a_n (z - z_0)^{n}
	 \] 
	 And this has $f(z_0)=a_0$. And $\sum_{n=0}^{\infty} a_n 0^{n} = a_0 (0)^{n}.$ 
	 A good way of writing the series is instead $a_0 + \sum_{n=1}^{\infty} a_n(z-z_0)^{n}$. 
	 Let's have a derivative, 
	 \[
	 f'(z) = \sum_{n=1}^{\infty} na_n (z-z_0)^{n-1}
	 \] 
	 \[
	 f''(z) = \sum_{n=2}^{\infty} n(n-1)a_n (z-z_0)^{n-2}
	 \] 
	 \[
	 f^{(k)}(z) = \sum_{n=k}^{\infty} n (n-1) (\cdots) (n-k+1) a_n (z-z_0)^{n-k}
	 \]
	 \[
	 f^{(k)}(z_0) = k (k-1)(\cdots)(2)(1) a_k
	 \] 
	 \[
		 a_k = \frac{f^{(k)}(z_0)}{k!}
	 \] 
	 The taylor series of $f$ centered at $z_0$ is
	 \[
	 \sum_{n=0}^{\infty} \frac{f^{(k)}(z_0)}{k!}(z-z_0)^{k}
	 \] 

\thm{
	Suppose $f(z)$ power series $\sum_{n=0}^{\infty} a_n (z-z_0)^{n}$ with $R>0$, suppose $\exists $ sequence of points $\in \mathbb{C}$ converging to $z_0$ so that $f() = 0$ at every one of these points.	Then of course $f(z_0)=0$ and $f=0$ $\forall |z-z_0|<R$. 
}
\pf{
	Proof by contradiction, suppose $f=0$ is false, there will be smallest $k$ such that $a_k\neq 0$. \[
		f(z) = \sum_{k=0}^{\infty} a_{k+n}(z-z_0)^{k+n} = a_k (z-z_0)^{k} + \sum_{n=1}^{\infty} (a_{k+n}/a_k )(z-z_0)^{k+n}
	\] 
	\[
	a_k (z-z_0)^{k} \left[
	1 + \sum_{n=1}^{\infty} \frac{a_{k+n}}{a_k}(z-z_0)^{n}\right]
	\]
	From continuity we cannot have a zero series shown in the third bracket. 
}

\begin{figure}[ht]
    \centering
    \incfig{converging-to-a-center-point-for-a-series}
    \caption{converging to a center point for a series}
    \label{fig:converging-to-a-center-point-for-a-series}
\end{figure}

	\section{Changing Center of Convergence} 
\begin{figure}[h]
    \centering
    \incfig{changing-the-center-of-convergence}
    \caption{changing the center of convergence}
    \label{fig:changing-the-center-of-convergence}
\end{figure}
	\[
	\frac{1}{1-z} = \sum_{n=0}^{\infty} z^{n}
	\] 
	\[
	\frac{1}{1-z} = \frac{1}{\frac{2}{3} - z + \frac{1}{3}} = \frac{3}{1 - \left(3z - 2\right)}
	\] 
	This can be written exactly like a summation,
	\[
	= 3 \sum_{n=0}^{\infty} (3z - 2)^{n}
	\]
	Setting the point of convergence at $-\frac{2}{3}$. 
	\[
	\frac{1}{1-z} = \frac{1}{\frac{1}{3} - (z- \frac{2}{3})}
	\]\[
	=\frac{1}{3} \sum_{n=0}^{\infty} \left(z- \frac{2}{3}\right)^{n}
	\]  

\thm{Let $D$ be an open connected subset of the complex plane $\mathbb{C}$. 
	Suppose $f$ is defined on $D$ and is differentiable on every point of $D$. Suppose (radical assumption) the derivative is always zero for all on $D$. The conclusion is, $f$ is constant. 
	}
\pf{
	$\frac{\partial f}{\partial x}$, $\frac{\partial f}{\partial y}$
}	

\section{Logarithm}
	Consider the function \[
	f(z) = \log(1-z)
	\] near $z=0$, log is differentiable but an argument is needed. If $z\to 1$, we will be near where $\log 0$  might appear. So unit disk, use principle value of $\log (1-z)$. If the first disk was centered at $z=0$, we have another disk at $z=1$. We can use the principle argument for $1-z$. This is differentiable for $|z| <1$. 
	\[
		\frac{\mathrm{d} }{\mathrm{d} z} \log(1-z) =\frac{-1}{1-z} = - \sum_{n=0}^{\infty} z^{n}
	\]
	Try to integrate, 
	\[
	\log(1-z)=-\sum_{n=0}^{\infty} \frac{z^{n+1}}{n+1} + C
	\] 
	We can see $C = 0$, $\impliedby z=0$. 
	\[
	\log(1-z)=-\sum_{n=1}^{\infty} \frac{z^{n}}{n}
	\]
	Radius of convergence of $\sum_{n=1}^{\infty} \frac{z^{n}}{n} = 1$. $z\to 1$ we get harmonic series that tries to blow up.
	\[
	\ln 2 = - \sum_{n=1}^{\infty} \frac{(-1)^{n}}{n} = \sum_{n=1}^{\infty} \frac{(-1)^{n-1}}{n}=1-\frac{1}{2}+\frac{1}{4}-\cdots
	\] The series and the equation are both valid if $|z|<1$ except for $z = 1$.

	\subsection{Regarding next class "Friday"}
	\begin{itemize}
		\item Line Integral (212 Ref)
		\item Green's Theorem 
		\item Amazing Results (Monday), for instance Cauchy's Integral Theorem. I need to read at home lol. 
	\end{itemize} Frank Jones: ``I can't resist to say what the result is". 
\end{document}
