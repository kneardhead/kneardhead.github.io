\documentclass[letter]{article}
\usepackage[monocolor]{ahsansabit}

\title{Computational Complex Analysis : : Class 15}
\author{Ahmed Saad Sabit, Rice University}
\date{\today}

\renewcommand{\frac}{\dfrac}
\begin{document}
\maketitle

Given some function which is undefined at a point $z_0$, we sometimes call $z_0$ a singularity of the function. We are now going to look at Holomorphic functions $f$ with a singularity at $z_0$. We are going to define around $z_0$ but not $z_0$. And specifically we want to discuss isolated singularities. And that means 

\begin{figure}[ht]
    \centering
    \incfig{function-has-to-be-holomorphic-around-but-not-at-the-point}
    \caption{function has to be holomorphic around but not at the point}
    \label{fig:function-has-to-be-holomorphic-around-but-not-at-the-point}
\end{figure}
 
Say $z = \frac{1}{n \pi }$. Then $\sin \frac{1}{z} = 0$. Now look at $\frac{1}{\sin \frac{1}{z}}$. Every single point other than the origin then becomes an isolated point that is a singularity.

Now we are going to classify the isolated singularities of holomorphic functions. So we will see that there are three classes. Suppose $z_0$ is an isolated singularity of $f$. Here's every where the function is holomorphic other than $z_0$. There is a unique Laurent's Expansion of $f$ in a neighborhood of $z_0$.

\[
f(z) = \sum_{n=-\infty}^{\infty} c_n (z-z_0)^{n}
\]
This converges for $0 < | z -z_0 | < r$. 
\[
c_n = \frac{1}{2\pi i} \int_C \frac{f(z_0 + s)}{s^{n+1}} \mathrm{d} s
\] 

Case 01: $c_n$ with $n<0$ are 0. 
\[
f(z) = c_0 + c_1 (z-z_0)+ \ldots
\] 

Riemann's Removable singularity theorem, if the modulus of $f$ is bounded near $z_0$ then removable. 

Case 02: At least one $c_n$ is not $0$ with $n < 0$. In fact only finitely many $c_n$ are of this nature. 

\[
	f(z) = c_{-m} / (z-z_0)^{m} + \cdots + c_0 + c_1(z-z_0) + \cdots 
\]
What is the limit of $|f(z)|$ as $z \to z_0$? Well it goes to infinity because $\frac{C_m}{(z-z_0)^{m}}$ blows up to infinity. This is called a Pole.

\begin{figure}[ht]
    \centering
    \incfig{a-visual-representation-of-a-pole}
    \caption{A visual representation of a pole}
    \label{fig:a-visual-representation-of-a-pole}
\end{figure}

Case 03: Infinitely many $c_n$ with $n<0$ are not zero. This is called ``Essential Singularity".

\section*{Casorati Weierstrass Theorem} 
\thm{Suppose $z_0$ is an essential singularity of $f$. Suppose $w$ is any complex number or $\infty$. Then the conclusion is there exists a sequence
\[
\zeta_1, \zeta_2, \zeta_3, \ldots, \zeta_k
\]
Here as $\lim_{k \to \infty} |\zeta_k| = z_0$ and \[
\lim_{k \to \infty} f(\zeta_k) = w
\]
If you think of $e^{\frac{1}{z}}$ then 
}

Case 01: $w = \infty$. By contradiction assume there is such sequence. Then $f(z)$ has the property that it cannot have very large as  $z \to  z_0$. 
\[
\exists C > 0 \text{ such that } |f(z) | < C \text{ for all $z$ close to $z_0$}
\]
From Riemann's Removable Singularity Theorem, $z_0$ is a removable singularity for $f$.

Case 02: $w \in  \mathbb{C}$. Again, if no such sequence exists, then there is a small disk containing $w$ which is never reached by any value of $f(z)$ near $z_0$. 

\begin{figure}[ht]
    \centering
    \incfig{z-plane-and-w-plane}
   \caption{z plane and w plane}
    \label{fig:z-plane-and-w-plane}
\end{figure}
For $|f(z)-w| \ge a$ for all $|z-z_0| < r$, 
\[
	\left|
\frac{1}{f(z) - w} \right| \frac{1}{a}
\]
$z_0$ is removable singularity $\frac{1}{f(z) - w'}$. 

This function extends 
\[
\frac{1}{f(z) - w} 
\] is $g(z)$. $g$ is holomorphic so it maybe $0$ at $z_0$. Factor out $(z-z_0)^{m}$ from $g$ : 
\[
g(z) = (z-z_0)^{m} h(z)
\] 
And 
\[
h(z_0) \neq  0 
\] 
\[
\frac{1}{f(z) - w} = (z - z_0)^{m} h(z)
\] 
\[
f(z) - w = (z - z_0)^{-m} \frac{1}{h(z)}
\] 
Laurent Expansion. It's a pole. Contradiction the essential singularity of $f$ at $z_0$. We did all the work to prove it wasn't an essential singularity instead it was a pole.

We are about to start a huge transition: Look at holomorphic function with isolated singularity at $z_0$ and look at the Laurant Expansion of 
\[
	f(z) = \cdots c_{-2} (z-z_0)^{-2} + c_{-1} (z-z_0)^{-1} + c_0 + c_1 (z-z_0) + c_2 (z-z_0)^2 + \cdots 
\]
$c_{-1}$ is important and it's called the  residue of $f$ at $z_0$. 
\[
\boxed{
	c_{-1}
}
\] 
What we are going to use this for is called \textsf{Residue Theory}. 
\end{document}
