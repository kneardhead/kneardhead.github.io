\documentclass[letter]{article}
\usepackage[monocolor]{ahsansabit}

\title{}
\author{Ahmed Saad Sabit, Rice University}
\date{\today}

\begin{document}\maketitle
$z$ plane can have a point $z$ and look at a direction around the point in $z$. So, $z + th$ can be a direction around $z$. Here $h \in \mathbb{C}, t \in \mathbb{R}$. We will get a linear map to $w$ where $w = f(z)$ and $f'(z) = Re^{i\theta}$.
\[
f(z+h) = f(z) + R e^{i\theta}h + \ldots
\]
We are going to look at $f(z+th)$ so that $t$ is small and 
\[
f(z) + t R e^{i\theta} h
\]
One arrow in $z$ is also another arrow in $w$. Here the direction is going to be $R e^{i\theta}h$. The angles are going to be the same. 

\section{Review}
Let's have a region $D$ that might have holes in it. Then the line integral around $\gamma$ is 
\[
\int_\gamma f(z) \mathrm{d} z = 0
\] For all loops $\gamma$ if and only if $f$ is differentiable. And is continuous, that is Green's theorems. The  Cauchy Integral theorem.

\section{Cauchy Integral Formula} 
Assume the same thing that $f$ is differentiable at every point of $D$ and $f'$ is continuous. Here's a daring move - let $z_0$ be a point $z_0 \in D$ that is not going to change. And try to apply a Cauchy integral theorem to $\frac{f(z)}{z-z_0}$, everywhere other than $z_0$. And see what Cauchy could do with it. What might I do to modify the situation? 

We can't have $z_0$ in the area of $D$ because the integral is going to be undefined while we are taking the integral at $z_0$. So $z_0$ being at the interior, we imagine a disk of radius $\epsilon$ around $z_0$ that is going to be deleted.

We obtain 
\[
	\int_{\partial D_\epsilon} \frac{f(z)}{z-z_0}\mathrm{d} z = 0
\]
And $D_{\epsilon} = D \setminus \text{safety disk}$. 
\[
	\int_{\partial D} \frac{f(z)}{z-z_0} \mathrm{d} z + 
	\int_{\partial \text{ safety disk (clock-wise)}} \frac{f(z)}{z-z_0} \mathrm{d} z = 0
\] 
\[
	\int_{\partial D} \frac{f(z)}{z-z_0} = \int_{\partial \text{safety disk, CCW}} \frac{f(z)}{z-z_0} \mathrm{d} z
\] 
Parametrize $z = z_0 + \epsilon e^{i \theta}$. Here $0 \le \theta \le 2\pi $. 

\[
	\int_{\partial D} \frac{f(z)}{z-z_0} \mathrm{d} z = 
i	\int_0^{2\pi }  f(z_0 + \epsilon e^{i \theta} )\mathrm{d} \theta
\] Left hand side does not depend on $\epsilon$. In spite of what it looks like. 
\[
\lim_{\epsilon \to 0} i \int_0^{2\ \pi } f (z_0) \mathrm{d} \theta = f(z_0) 2\pi i
\] 
\[
\boxed{
	f(z_0) = \frac{1}{2 \pi i} \int_{\partial D} \frac{f(z)}{z-z_0} \mathrm{d} z
}
\] 
Cauchy integral formula, same assumptions on $D$ and $f$ having $f$ being continuous on $D \cup \partial D$ $f'(z) $ exists for all $z \in D$.
\[
	f(z_0) = \frac{1}{2 \pi i} \int_{\partial D} \frac{f(z)}{z-z_0} \mathrm{d} z
\]
Here $z$ becomes the dummy variable and use $s$ instead. 
\[
	f(z) = \frac{1}{2 \pi i} \int_{\partial D} \frac{f(s)}{s-z} \mathrm{d} s
\]
Observation is we have differentiate both sides with respect to $z$ 
\[
	f'(z) = \frac{1}{2\pi i} \int_{\partial D} f(s) 
	\frac{\mathrm{d} }{\mathrm{d} z} \left(\frac{1}{s-z}\right)\mathrm{d} s
\]
\[
	= \frac{1}{2 \pi i} \int_{\partial D} f(s) \frac{1}{(s-z)^2}\mathrm{d} s
\]
\[
	f^{(n)}(z) = \frac{n!}{2 \pi i} \int_{\partial D} f(s) \frac{\mathrm{d} s}{(s-z)^{n+1}}
\]
The math looks cute kintu kisu toh bujtesina :)

Line integral is path independent if it has an anti-derivative. Wait what. 

Morera's Theorem follows
\thm{
For $f$ continuous on D and suppose and line integral of $f $ along loops in $D$ are always $0$ implies that $f'$ exists at every point. 
}
\pf{
We already know that $\forall $ differentiable $F$ there is $f = F'$. But now we know that $F''$ exists, hence $f'$ exists. 
}

\df{
In older days this was called Analytic. 

Important Terminology, a function defined on an open set $D \subset \mathbb{C}$ which is differentiable at every point of $D$ is called Holomorphic. 
}

\df{
A function $f$ is said to be analytic if near endpoint $z_0$ of it's domain, it equals a series 
\[
f(z) = \sum_{n=0}^{\infty} a_n (z-z_0)^{n}
\] With positive radius of convergence.:q:q

}

\end{document}
