\documentclass[letter]{article}
\usepackage[monocolor]{ahsansabit}

\title{Computational Complex Analysis : : Class 22}
\author{Ahmed Saad Sabit, Rice University}
\date{\today}

\begin{document}
\maketitle
\section*{Counting Theorem}

Set we have a nice curve with a boundary region. And we have a holomorophic function $f $ with some isolated singularities, namely Poles. We want to count the number of zeroes and number of poles of $f$. We assume we can use the residue theorem and on the curve we don't have $f$ being infinite or zero. 

Now we apply the residue theorem not to $f$ itself but to $f'(z) / f(z)$. Examine 
\[
\int_C \frac{f'(z)}{f(z)} \mathrm{d} z
\] 
What are the singularities inside? Wherever $f$ is zero we have singularity. Wherever $f$ is infinite we also have it same. Examine singularities of $f'/f$.

Let's say $f(z_0) = 0$. Now near $z_0$, 
\[
f(z) = (z-z_0)^{m} g(z)
\] 
so that $g(z_0) \neq  0$. 
\[
	(z-z_0)^{m} \left(
a_0 + a_1 (z-z_0)^{1} + a_2 (z-z_0)^{2} + \cdots
	\right)
\]
Let's compute our quotient 
\[
f'/f = \frac{(z-z_0)^{m} g'(z) + m (z-z_0)^{m-1} g(z)}{(z-z_0)^{m} g(z)}
\]
\[
 = \frac{g'(z)}{g(z)} + \frac{m g(z)}{g(z)} \frac{1}{z-z_0} + \cdots
\] 
So the residue of $\frac{f'}{f}$ at $z_0$ is equal to 
\[ 
	\text{Res} \left(\frac{f'}{f} , z_0\right) = m
\]

Now let's apply the same thing for a pole. 
\[
f(z) = \frac{C}{(z-z_0)^{m}} + \cdots
\]
\[
= \frac{1}{(z-z_0)^{m} } g(z) 
\]
\[
g(z_0) = C
\] 
\[
\log f = -n \log(z-z_0)g + (z-z_0)^{-n} \log g
\] 
\[
\frac{f'}{f} = \frac{-ng }{z-z_0} + \cdots
\]

Residue Theorems 
\[
\frac{1}{2 \pi i } \int_C \frac{f'(z)}{f(z)} \mathrm{d} z = \text{ total number of zeroes} - \text{ total number of poles}
\]

Let's consider a polynomial $P(z)$ with order $n\ge 1$ and consider the curve to be a circle. If it has zeroes it has finitely many so we just need to draw a big radius. 

\[
\int_C \frac{1}{2 \pi i } \frac{P'(z)}{P(z)} \mathrm{d} z = \mathcal Z
\] 
Rough estimates $P_z$ will have leading term $cz^{n} + \text{lower order}$ if $C \neq  0$. Then $P'(z) = nc z^{ n-1}$. 
\[
P'/P = \frac{ncz^{n-1} (1+\ldots)}{ cz^{n} (1+\ldots)} = \frac{n}{z} + \cdots 
\]
\[
\int P' / P \mathrm{d} z = \int \frac{n}{z} \mathrm{d} z = n \frac{2 \pi i }{R^{n-1}}
\] 
\[
P(z) \text{ has exactly } n \text{ zeroes in } \mathbb{C}
\]

\section*{Rouche's Theorem} 
Back to the general situation. $f$ is defined to be a holomorphic as before and $g$ is also defined holomorphic as before. But assume $|g(z)| < |f(z) | $ on $\mathbb{C}$. $f\neq 0 $ on $\mathbb{C}$ and $f + g \neq  0 $ on $\mathbb{C}$.

The two functions $f(z)$ and $f(z) + g(z)$ have the same value of number of zeroes minus the number of poles inside $\mathbb{C}$. 

To prove the theorem we need to show 
\[
\int f'/f \mathrm{d} z = \int \frac{f' + g'}{f + g}
\]  
\[
\int_C \left(
f'/f - \frac{f' + g'}{f+ g}
\right)
\] 
Take common denominator
\[
= \frac{f' (f+g) - f(f'+g') }{f(f+g)} = 
\frac{f'g - fg' }{f(f+g)} 
= 
\frac{ - \left(\frac{g}{f}\right)'}{f(f+g)}
\]
$h = g / f$ gives $ = \frac{- h' }{1  + h}$. So we now have to see $h$ values on $\mathbb{C}$. 
$|h| < 1$ from the starting point. We need to prove $\int - \frac{h'}{1+h} \mathrm{d} z$. 

$\log h$ is well defined because it's argument $- \frac{\pi}{2} + \frac{\pi}{2}$ and now 
$\log 1+h$ has derivative $h'/1+h$. 
\[
\int_C \frac{d}{dz} \log( 1+ h) = 0
\]
This can be used to solve the Fundamental Theorem of Algebra here again. 

\[
P(z) = cz^{n } + \cdots
\] 
\[
f(z) = cz^{n}
\] 
And this dominates the  lower order term on the big circle. 
\end{document}
