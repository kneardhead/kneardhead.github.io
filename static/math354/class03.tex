\documentclass[letter]{article}
\usepackage[monocolor]{ahsansabit}

\title{Honors Linear Algebra : : Class 03}
\author{Ahmed Saad Sabit, Rice University}
\date{\today}

\begin{document}
\maketitle 

\section{Finite Dimensional Vector Spaces}
Finitely many vectors spanning over a space makes the Vector Space Finite dimensional. 
\df{
Spanning Set:
}

	\thm{Linear Dependence Lemma:
	Suppose $v_1, \ldots, v_m$ is a linearly dependent list of vectors in the vector space $V$. Then there exists $v_k$ such that $v_k \in span(v_1, \ldots, v_n)$, and $span(v_1, \ldots,v_m) = span(v_1, \ldots, v_m, \text{ without having } v_k)$
	}

	\thm{
	Then length of any linearly independent list is less than or equal to the length of any spanning list.			
	}

	\pf{
	Let's prove it through induction. Namely, suppose linearly independent list of $m$ vectors and spanning list of $n$ vectors, assuming a contradiction $n<m$. 
	\[
	n = 4, m=5
	\] Linearly independent vectors $v_1,v,2,v_3,v_4,v_5$. Spanning vectors $w_1,w_2,w_3,w_4$. Technique is we adjoin a vector $u_1$ into $w_1, w_2,w_3,w_4$. We can can remove $w_4$ and the system will still span, $u_1, w_1, w_2, w_3$ by Lemma. $u_1, u_2, u_3, w_1$ still spans.  
	}

	\df{
	Let $V$ be a finite dimensional vector space. A basis for $V$ is a list which is both linearly independent and spanning.
	}
	An observation is in this case the length of the basis for $V$ is independent of the choice of the basis. Then length is called the dimension of $V$. Examples of dimension: 
	\begin{table}[htpb]
		\centering
		\label{tab:label}
		\begin{tabular}{c|c}
			Vector Space & Dimension \\
			$\mathbb{R}^{n}$ & n \\
			$\mathbb{C}^{n}$ & n \\ 
			$\mathbb{P}_n$ & $n+1$  \\
			$V \oplus W$ & $\dim(V)+\dim(W)$
		\end{tabular}
	\end{table}
	Here $\mathbb{P}_n$ is the set of polynomials of degree $\le n$. 

	\thm{2.30: 
		Let $V$ have a spanning set $w_1, w_2, \ldots, w_n$. This spanning set contains a basis for the vector space.
	}
	\pf{
	When can I not use $w_1$ for my linearly independent set? If it's stupid, if $w_1 = 0$, then don't use it. If $w_1 = 0$, delete it and go on. If not zero, choose it! So my first element of linearly independent set, 
	\[
	w_1
	\] If $w_2$ is a multiple of $w_1$, I better not use it.  $w_2 \neq a w_1$. Then we pick $w_3$ to not be a linear combination of $w_1,w_2$ and keep going. And eventually we will get the result, which is a spanning list $w_1,w_2, \ldots, w_n$. By the process they are linearly independent.	 
	}

	We can also have a reverse theorem, 
	\thm{
	Every linearly independent list extends to a basis. 
	}

	\thm{
	If $V$ is a finite dimensional vector space and $U$ is a subspace of $V$, then, $U$ is finite dimensional. }
	Kind of crazy this needs a proof so I won't go into that - Frank Jones, 2024.

	\thm{
		$V$ be a finite dimension. $U \subset V$. Then, there exists, another subspace such that $W \subset V$ such that $U \oplus W$ is $V$. 
	}
	\pf{
	Choose a basis for $ U$: $w_1, w_2, \ldots$. In the usual way, extend the list to get a basis for the vector space. We will have $w_1, \ldots, w_n, u_1, $, $w_n$ will form basis for $U$ and $u_n$ will form basis for $W$. }


	\thm{2.42 $V$ is finite dimensional. A spanning set of the right length is automatically a basis. An independent list of the correct length is also a basis.} 

	\thm{2.42 Let $V_1$, $V_2$ be subspaces of a finite dimensional vector space. Then we can form 
		$V_1 \cap V_2$, and we can form $V_1 + V_2$. These have dimensions, 
		\[
		\dim (V_1 + V_2) + \dim (V_1 \cap V_2) = \dim V_1 + \dim V_2
\] }\pf{Axler is proper}

\section{Section 1C Problem 12} 
	\pr{
	Prove that the union of two subspaces of $V$ is a subspace of $V$ if and only if one of the subspace contains the other. In that case $V_1 \cup V_2$ is simply $V$. 
	}

	Union of three subspaces is a subspace if one of the three contains the other two. 

	\section{Soul less problem given soul}	\pr{
	Derivation of formula for $\sum_{k=1}^{n} k^2$		
	}
	\solu{
		Start with $\sum_{k=1}^{n} k^3 $ (bro!)
		\[
		\sum_{1}^{n}  k^3 = n^3 + \sum_{1}^{n-1} k^3 
		=
		n^3 + \sum_{k=1}^{n} (k-1)^3
		= 
		n^3 + \sum_{k=1}^{n} (k^3-3k^2+3k-1)
		\] The $\sum_{k=1}^{n} k^3$ cancels both side. 

		\[
		0 = n^3 + \sum_{k=1}^{n} (-3k^2 + 3k -1)
		\] 
		Turns out
		\[
		3 \sum_{k=1}^{n} k^2 = n^3 + \sum_{k=1}^{n} (3k-1)
		\] Using the idea of Arithmatic progression, we get, 
		\[
		= n^3 + \frac{3n^2 + n}{2} = \frac{2n^3+3n^2+n}{2}
		\] 

		Proves, \[
		\sum_{n=1}^{n} n^2 = \frac{n(2n^2+3n+1)}{6}
		\] 

	}


\section{Home Reading} 
I have found this chapter to be a of tremendous confusion because nothing here makes sense to me. I will re-read this whole chapter from the beginning and note them down here. 

\section*{Introduction to Finite Dimensional Vector Spaces} 
Some key points we are about to get blessed with are
\begin{itemize}
	\item Linear Combinations of Lists of Vectors		
\end{itemize}
Whoa whoa wait that's mouthful. So what I understand, that you can have a random circus of vectors 
\[
	(\vec{s}, \vec{t}, \vec{u}, \vec{p}, \vec{i}, \vec{d}), (\vec{s}, \vec{p}, \vec{a}, \vec{c}, \vec{e})
\] We have to list of random vectors up there, now they can get into combinations.  
\begin{itemize}
	\item Basis	
\end{itemize}
These kids are small enough to be \textbf{independent} but big enough that their \textbf{linear combinations} fill up the entire space. Okay makes a lot of sense. You can literally have any random vector in $\mathbb{R}^3$ just from combining $(1,0,0),(0,1,0),(0,0,1)$. Let's say you a vector $(4,9,2)$. Then you need to
\[
4(1,0,0) + 9(0,1,0) + 2(0,0,1) = (4,9,2)
\]

Fair. 

\subsection*{Span}

\df{
Linear Combination: Let's have a list of vectors $\vec{v}_1, \vec{v}_2, \vec{v}_3, \vec{v}_4, \ldots$ in the same space $ V$, then a possible linear combination is \[
\vec{m} = a_1 \vec{v}_1 + \ldots + a_n \vec{v}_n
\] $\vec{m}$ is a linear combination of that vector list above.  
}

For sake of example, let's make a vector list, $(1,0), (3,2), (4,1)$. A linear combination of this vector can be \[
\vec{t} = 5(1,0) + 8(3,2) - 2(4,1) = (5,0) + (24, 16) - (8, 2) = (21, 14) 
\]
So we just showed for this list $(21, 14)$ is a valid linear combination. 


\df{
Span: The set that contains all the possible linear combinations of a list of vectors $v_1, v_2, \ldots$ is called the span of the list. We can have any value of $a_i$ here, 
\[
	\text{span}(\vec{v}_1, \vec{v}_2, \vec{v}_3, \ldots) = {a_1 \vec{v}_1 + a_2 \vec{v}_2 + \ldots : a_1, a_2, \ldots \in \mathbb{F}}
\] 
}

We have to tediously put all the linear combinations of a list of vectors to get this specific set. Any linear combination of a list of vector is a member of the span. I like to think in this way. Say we have a few vectors $(1,0), (3,3)$. So the linear combination $5(1,0) + 6(3,3)$ is a member of the $\text{span}((1,0),(3,3))$ 

I am trying to get this into my mind straight that ``span is the set of all linear combinations of a vector-list".

\pr{
This is not a math problem. This is an actual goddamn problem I am suffering with. So there is a theorem \emph{The span of a list of vectors in $V$ is the \textbf{smallest} subspace containing all the vectors in the list.} I want a contrast, can we have a subspace that is not the smallest? 
}

But wait, do we know what a subspace is (for real Sabit you are asking this to yourself now?). 

\end{document}
