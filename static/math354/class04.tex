\documentclass[letter]{article}
\usepackage[monocolor]{ahsansabit}

\title{Honors Linear Algebra : : Class 04}
\author{Ahmed Saad Sabit, Rice University}
\date{\today}

\begin{document}
\maketitle
	\section{Dimensions of a Sum}
	Let the subspaces $V_1, V_2$ be two finite dimensional vector space. 
	We talked about, 
	\[
	V_1 + V_2
	\] 
	\[
	V_1 \cap V_2
	\] 
	\[
	\dim(V_1+V_2)=\dim(V_1)+\dim(V_2)-\dim(V_1\cap V_2)
	\] 
	
\pf{
	We begin the proof with a basis of the intersection $V_1 \cap V_2$, because then you can complete it. We extend it to a basis for $V_1$. Then for $V_2$.  
}

\pr{
	Exercises in Section 2C: (3) The vector is a polynomial of degree $4$. $\mathbb{P}_4(\mathbb{F})$. The dimension is 5, $1, x,x^2,x^3,x^{4}$. $U$ is the set of polynomials in $\mathbb{P}^{4}$ such that\[
	U = \{p \in \mathbb{P}_4(\mathbb{F}) | p(6) = 0\} 
	\]
	Basis is $x-6, (x-6)^2, (x-6)^3, (x-6)^{4}$. 

}

\pr{
	Exercise (4)
	\[
	\{p \in \mathbb{P}_4(F) | p''(6) = 0\} 
	\] 
	Dimension of this space is 4. Because vector space of dimension $5$ then put one constraint and that reduces dimension. Without $6$ being a constraint we would have the second derivatives that can possibly equal $6$
	\[
	1, x, x^3, x^{4}
	\] 
	Now with the constraint
	\[
	1, (x-6), (x-6)^3, (x-6)^{4}
	\] 
	Is there a choice of natural basis? ``It is my own personal feeling of joy"
}
\pr{
	Exercise (10): Bernstein Polynomials in $\mathbb{P}_m(\mathbb{F})$
	\[
	\mathbb{P}_k(x) = x^{k} (1-x)^{m-k}
	\] For $\mathbb{P}_1(F)$
	\begin{align*}
		\mathbb{P}_0(x) &= 1-x \\
		\mathbb{P}_1(x) &= x 
	.\end{align*}
	For $\mathbb{P}_2(F)$
	\begin{align*}
		\mathbb{P}_0(x) &= (1-x)^2  \\
		\mathbb{P}_1(x) &= x(1-x) \\
		\mathbb{P}_2(x) &= x^2 
	\end{align*}
	Then, $p_0, p_1, \ldots, p_m$ form a basis for $\mathbb{P}_m$. Let's use the definition of linear independence.
	For $m=4$
	\begin{align*}
		\mathbb{P}_0(x) &= (1-x)^{4} \\ 
		\mathbb{P}_1(x) &= x(1-x)^3 \\
		\mathbb{P}_2(x) &= x^2(1-x)^2 \\
		\mathbb{P}_3(x) &= x^3(1-x) \\
		\mathbb{P}_4(x) &= x^{4} 
	\end{align*}
	\[
	c_0(1-x)^{4}+c_1 x(1-x)^3 + c_2 x^2(1-x)^2 + c_3 x^3(1-x) + c_4x^{4} = 0
	\]
	$x=0$ then $c_0 = 0$. Factor out $x$ then you can set $x=0$. 
	\[
	x \left( c_1 (1-x)^3 + c_2 (1-x)^2 + c_3 x^2(1-x) + c_4 x^3 
	\right) = 0
	\]  
}

\pr{
	Exercise (13): Suppose $U,W$ are $5- $dimensional subspace of $\mathbb{R}^{9}$. Then $U \cap W$ does not have $\{\phi\} $. 
}

\pr{
        $U,W$ are 4 dimensional subspaces of $\mathbb{C}^{6}$. Prove that, there are at least two vectors in the intersection such that neither is a scalar multiple of the others. 
	\[
	\dim(U \cap V) \ge 2
	\] 
}

\section{Linear Maps} 
	The assumptions are $\mathbb{F}$ is either $\mathbb{R}$ or $\mathbb{C}$. $U,V,W$ are vector spaces of $\mathbb{F}$. $3A$ section starts with the idea on vector space of linear maps.

\df{
	A linear map from $V$ to $W$ is a function $T$, that maps
	\[
	T: V \to  W
	\] 
	With the following properties, each have linear structure with addition and scalar multiplication. You want to honor the definition of the vector space. 
	 \[
	 T(u+v) = T(u) + T(v)
	 \] 
	 \[
	 T(\lambda u) = \lambda T(u)
	 \] 
	 Some mathematicians like to use the words, ``Linear Transformation", or ``Linear Operator". Linear maps are shown like, $\mathcal{L}(v,w)$. If $v=w$, $\mathcal{L}(v,v)=\mathcal{L}(v)$. Sabit will use $\mathbb{L}(v)$ instead. Hehe.
}

Differentiation is a linear map, 
	\[
	D: \mathbb{P}(F) \to \mathbb{P}(F)
	\] 
	So $(Df)(x) = f'(x)$.

Integration $T \in  \mathbb{L}(\mathbb{P}(F), F)$, as
	\[
	T(p) = \int_0^1 p(x) \mathrm{d} x	
	\] 

Backward shift
	$V$ is all the sequences of the form $x = (x_1,x_2,x_3, \ldots)$. $Tx = (x_2, x_3, x_4, \ldots)$. 
	Then there exists a unique $T \in  \mathbb{L}(v,w)$ such that,
	\[
	T v_k = w_k
	\] 
	Operations on linear maps
	\[
	T,S \in \mathbb{L}(V,W)
	\] 
	Then $$(T+S)(\vec{v}) = T(\vec{v}) + S(\vec{v})$$ and $$(\lambda T)(\vec{v}) = \lambda T(\vec{v})$$. You can even have vector spaces of $\mathbb{L}$ linear maps. 

\subsection{Product Linear Maps}
	$T \in \mathbb{L}(U,V)$, $S \in \mathbb{L}(V,W)$
	\[
	U -^{T}\to  V -^{S}\to  W
	\]
	$ST$ is defined to be
	\[
	ST(\vec{u}) = S (T(\vec{u}))
	\] 

\end{document}
