\documentclass[letter]{article}
\usepackage[monocolor]{ahsansabit}

\title{Honors Linear Algebra : : Class 05}
\author{Ahmed Saad Sabit, Rice University}
\date{\today}

\begin{document}
\maketitle

\section{Exercises 2C} 
\textbf{(8)}

Suppose $v_1, v_2, \ldots, v_m$ are linearly independent. Dude please use vector notations. $\vec{v}_1, \vec{v}_2, \vec{v}_3, \ldots, \vec{v}_m$ are linearly independent. Suppose $\vec{w}$ is a vector too. Prove that 
\[
\dim (
\text{span}
\{\vec{v}_1+\vec{w}, 
\vec{v}_2+\vec{w},
\vec{v}_3+\vec{w}, 
\ldots
,
\vec{v}_m + \vec{w}\} 
) \ge m-1
\] 

\textbf{(15)}
$V_1, V_2, V_3$ subspaces. Such that
\[
\dim V_1 + \dim V_2 + \dim V_3 > 2 \dim V
\] 
Prove that $V_1 \cap V_2 \cap  V_3 \neq \{0\} $

\textbf{(19)}
Prove or give a counter example. \[
\dim(V_1) + \dim(V_2) + \dim(V_3) = \dim(V_1+V_2+V_3) + \dim(V_1 \cap V_2) + \dim (V_2 \cap  V_3) + \dim(V_1 \cap V_3) - \dim (V_1 \cap  V_2 \cap  V_3)
\] 

$\implies$ Let's take $\mathbb{R}^{2}$, $V_1: \text{x-axis}$, $V_2:\text{y-axis}$, $V_3:x=y$. Sum of their three dimension is $3$. 
$3 = 2 + 0 + 0 + 0 - 0$

\textbf{(20)}
True version of $19$. 

\section{Exercises 3A} 
\textbf{(16)} 
Suppose $V$ is a finite dimensional vector space and the $\dim V \ge 2$. Prove their exists linear operators $S,T \in \mathcal{L}(V)$ such that their product,
\[
ST \neq TS
\]
Example: there are two vectors $\vec{v}$ and $\vec{w}$ which are linearly independent. So there will be a basis $v, w, \ldots$. So all the vectors in $V$ have the form,
\[
c_1 \vec{v} + c_2 \vec{w} + \cdots
\]
Every vector is uniquely determined by saying what these vector coefficients are. 
$S(V)$ be defined such that, 
\[
S(\vec{v}) = \vec{v}
\] 
\[
S(\vec{w}) = \vec{0}
\] 
\[
T(\vec{v}) = 0
\] 
\[
T(\vec{w}) = \vec{w}
\] 

Let's compute $ST$ and $TS$. 
\[
ST(\vec{v}) = S(T(\vec{v})) = S(0) = 0
\] 
\[
TS(\vec{v}) = T(S(\vec{v})) = T(v) = 0
\] 
\[
ST(\vec{w}) = S(T(\vec{w})) = S(\vec{v})
\] 
\[
TS(\vec{w}) = T(0) = 0
\] 
\textbf{(11)} $V$ is a finite dimensional. $T \in \mathcal{L}(v)$. That means $T$ maps to itself. And $T$ commutes with every $S \in \mathcal{L}(v)$. $TS = ST$, then prove $T$ is a scalar multiple of the identity.

Remark: ``I already knew this result for $n \times n$ matrices. And I have loved assigning it as Homework".

For all $f \in \mathcal{L}(V, \mathbb{F})$, which is not 0. I.E $\vec{v}$ such that $f(\vec{v}) \neq 0$. Proof: Use a basis for $V$ : $\vec{v}_1, \vec{v}_2, \ldots, \vec{v}_n$ and $\vec{x} = \sum_{n=1}^{n} c_n \vec{v}_n$. 
\[
f(\vec{x}) = c_1
\] 
The set of all $\mathcal{L}(V, \mathbb{F})$ is called the dual space of $V$. Define $S \in \mathcal{L}(v)$ and $S(x) = f(x) v$ 
$ST = TS$, 
\[
ST(x) = S(T(x)) = f(T(x))V
\] 
\[
TS(x) = T(S(x)) = T(f(x)v) = f(x) T(v)
\] 
\[
f(x) T(v) = f(T(x)))v
\] 
\[
f(x) \neq  0
\] 
\[
T(v) = \frac{f(T(x))}{f(x)} \ v
\] 

\section{3B Null Space and Ranges} 
3.11 Null Space 

\df{
	Let $T \in \mathcal{L}(V,W)$. The null space of $T = \{v \in V | T(v) = 0\} = \text{null}(T)$
}
Fact: The null space of $T$ is a subspace of $T$. It's a subset because 
\[
T(\vec{v}+\vec{w}) = T(\vec{v}) + T(\vec{w}) = 0
\] Hence $\vec{v}+\vec{w} \in \text{null}(T)$. 
Whatever from the subspace I put in $T$ I get $0$. 
\[
\text{Range}(T) = \{T (v) | v \in V\} 
\] 
\df{
$T$ is injective if the equation $T(\vec{v}_1) = T(\vec{v}_2)$ $\implies \vec{v}_1 = \vec{v}_2$
}
Proof: $T(\vec{v}_1 - \vec{v}_2) = T(\vec{v}_1) - T(\vec{v}_2) = 0$, so 
\[
\therefore v_1 - v_2 \in \text{null} (T)
\] 
$T$ is injective if and only if $\text{null}(T)$ is $\{0\} $.

\df{
$T$ is surjective if $\text{range}(T)=W$. 
}

\section{Fundamental Theorem of Linear Algebra} 

\[
\dim (V) = \dim (\text{null}(T)) + \dim (\text{range}(T))
\] 

\end{document}
