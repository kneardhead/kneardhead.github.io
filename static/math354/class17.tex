\documentclass[letter]{article}
\usepackage[monocolor]{ahsansabit}

\title{Honors Linear Algebra : : Class 17}
\author{Ahmed Saad Sabit, Rice University}
\date{\today}

\begin{document}
\maketitle

\section*{5B: Minimal Polynomial} 
\subsection*{5.22 Existence, Uniqueness, and degree of Minimal Polynomial}

Ey Ey Monica 
Ey podei Monica. 

Definition 5.21 and 5.22. 

$dim V = 1$ case. 

$T \in  \mathcal L (V)$ and $V = \mathbb{F} \cdot \vec{v}$. Now,
\[
T(w ) = T(a v) = a T(v) = ab v
\]
So, 
\[
T = b I 
\]
and a monic polynomial, 
$p(z) = z+ \text{ const }$
So, 
\[
p(T) = T + \text{ const } I 
\]
Zero if $\text{ const } = b$
so, 
\[
p(z) = z - b
\] Is the needed polynomial. 

To prove uniqueness, 
\[
p(T) = 0
\]
\[
r(T) = 0
\]
Now consider the two minimal polynomials (with leading coefficient 1 and minimum degree), the degree of polynomial $p -r $is $< $ degree of $p$. So, 
\[
p(z) = z^{m} + \cdots
\]
\[
r(t) = z^{m} + \cdots
\]
\[
	(p-r) T = 0
\] 

5.27 : Eigenvalues are the zeros of minimal polynomial. 

$\lambda$ is a eigenvalue of $T$ if and only if the minimal polynomial $p$ for $T$ satisfies $p(\lambda) = 0$. 
$\impliedby$ reasoning, $p(\lambda) = 0$ so divisible by $z - \lambda$. \[
p(z) = (z-\lambda) q(z)
\]
\[
p(T) = (T - \lambda I) q(z) = 0 
\]
choose $q(v) \neq  0$, then,
 \[
p(T) v = (T - \lambda I) q(T) v = 0
\]
Eigenvalue of $T$. 


$\implies$ $\lambda$ is eigenvalue of $T$. Then, \[
T v = \lambda v
\] for an eigenvector $\neq  0$ then, 
\[
p(T) = 0
\]  and $p(T) v = 0$. 

5.29: Use polynomial algebra to try to divide $q$ by $p$. 

\section*{Exercise 6. }

applied $T$ to the equation again. That gave
\[
T^2 + I = 0
\] 
Maybe minimal polynomial has lesser degree, then, 
\[
z + C = 0
\] 
Then, 
\[
T + cI = 0
\]
\[
T = - cI 
\]
this is not the case, and the $z+C$ is not the right one. So the correct minimal for this is,
\[
z^2 + 1 
\]


\end{document}
