\documentclass[letter]{article}
\usepackage[monocolor]{ahsansabit}

\title{Honors Linear Algebra : : Class}
\author{Ahmed Saad Sabit, Rice University}
\date{\today}

\begin{document}
\maketitle

\section*{Exercise 3E6}
Suppose $U, W$ are subspaces of $V$. One of the translates of $U$ is a translate of $U$ is the translate of $W$. 
\[
v  + U = x + W
\] 
Then $U$ and $W$ are same to prove. 

From what's given, $v = x  + w$. So, 
\[
v - x \in W
\] 
Suppose $u \in U$. Then $v+ u$ belongs to $v + U$. And  this equals to $x + W$. 
So, 
\[
u \in x - v + W 
\] 
Hence $U \subset W$. Likewise $W \subset U$ and $U = W$ 

\section*{3F: Duality} 
\df{
Consider the linear functions from $V$ to the scalars. 
\[
\mathcal L (V, \mathbb{F})
\]
Each one of these $\phi \in  \mathcal L (V, \mathbb{F})$ is a linear function from $V$ to the scalars. We call \emph{Linear Functional} on $V$. 

An added notation would be $V'$ on this space would be set of all functionals on $V$. 
}

An example can be $V = \mathcal P (\mathbb{F})$. A functional can be \[
\int _0^{1} p \, \mathrm{d} x
\]
Another example can be evaluation at, say, 10, 
\[
p(10)
\] 
Functional on $\mathbb{F}^{\infty}$ can be 
\[
\vec{x} = \left(x_1, x_2, \ldots, x_n, \ldots\right)
\]
A functional can be $\phi \in \mathcal L (\mathbb{F}^{\infty}, \mathbb{F})$, 
\[
\phi (x) = x_1 + \ldots  + x_n
\]
It should be apparent that $V'$ is a vector space. Now, given for a basis $V$, we obtain the very important dual basis for $V'$ 

\[
\phi_k (v_j) = 1 \text{ or } 0
\] 
Here it's one if $ k = j$. These linear functionals are linearly independent. For if, 
\[
c_1 \phi_1  + \ldots + c_m \phi_m = 0
\] 
Then evaluate at $v = v_k$. 
\[
c_k \phi_k(v_k) = 0
\] 
\[
c_k = 0
\] 
From here we can say 
\[
\dim V' = \dim V
\]
\[
\vec{v} = \phi_1(\vec{v}) \vec{v}_1 + \ldots + \phi_m (\vec{v}) \vec{v}_m
\]

\section*{}
Given $T \in  \mathcal L (V, W)$, 
\[
V \to  W 
\] 
We define dual of $T$ by 
\[
W' \to V'
\]
\df{
$T'$ start with a linear functional $\phi$ on $W$. 
\[
T' (\phi) (v) = \phi(T(v) ) 
\] 
\[
	(T' \cdot \phi) \cdot v = \phi \cdot (T \cdot v)
\]
Hence 
\[
T' \cdot \phi = \phi \cdot T
\]

}

\section*{Null Space and Range of Dual of Linear Map} 
\df{
3.121
}
Example 3.122

EX3F1 

Each linear functional is surjective or $0$. This is easy to see that if  $\phi(v) \neq  0$ then $\lambda \phi(v) = \phi(\lambda v)$ so you can have any value as you want. 









































\end{document}
