\documentclass[letter]{article}
\usepackage[monocolor]{ahsansabit}

\title{Honors Multivariable Calculus : : Class 07}
\author{Ahmed Saad Sabit, Rice University}
\date{\today}

\begin{document}
	\maketitle 

	Let $\mathcal P \in  \mathcal L(V)$ and $P^2 = P$ prove that 
	 \[
	V = \text{null} P \oplus \text{range} P
	\] 

Suppose $u \in \text{null}P \cap \text{range} P$ and then 
\[
P u = 0
\] 
\[
u = Pv
\] 
\[
0 = Pu = P^2 v = Pv = u
\] 
Now we need to show every vector can be written as the sum of the $\text{null} P \oplus \text{range}P$. 
Next $u \in  V$ and $P(I - P  ) = 0$ 
\[
 u = P u + (I - P)u 
\] 
\[
P(I-P) u = (P - P^2)u = 0 u = 0 
\]

\section{Problem 3B27} 
$\implies$ assume $S = TE$. Every vector in range $S$ is of the form $Sv$. Hence $TEv = Sv$. $Sv \in \text{range}T$. 

$\implies$ Assume range of $S$ is $\subset $ range of $T$. Let' $v_n$ be a basis for $\mathbb{V}$. Then 
$S v_k$ is $T w_k$. I want to find $E$ such that $S = TE$, $S v_k = TE v_k$ 
Through defining $E$ by $Ev_k = w_k$ then 
\[
E (\sum_{n=1}^{N} c_n v_n) = \sum_{n=1}^{N} c_n w_n
\]

\section{Fundamental Theorem of Linear Algebra}
Given a linear function $T \in \mathcal L (V, W)$, and we assume $V$ is finite dimension. Then the 
\[
\dim V = \dim \left(\text{null} T\right) + \dim \left(\text{range} T\right)
\]
Null is in $T$ and the range is in $W$.

\textbf{Proof by us}
\begin{itemize}
	\item Choose a basis for $\text{null}T$ hence $\{\vec{v}_m\} $.
	\item Extend this basis to achieve a basis for $\mathbb{V}$ itself. $\{\vec{v}_m\}  $ and $\{\vec{v}_{m+1} \ldots \vec{v}_n\} $ 
	\item We assume that $n>m$.
	\item Look at the range of $T$, any vector of the form
		\[
			\vec{v} = c_1 \vec{v}_1 + \ldots + c_{m+1} \vec{v}_{m+1} + \ldots
		\]
		\[
			Tv = 0 + \ldots + 0 + c_{m+1} T \vec{v}_{m+1} + \ldots c_n T \vec{v}_{n}
		\]
		Proof will be over if $T \vec{v}_{m+1} , \ldots, T \vec{v}_n $ are linearly independent.\item Start with an equation of the form 
		\[
			d_{m+1} T \vec{v}_{m+1} + \ldots + d_n T \vec{v}_n = 0
		\]
		\[
			T ( d_{m+1} v_{m+1} + \ldots d_n v_n ) = 0
		\] 
		Then the equation in the bracket is belonged in $\text{null}T$. 
\end{itemize}

\section{Problem : 3B 28} 
Suppose $D$ is a member of $\mathcal L (\mathcal P (\mathbb{R}))$ and suppose $\deg D p$ is $\deg p - 1$ for non-constant polynomial $p$. Prove that $D$ is surjective.

\textbf{Proof} Consider the subspace of $\mathcal P (\mathbb{R})$ to be the span of $x ,x^2, \ldots, x^{n}$. Notice: no constant polynomial except $0$. When we restrict $D$ to this subspace, $\text{range} D$ to be equal to the span of $1,x, \ldots, x^{n}$. 

Definition of degree 
\[
\deg(1) = 0
\] 
\[
\deg(0) = - \infty
\]

\section{Problem 3B19} 
\section{Problem 3B20}
	\end{document}
