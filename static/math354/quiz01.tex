\documentclass[letter]{article}
\usepackage[gruvbox]{ahsansabit}

\title{Honors Linear Algebra : : Quiz 01}
\author{Ahmed Saad Sabit, Rice University}
\date{\today}

\begin{document}
	\maketitle 

	\section*{Problem 01} 
	Let's consider the base case $n = 1$ 
	\[
		(1-a)^{n} \ge 1-na \quad \rightarrow \quad (1-a) \ge 1 - a
	\]
Base case is true. 

Let's try the case $k$, 
\[
	(1-a)^{k} \ge 1 - ka
\] 
Let's try the case $k+1$ 
\[
	(1-a)^{k+1} \ge 1 - (k+1)a \quad \rightarrow \quad (1-a)^{k} (1-a) \ge 1 - a - ka
 \]
 If we can somehow prove $k+1$ case is true by using $k$ case then we can confidently say that the statement is true (proven through induction). 

 Multiply both sides of $(1-a)^{k} \ge (1 - ka)$ by $(1-a)$ 
 \[
	 (1-a)^{k} (1-a) \ge (1-ka)(1-a)
 \]
 We get, 
 \[
	 (1-a)^{k} (1-a) \ge 1 -a -ka + ka^2
 \]
 By out definition $ka^2$ is supposedly $ka^2 > 0$ hence, this implies
 \[
 \implies (1-a)^{k} (1-a) \ge 1 -a -ka
 \]
 This is exactly the $k+1$ case that is proven, so we are done because $k+1$ is true. 

 \section*{Problem 02} 
 \subsection*{(a)}
 To be a subspace we need $(0,0,0,0,0)$ to be a member of the system. But if $x_3 = 5 x_4 + b$ is a condition, then if $x_3 = 0$, $x_4 \neq  0$ and if $x_4 = 0$ then $x_3 \neq  0$ given $b \neq  0$. But then the $\vec{0}$ member isn't possible, hence the tuple forms a vector space if and only if $b=0$.

 \subsection*{(b)} 
 We can have a continuous function $f$ such that $f(x) = 0$ for any $x$ and $f$ is defined to be the zero member. 

 From theory of calculus (I am taking 232 so this occupies my day and night at this point) we know sum of two continuous function is also continuous hence $f,g \in \mathbb{R}^{[0,1]}$ implies $f+g \in \mathbb{R}^{[0,1]}$. The set by definition includes it. 

 Continuous function is still continuous given the function is $f$ so that $f(x) = X$ and $ \alpha f(x) = \alpha X$. Both of the additive and multiplicative members are included in the set of all continuous real valued function so it is a subspace. 

 \subsection*{(c)}
 Similar to before, we can define a differentiable function $f$ such that $f(x) = 0$. So $0$ is a member function in the subspace. 

 Now, sum of two differential operators (or functions) is also differentiable. This follows from the sum rule of limits in calculus. So $f,g \in \mathbb{R}^{\mathbb{R}}$ implies $f+g \in \mathbb{R}^{\mathbb{R}}$. The set by definition includes them. 

 Similarly we can have the function $f(x) = X$ and given $\alpha f(x) = \alpha X$, this is differentiable too and hence by definition the multiplicative closure is followed as the set also is defined to include the functions $\alpha f$. 

 \subsection*{(d)} 
 We already know the set of real valued differentiable functions itself is a subset given it contains the zero function. Now the subset of the this space so that $f'(2) = b$ can only be a subspace if that itself contains the zero function. The zero function implies $f'(2) = 0$. If $b\neq 0$ then the zero function isn't contained and the set can't be a subspace. 

 Now from basic calculus we know $f'(2) + g'(2) = (f' + g')(2) = 0$ so the addition closure is followed. Also, $\alpha f'(2) = \alpha (0)$ so multiplicative closure is also followed. Set of all such functions successfully form a subset. 

 \subsection*{(e)	}
Set of all sequences of complex numbers (let's call it $\mathbb{S}$) will also include the summation of the sequences. So $a_\infty$ and $b_\infty$ being members of this set, $a_\infty + b_\infty$ must also be a member of this set by definition. 

Similarly, set of all sequences implies it will include $\alpha \left(a_\infty\right)$ as well where $a \in \mathbb{F}$.

Now we need a $0$ sequence for this set. If the condition is that it is the set of all sequences that approaches to the limit of $\infty$ then $0,0,0,0\ldots$ sequence is also a member, hence the set also has a zero member. This set $ \mathbb{S}$ hence will be subspace of $\mathbb{C}^{\infty}$
\end{document}
