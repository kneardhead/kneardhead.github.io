\documentclass[letter]{article}
\usepackage[monocolor]{ahsansabit}

\title{Honors Multivariable Calculus : : Class 23}
\author{Ahmed Saad Sabit, Rice University}
\date{\today}

\begin{document}
\maketitle

\section*{INTEGRATION: (lesgoo)}
Math 212 understanding, 
\[
D \subset  \mathbb{R}^{n} 
\] 
and let's be in $f: D \to  \mathbb{R}$. The integral 
\[
\int_D f =
\] 

\[
	\int_{[1,5]} 3x^2 \mathrm{d} x \text{ is defined via Riemann Sums}
\]
We just going to think we are summing the area of a bunch of rectangles. 
So think about breaking the system into Pieces. 

$\int_D f$ is break $D$ into pieces and each piece take ``volume of the piece in n-dim" and multiply ``value of $f$ on piece". This 
is the informal but 
useful understanding of what the integration is doing. What does integral measure? 

signed area, which is the volume of between the graph of $f$ in $\mathbb{R}^{n+1}$ and domain $D \times \{0\} \subset \mathbb{R}^{n} \times \{0\} \subset \mathbb{R}^{n+1} $

If $f$ is density like stuffs, the integral gives the mass. Defining $\delta$ density $r$ position, moment of inertia
\[
\int_D \delta r^2 = \text{ Moment of Inertia}
\]

Integrals might give average. Take $N$ values of $f$ each $\frac{1}{N}$ of $D$. 
\[
\frac{\sum_{n=1}^{N} f(x_i)}{N} = \frac{\sum_{n=1}^{N} f(x_i) \frac{\text{area of } D}{N}}{\text{area of } D} \approx \int_D f
 \]

 So $\int_D f$ are $n$ dimensional volume of $D$, gives average of $f$. 




\section*{Usage of Inverse Functions using Implicit Theorem} 
Y'all know about Polar Coordinates.
\[
	(x,y) = (r \cos \theta , r \sin \theta) = F(r, \theta)
\] 
Hence, $F : \mathbb{R}^2 \to \mathbb{R}^2$.

\df{
	Given $C^{1}$ $\mathbb{F}:\mathbb{R}^{n} \to \mathbb{R}^{n}$ with $\vec{a} \in  \mathbb{R}^{n}$ where $dF_{\vec{a}} $ is invertible. Then $F$ is locally a bijection, or, there are open sets $U $ where $a \in U$ such that $f (\vec{a}) \in V$ such that $F$ restricted to $U$ is a bijection between $U$ and $V$. The inverse $G$ sending $V$ to $U$ is $C^{1}$, and  
	\[
		dG_{f(\vec{a})} = \left(d F_{\vec{a}} \right)^{-1}
	\] 
}

\pf{
Let $F = (f_1, \ldots, f_n) $ and $f_i:\mathbb{R}^{n} \to \mathbb{R}$. Consider $\mathbb{R}^{2n}$ consists of $x_1, \ldots, x_n$ and $y_1, \ldots, y_n$. And we got
\[
y_1 = f_1 (x_1, \ldots, x_n)
\] 
\[
y_n = f_n (x_1, \ldots, x_n)
\]
These functions show the graph of $f _n$. Apply implicit function theorem, 
\[
y_1 - f_1 (x_1, \ldots, x_n) = 0
\] 
\[
g_n(x_1, \ldots, x_n , y_1, \ldots, y_n) = y_n - f_n (x_1, \ldots, x_n) = 0
\]
To apply Implicit function theorem, we need $mx$ of 
\[
\frac{\partial g_i}{\partial x_j} 
\] to be invertible. 
This is the same as $mx$ of $- \frac{\partial f_i}{\partial x_j}$ at $\vec{a}$. This is $- d F_{\vec{a}}$. 
}

\end{document}
