\documentclass[letter]{article}
\usepackage[monocolor]{ahsansabit}

\title{}
\author{Ahmed Saad Sabit, Rice University}
\date{\today}

\begin{document}
\maketitle

\section*{Problem 01} 
\subsection*{(a)}
Directional derivative is taken along a direction, say $\vec{v}$ such that $\vec{v} = \begin{pmatrix} a \\ b \end{pmatrix} $ where $a,b \in \mathbb{R}$. From the definition of a directional derivative 
\[
	D_{\vec{v}} f(0,0) =  
	\lim_{h \to {0}} 
	\frac{f(ah,bh) - f(0,0)}{h} = 
	\frac{
\frac{a^2 b h^3}{(a^2 + b^2) h^2}
	}{h} = 
	\frac{a^2b}{a^2 + b^2}
\]

\subsection*{(b)}
Let's assume that $f$ is differentiable at the origin. Then 
\[
	D_{\vec{v}} f(\vec{a}) = \sum_{n=1}^{N} v_n D_n f(\vec{a})
\]
Here $N = 2$. If $\vec{p} = \vec{0} $ and $\vec{v} = (a,b)$ then

\[
	D_{\vec{v}} f(\vec{p}) = \sum_{n=1}^{N} v_n D_n f(\vec{p}) =  
	a D_1 f(0,0) + b D_2 f(0,0)
\]
\[
= a \left( \frac{a^2 ( 0) }{a^2  }\right) + 
b \left( \frac{(0) b}{b^2}\right) = 0
\]
This is a contradiction because we had already solved $D_{\vec{v}} f(0,0) = \dfrac{a^2b}{a^2+b^2}$ yet the partial differentiation addition rule doesn't sum up to the directional derivative, hence contradicting the initial assumption of $f$ being differentiable. 

\subsection*{(c)} 
If $f$ is continuous at the origin then 
\[
\lim_{(x,y) \to (0,0)}  f(x,y) = f(0,0)
\] 
If $f$ has a limit then 
\[
 |f(x,y) - f(0,0)| < \epsilon
\]
would mean $|(x,y) - (0,0)| < \delta$,  for some $\epsilon,\delta > 0$. Considering $f(0,0) = 0$ as stated in the function definition (and also considering the function to be continuous), let's check if the limit exists and if it's $f(0,0) = 0$. 
\[
\left| 
\frac{x^2 y }{x^2+y^2} 
\right| < \epsilon
\] 
 Say  $|x| < \gamma$ and $|y| < \gamma$, then it implies $$\frac{\gamma}{2 }  < \epsilon$$ 
 Hence meaning that $|x| < 2 \epsilon$ and $|y| < 2 \epsilon$. This implies 
 \[
 \sqrt{x^2 + y^2}  < 2\sqrt{2} \epsilon
 \] 
Setting $2 \sqrt{2}  \epsilon = \delta$ gives us the following as $\sqrt{x^2+y^2}  = | (x,y) - (0,0)| $
\[
| (x,y) - (0,0 )| < \delta 
\] 
Hence the limit exists, and it also happens to be equal to $f(0,0) = 0$. So the function is continuous.


\section*{Problem 02} 
\subsection*{(a)}
The simple single variable derivative where we are free to consider $x \neq 0$, 
\[
f'(x) = 2x \sin\left(\frac{1}{x}\right) - \cos \left(\frac{1}{x}\right)
\] 
I did the simple derivative on paper. 

\subsection*{(b)} 
\[
f'(0) = \lim_{h \to 0} \frac{f(h) - f(0)}{h} = 
\lim_{h \to 0} h^2 \sin (1 / h)
\]
The $\sin (\frac{1}{h}) $ is constrained within the range $[-1,1]$ so it does not blow up to infinity. The coefficient $h^2$ does approach to $0$, and as the $\sin 1 / h$ factor is not growing, we can safely say that the limit is $0$. So, 
\[
f'(0) = 0
\] 

\subsection*{(c)} 
If the function has a limit, the sequence $\{f(\vec{x}_k)\} $ will always converge to the limit $L$ for any possible sequence $\{\vec{x}_k\} $ that also converges to a limit $\vec{a}$. If $\{f(\vec{a})\} = L $ then we can safely say this function is continuous. 

\[
f'(x) = 2x \sin \left(\frac{1}{x}\right) - \cos \left(\frac{1}{x}\right)
\] 
Let $x = \frac{1}{\theta }$, and $x \to 0$ for $\theta \to  \infty$ (trivial). So, if $f'$ is continuous, 
\[
\lim_{x \to a} f'(x) = f'(a)
\]
\[
\lim_{\theta \to \infty} f'(1 / \theta) = \lim_{\theta \to \infty}  \frac{2}{\theta} \sin\left(\theta\right) - \cos \left(\theta\right)
\]
Using the previous similar reasoning we know that $\frac{2}{\theta} \sin \theta \to 0$ for $\theta \to 0$. But $\cos \theta$ can be anything in between $-1$ and $1$ given $\theta$. This limit can't exist because $\cos \theta$ is anything in the range $[-1,1]$ 

\section*{Problem 03} 
\thm{
Folland Theorem 2.19: 

Let $f$ be a function defined on an open set in $\mathbb{R}^{n}$ that contains the point $\vec{a}$. Suppose that the partial derivatives $\frac{\partial f}{\partial x_j}$ for all $j$ exist around the local neighborhood of $\vec{a}$ and they are continuous. Then $f$ is differentiable at $\vec{a}$. 
}

\subsection*{(a)}
Let $(f_1(\vec{x}) , f_2(\vec{x})) $ be a vector $\vec{v}$. Then differentiablity means 
\[
\frac{f(\vec{v}+\vec{h}) - f(\vec{v}) - \vec{c}\cdot \vec{h}}{|\vec{h}|} \to  0 
\] 
as $\vec{h} \to  0$ where $\vec{c} = \left(\partial_1 f(\vec{v}), \partial_2f(\vec{v})\right)$. 



\end{document}
