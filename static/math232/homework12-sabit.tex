\documentclass[letter]{article}
\usepackage[monocolor]{ahsansabit}
\usepackage{tikz}
\usepackage{tikz-3dplot}
\usetikzlibrary{calc, positioning, math}
\title{Honors Multivariable Calculus : : Homework 1x}
\author{Ahmed Saad Sabit, Rice University}
\date{\today}

\begin{document}
\maketitle

\section*{Problem 01} 
\begin{tcolorbox}[colback=white, colframe=NordBlue, sharpish corners]
	\textbf{Aid for my brain: }
	Let the compact region be $R \in \mathbb{R}^{n-1}$ where $n = 3$. Common sense tells us $R$ is basically a disk in $\mathbb{R}^{2}$ or for now $x-y$ plane if we want what is in the figure. The tip of the cone is $\vec{a}$. 

	I think from the question our $R$ doesn't really need to be necessarily a disk. Now the region is all the lines that join from $\vec{a}$ to $R$. If $\vec{x} \in R$ then considering a linear map $\gamma_{\vec{x}}: [0,1] \to \mathbb{R}^{n} $ such that $\gamma_{\vec{x}}(0) = \vec{a}$ and $\gamma_{\vec{x}}(1) = \vec{x} \in R$ 

\begin{center}
\newcommand{\radiusx}{2}
\newcommand{\radiusy}{.5}
\newcommand{\height}{3}
\begin{tikzpicture}
\coordinate (a) at (-{\radiusx*sqrt(1-(\radiusy/\height)*(\radiusy/\height))},{\radiusy*(\radiusy/\height)});

\coordinate (b) at ({\radiusx*sqrt(1-(\radiusy/\height)*(\radiusy/\height))},{\radiusy*(\radiusy/\height)});

\draw[fill=gray!30] (a)--(0,\height)--(b)--cycle;

\fill[gray!50] circle (\radiusx{} and \radiusy);

\begin{scope}
\clip ([xshift=-2mm]a) rectangle ($(b)+(1mm,-2*\radiusy)$);
\draw circle (\radiusx{} and \radiusy);
\end{scope}

\begin{scope}
\clip ([xshift=-2mm]a) rectangle ($(b)+(1mm,2*\radiusy)$);
\draw[dashed] circle (\radiusx{} and \radiusy);
\end{scope}

\draw[dashed] (0,\height)|-(\radiusx,0) node[right, pos=.25]{$a_n$} node[above,pos=.75]{$r$};

\draw (0,.15)-|(.15,0);
\end{tikzpicture}

\end{center} 
The line segment is set of all points such that, 
\[
	\Gamma = \{s \in \mathbb{R}^{n} : s = \vec{a} t + (1- t ) \vec{x} \text{ where } t \in  [0, 1], \vec{x} \in R, \vec{a} \in \mathbb{R}^{n}\} 
\]
Here $x \in  R$. Every point $\vec{p} \in \Gamma$ is a memeber of the cone.  
\end{tcolorbox}

The volume of the region $\Gamma$ (which is the defined cone) is going to be,
\[
\int_\Gamma 1 = \text{Volume}
\] 
Now the burden is to find a region $\Gamma$. 

\end{document}
