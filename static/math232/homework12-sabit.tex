\documentclass[letter]{article}
\usepackage[monocolor]{ahsansabit}
\usepackage{tikz}
\usepackage{tikz-3dplot}
\usetikzlibrary{calc, positioning, math}
\title{Honors Multivariable Calculus : : Homework 1x}
\author{Ahmed Saad Sabit, Rice University}
\date{\today}

\begin{document}
\maketitle

\section*{Problem 01} 
\begin{tcolorbox}[colback=white, colframe=NordBlue, sharpish corners]
	\textbf{Aid for my brain: }
	Let the compact region be $R \in \mathbb{R}^{n-1}$ where $n = 3$. Common sense tells us $R$ is basically a disk in $\mathbb{R}^{2}$ or for now $x-y$ plane if we want what is in the figure. The tip of the cone is $\vec{a}$. 

	I think from the question our $R$ doesn't really need to be necessarily a disk. Now the region is all the lines that join from $\vec{a}$ to $R$. If $\vec{x} \in R$ then considering a linear map $\gamma_{\vec{x}}: [0,1] \to \mathbb{R}^{n} $ such that $\gamma_{\vec{x}}(0) = \vec{a}$ and $\gamma_{\vec{x}}(1) = \vec{x} \in R$ 

\begin{center}
\newcommand{\radiusx}{2}
\newcommand{\radiusy}{.5}
\newcommand{\height}{3}
\begin{tikzpicture}
\coordinate (a) at (-{\radiusx*sqrt(1-(\radiusy/\height)*(\radiusy/\height))},{\radiusy*(\radiusy/\height)});

\coordinate (b) at ({\radiusx*sqrt(1-(\radiusy/\height)*(\radiusy/\height))},{\radiusy*(\radiusy/\height)});

\draw[fill=gray!30] (a)--(0,\height)--(b)--cycle;

\fill[gray!50] circle (\radiusx{} and \radiusy);

\begin{scope}
\clip ([xshift=-2mm]a) rectangle ($(b)+(1mm,-2*\radiusy)$);
\draw circle (\radiusx{} and \radiusy);
\end{scope}

\begin{scope}
\clip ([xshift=-2mm]a) rectangle ($(b)+(1mm,2*\radiusy)$);
\draw[dashed] circle (\radiusx{} and \radiusy);
\end{scope}

\draw[dashed] (0,\height)|-(\radiusx,0) node[right, pos=.25]{$a_n$} node[above,pos=.75]{$r$};

\draw (0,.15)-|(.15,0);
\end{tikzpicture}

\end{center} 
The line segment is set of all points such that, 
\[
	\Gamma = \{s \in \mathbb{R}^{n} : s = \vec{a} t + (1- t ) \vec{x} \text{ where } t \in  [0, 1], \vec{x} \in R, \vec{a} \in \mathbb{R}^{n}\} 
\]
Here $x \in  R$. Every point $\vec{p} \in \Gamma$ is a memeber of the cone.  
\end{tcolorbox}

The volume of the region $\Gamma$ (which is the defined cone) is going to be,
\[
\int_\Gamma 1 = \text{Volume}
\] 
Now the burden is to find a region $\Gamma$. 

The trick we are going to apply is to translate a relatively more simple shape into our cone shape by the following definition. 

\df{
	Define the $D$ region to be $R \times [0,1]$. Volume of this shape is 
	\[
	\int_D 1 = \text{vol}(R)
	\] 
}
We need to find a transformation $T : D \to \Gamma$. From definition of cone we can write, 
\[
	T(x_1, x_2, \ldots, x_{n-1}, x_n) = 
	x_n (\vec{a} - (x_1, \ldots, x_{n-1}, 0 ) ) + 
	(x_1 , \ldots, x_{n-1}, 0)
\]

In vector form, 
\[
T(x_1, \ldots, x_n) = 
\begin{pmatrix} x_n a_1 - x_n x_1 + x_1 \\
\vdots \\
x_n a_{n-1} - x_n x_{n-1} + x_{n-1} \\ 
x_n a_n \end{pmatrix} 
\]

$T$ is injective (check appendix after this problem). So we can compute, 
\[
\mathrm{d} T = 
\begin{bmatrix} 1- x_n & 0 & \cdots & 0 & a_1-x_1 \\ 
	0 & 1-x_n & \cdots & 0 & a_2 - x_2 \\ 
	\vdots & \vdots & \ddots & \vdots & \vdots \\
	0 & 0 & \cdots & 1-x_n & a_{n-1} - x_{n-1} \\ 
	0 & 0 & \cdots & 0 & a_n 
\end{bmatrix} 
\]
$dT $ being an upper triangular matrix, 
\[
\det (\mathrm{d} T) = (1-x_n)^{n-1} a_n
\]
Now through change of coordinates we can find the integral, 
\[
\int_D 1 = \int_\Gamma 1 \cdot T | \det (\mathrm{d} T) | 
\]
Because $1\cdot T = 1$, 
\begin{align*}
	&= \int_\Gamma | \det(\mathrm{d} T) | \\
	&= \int_\Gamma | (1-x_n)^{n-1} a_n | \\
	&= \int_\Gamma (1-x_n)^{n-1} a_n \quad \text{(integration over $x_n > 0$)} \\
	&= a_n \int_\Gamma (1- x_n )^{n-1} \\
	&= a_n \int_{R}^{} \int_{0}^{1} (1- x_n)^{n-1} \mathrm{d} x_n   
\end{align*}
We can carry out the integration using $u = 1- x_n$ substitution method and we can get, 
\[
	a_n \text{vol}(R) \left[\frac{u^{n}}{n}\right]_0^{1}
\]
Hence, 
\[
\boxed{
\frac{a_n}{n} \text{vol}(R)
}
\] 


\subsection*{$T$ is injective}
For this transformation to be valid, $T$ must be injective other than places that are content zero. We can see that $T$ is injective everywhere on it's domain except for the point where the $x_n = 1$. 

$T$ is only valid if $x_n = 1$ ``place" or $R \times \{1\} $ is content zero in $\mathbb{R}^{n}$. 

Consider $f: R \subset R^{n-1} \to \mathbb{R}$ given by $f(\vec{x}) = 1$. Note that the graph of $f$ is precise $R \times \{1\} $. Since our region $R$ is compact we know that the graph of a continuous function is content zero. Therefore $R \times \{1\} $ is content zero. 


\section*{Problem 02} 
Let $D$ denote the region described by the $n$ dimensional solid ellipsoid
\[
\frac{x_1^2}{a_1^2} + \cdots + \frac{x_n^2}{a_n^2} \le 1
\] 
Here $a_i$ are positive. The volume is, 
\[
\int_D 1
\] 
We need to do a transformation now, so define $T : D' \to  D$ defined by
\[
T(x_1, \ldots, x_n) = (a_1 x_1 ,\ldots, a_n x_n)
\] 
where $D'$ is the unit ball in $\mathbb{R}^{n}$. Then note that $T$ is surjective to $D$ and injective on it's domain. 

\[
\text{d}T = 
\begin{bmatrix} a_1 & \cdots & 0 \\ 
\vdots & \ddots & \vdots \\ 
0 & \cdots & a_n \end{bmatrix} 
\]
So easily,
\[
\det (\mathrm{d} T) = a_1 \cdots a_n 
\]
Now we perform the change of coordinates and compute, 
\[
	\int_D 1 = \int_{D'} 1 \cdot T | \det (\mathrm{d} T) | 
\]
Carrying out the computation we simply have, 
\[
	\int_{D'} |a_1 \cdots a_n| 
\]
So from here, 
\[
= (a_1 \cdots a_n) \text{(volume of the unit ball in }\mathbb{R}^{n})
\]
\[
\boxed{
	(a_1 \cdots a_n) \frac{\pi^{\frac{n}{2}}}{\Gamma \left(\frac{n}{2} + 1\right)}
}
\] 

\section*{Problem 03} 
Denote the region enclosed by the blue curve and red line by $D$. The average value of $f$ on this region is, 
\[
\overline{f} = 
\frac{\int_D f}{ \text{vol} (D)}
\] 

We first need to compute $\int_D f$. So consider,
\[
T : D' \to  D \quad 
T(r, \theta) = (r \cos \theta, r \sin \theta)
\] From here we can see, 
\[
| \det (\mathrm{d} T) | = r
\]
We can compute the integral $\int_D f $ as, 
\[
	\int_D  f= \int_{D'} f \cdot T \cdot r  = 
	\int_{D'} (r^2 \cos ^2 \theta + r^2 \sin ^2 \theta) r 
\]
From here, 
\[
	\int_{D'} r^3  = \int_{0}^{2 \pi } \int_{0 }^{\theta} r^3 \mathrm{d} r \mathrm{d} \theta  
\]
Computing the integral gives, 
\[
\frac{8}{5} \pi^5
\]

Now we are required to evaluate the volume. 
\[
	\int_D 1 = \int_{D'} (1 \cdot T) \cdot r = 
	\int_{0}^{2 \pi } \int_{0}^{\theta} r \mathrm{d} r \mathrm{d} \theta 
\] 
Solving this gives us, 
\[
\frac{4}{3} \pi^3 
\] 
So the division of the two results gives us the average value of $f$ on $D$, 
\[
\frac{6}{5} \pi^2
\]

\section*{Problem 04} 
Similar to the first problem let's start by a simple volume first, define the region $R \times [0, 2 \pi]$. From here we define 
\[
	T : R \times [0, 2\pi ] \to D
\]
with 
\[
T(x,y,z) = 
(x , y \cos z, y \sin z)
\] 
Note how $T$ is a surjective map onto $D$ and is injective at every points except the content zero portion that is  $z = 0$ and $z = 2 \pi$. 

We will now compute, 
\[
\mathrm{d} T = 
\begin{pmatrix} 1 & 0 & 0 \\ 
0 & \cos z & - y \sin z \\ 
0 & \sin z & y \cos z \end{pmatrix} 
\]

From here, 
\[
\det (\mathrm{d} T) = 1 \cdot 
\det 
\begin{pmatrix}   \cos z & - y \sin z \\ \sin z & y \cos z  \end{pmatrix} 
\]
\[
\det (\mathrm{d} T) = y
\]
Now we can carry out this integration, 
\[
\int_D 1 = 
\int_{R \times [0, 2\pi ]} 1 \cdot T | \det (\mathrm{d} T) |  = 
\int y = \int_{R}^{} \int_{0 }^{ 2 \pi } y dz   
\]
Solving this get's us, 
\[
2 \pi \int_R y 
\]
Now from here we compute, 
\[
2 \pi \cdot \text{avg of y on R } \cdot \text{vol}(R)
\]
We get,
\[
2 \pi \int_R y
\] 
From this we showed that, 
\[
\text{vol} (D) = \int_D 1 = 2 \pi \text{(avg of y on R)} \cdot \text{vol}(R)
\]

\section{Problem 05} 
The unit ball on $\mathbb{R}^{4}$ is defined by 
 \[
x_1 ^2 + x_2 ^2 + x_3^2 + x_4^2 \le 1
\] 
Let denote $D$ on this region the unit ball, and to find the unit ball volume we find 
$\int_D 1 $. 

The coordinates can be changed by the following 
$T : D' \to  D$ through, 
\[
T(r_1, \theta, r_2 , \theta_2) = 
(r_1 \cos \theta , r_1 \sin \theta_1, r_2 \cos \theta_2, r_2 \sin \theta_2  )
\] 
where $D' = S_1 \times [0, 2\pi], S_2 \times [0, 2\pi] $ such that, $\forall  r_1 \in S$ and  $\forall  r_2 \in S_2$ we have $r_1 ^2 + r_2^2 \le 1$ and $r_1, r_2 \ge 0$. 

\[
\mathrm{d} T = 
\begin{bmatrix} \cos \theta_1 & 
- r_1 \sin \theta_1 & 0 & 0 \\ 
\sin \theta_1 & r_1 \cos \theta_1 & 0 & 0 \\
0 & 0 & \cos \theta_2 & -r_2 \sin \theta_2 \\
0 & 0 & \sin \theta_2 & r_2 \cos \theta_2 \end{bmatrix} 
\]
Now that $\mathrm{d} T$ is a block matrix, therefore we calculate its determinant as, 
\[
\det (\mathrm{d} T) = 
\det 
\begin{pmatrix} \cos \theta_1 & -r_1 \sin \theta_1 \\ 
\sin \theta_1 & r_1 \cos \theta_1 \end{pmatrix} 
\cdot 
\det 
	\begin{pmatrix} \cos \theta_2 & -r_2 \sin \theta_2 \\
	\sin \theta_2 & r_2 \cos \theta_2\end{pmatrix} 
\] 
\[
= (r_1 \cos ^2 \theta_1 + r_1 \sin ^2 \theta_1) \left(r_2 \cos ^2 \theta_2 + r_2 \sin^2 \theta_2\right) = r_1 r_2
\]
Through change of variables, 
\[
	\int_D 1 = \int_{D'} 1 \cdot T | \det (\mathrm{d} T)| 
\] 
\[
	= \int_{D'} r_1 r_2 
\]
This integral can be computed, 
\[
	\int_{D'} r_1 r_2 = 
	\int_{S_1 \times S_2} \int_{0}^{2 \pi } \int_{0}^{ 2 \pi } 
	r_1 r_2 \mathrm{d} \theta_1 \mathrm{d} \theta_2
\] 
\[
	= \int_{S_1 \times S_2} r_1 r_2 (4 \pi ^2)
\] 
Inside this region we have $r_1 ^2 + r_2^2 \le 1$. This is a unit circle and through change of ariables again, 
\[
G: S \to  S_1 \times S_2
\] 
\[
G (R , \alpha) = G(R \cos \theta, R \sin \alpha)
\] where $S = [ 0, 1] \times [0, \frac{\pi}{2} ] $ we constrain the input $\alpha \in  [ 0, \frac{\pi}{2}] $ upto content zero. 

\[
	\mathrm{d} G = \begin{bmatrix} \cos \alpha & - R \sin \alpha \\
	\sin \alpha & R \cos \alpha\end{bmatrix} 
\]
\[
\det (\mathrm{d} G) = R \cos ^2 \alpha + R \sin ^2 \alpha = R
\]
Then by the change of variables it follows, 
\[
	\int_{S_1 \times S_2} r_1 r_2 = \int_S (R \cos \alpha) (R \sin \alpha) \cdot  R
\] 
\[
= \int_{0}^{\pi / 2}  \int_{0 }^{1}  R^3 \cos \alpha \sin \alpha \mathrm{d} R \mathrm{d} \alpha  
\]
Completing this integral we find, 
\[
\boxed{
\frac{1}{8}
}
\] 
Using this in our previous integral, \[
	\int_{D'} r_1 r_2 = 4 \pi ^2 \left(\frac{1}{ 8}\right) = \frac{\pi^2}{2 }
\] 
Hence the volume is $\pi^2 / 2$. 
\end{document}
