\documentclass[letterpaper]{article}
\usepackage[monocolor]{ahsansabit}

\title{Honors Multivariable Calculus}
\author{Ahmed Saad Sabit}
\date{\today}

\begin{document}
\maketitle

\df{
Let's start with a function that maps from $\mathbb{R}^{n}$ to $\mathbb{R}^{m}$. \[
f: \mathbb{R}^{n} \to \mathbb{R}^{m}
\] Then we define a limit of the function as it ``approaches" $\vec{a}$, but $\vec{x} \neq \vec{a}$
\[
\lim_{\vec{x} \to \vec{a}} f(\vec{x}) = \vec{L}
\] means that,
\[
\forall \epsilon > 0, \, \exists \delta> 0
\] such that
if 
\[
|\vec{x}-\vec{a}| < \delta
\] \textbf{then},
\[
\boxed{
|f(\vec{x}) - \vec{L} | < \epsilon
}\] 
}

\begin{figure}[ht]
    \centering
    \incfig{diagram-on-the-limit}
    \caption{Diagram on the limit}
    \label{fig:diagram-on-the-limit}
\end{figure}

\df{
	If $\vec{a}$ is inside of $\mathbb{R}^{n}$ and $r > 0$, then the openball of radius $r$ around $\vec{a}$ is $B_r (\vec{a}) = \{ \vec{x} \in \mathbb{R}^{n} : | \vec{x} - \vec{a} | < r \}$
}

Using this ball definition we can use the limit definition to be, 
\[
	\vec{x} \in B_\delta(\vec{a}) \smallsetminus \{a \}
\] 
Where $f(x) \in B_\epsilon(\vec{L})$

There is also limits of sequences,
\[
	\{ \vec{x_n} \} \in \mathbb{R}^{n}
\] 
\[
\lim_{k \to \infty}  \vec{x_k} = \vec{L}
\] 
$\vec{x_k}$ gets arbitrarily close to $\vec{L}$ when $k$ gets large enough.

What change do you want to do in the definition of the limit so that it can take in account for the sequence limit? I am thinking about the Bolzano-Weirstrass theorem, that the ball around the limit should have infinitely many points. Dr. Wang is like what about $1,0,1,0$ sequence? Then there are still infinitely many points in $1 $ and $0$

Now the definition would be like, starting with $\forall \epsilon>0$ we will have $|\vec{x_k} - \vec{L}|$ is smaller than $\epsilon$. The change for sequence is we need to define a large number $k > N$. $N \in \mathbb{Z}^{+}$, and you have liberty to pick whatever $N$ you want. (I remember reading this in the first few chapters of Serge Lang complex analysis). 


Now we are talking about sequences. But they are kinda single variable. But the definition of limit can be expressed through sequences. 

\df{
Proposition. 
\[
f : \mathbb{R}^{n} \to \mathbb{R}^{m}
\] So, as,
\[
\lim_{\vec{x} \to \vec{a}}  f(\vec{x}) = \vec{L}
\] if and only if, for all sequeneces $\{\vec{x}_k\}$ that converges to $\vec{a}$ where $\vec{x}_k \neq \vec{a}$ for all $k$. We have $\{f(\vec{x}_k)\} \to  \vec{L}$.  
}

\pf{ Forward direction proof.
Suppose the above limit approaching $\vec{L}$, we want to show that for any sequence converging to $\vec{a}$, we get $f(\vec{x})$ converge to $\vec{L}$.  Suppose we have the sequence that converges. 

Let $\epsilon > 0$, because we know $\lim_{\vec{x} \to \vec{a}} f(\vec{x}) = \vec{L}$, we have a $\delta>0$ such that, if $|\vec{x} - \vec{a}|$ is less than $\delta$, then $f(\vec{x}) - \vec{L}$ is less than $\epsilon$.  

Since $\vec{x_k}$ converges to $\vec{a}$, we know $\forall N \in \mathbb{Z}$, such that $k> N$, $|\vec{x}_k - \vec{a} | < \delta$. Hence if $k > N$, then $|f(\vec{x_k}) - \vec{L} | < \epsilon$.

Proof in other direction. 
}

\pr{\[
		\lim_{(x,y) \to (0,0)}  \frac{y}{\sqrt{x^2 + y^2} }
	\]
}
\solu{
This function, if you take a sequence $x_k$ along $x$-axis then from $\infty $ to $0$, for all $x_k$ you get $0$. For a vertical line it's always $1$. So limit does not exist. Graphing this is a warp at the origin.
}

For a function $f: \mathbb{R} \to  \mathbb{R}$ and $f(x) = x^2$. 
\df{
	The graph of a function $f: D \to  \mathbb{R}^{m}$ where $D \subset \mathbb{R}^{n}$ is the set of points $\{(\vec{x}, f(\vec{x})): \vec{x} \in D \} \subset \mathbb{R}^{m+n}$ 
}
Now, $p(t) = (\cos t, \sin t)$, can be thought of as a circle, but this is not the graph. The graph is, 
$\{ (t, \cos t, \sin t) \}$ 

\df{
Laplace the domain of representation is something like,
\[
\frac{Y(s)}{F(s)}
\] 
}
\begin{figure}[ht]
    \centering
    \incfig{this-is-a-diagram-for-aaron-cole}
    \caption{This is a diagram for Aaron Cole}
    \label{fig:this-is-a-diagram-for-aaron-cole}
\end{figure}
\end{document}
