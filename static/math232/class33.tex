\documentclass[letter]{article}
\usepackage[monocolor]{ahsansabit}

\title{Honors Multivariable Calculus : : Class 33}
\author{Ahmed Saad Sabit, Rice University}
\date{\today}

\begin{document}
\maketitle
The 212 definition of scalar surface integral. Say that is in $\mathbb{R}^{3}$. Let there be a function $f : \mathbb{R}^{3}\to \mathbb{R}$. Then, 
\[
\iint_S f \, \mathrm{d} S
\]  
is ``defined" using Riemann Sums. Break S into pieces. For each piece take a value of $f$ on that piece. So the integral is going to be the 
small area times the function. And then add that up. That gives you a Riemann sum and consider the number of pieces goes to infinity.

A use of this is that integrating $1$ like $\iint 1 \, \mathrm{d} S$ gives the total surface area. More generally, if $\delta$ is a per area density,
then it gives the total. The average of $f$ is 
\[
\frac{\iint f \, \mathrm{d} S}{ \iint \mathrm{d} S}
\] 

\df{
$S \subset \mathbb{R}^{3}$ is a parametrized $C^{1}$ surface if there exists 
\[
T : D \to \mathbb{R}^{3}
\]
and $D \subset \mathbb{R}^2$ has non empty interior such that $T$ is $C_1$, the image of $T$ is $S$ and $T$ is injective other than a set of content zero. 
}

Here if $S$ is a compact $C_1$ parametrized surface in $\mathbb{R}^{3}$ with parametrization $T: D \to \mathbb{R}^{3}$ then define $\iint_S f \, \mathrm{d} S$ to be just 
\[
\iint_D f\cdot T \text{ (change of coordinate )} 
\]
\[
\iint_D f \cdot T \,  \left| \frac{\partial T}{\partial u} \cdot \frac{\partial T}{\partial v} \right| 
\]

Let's take a 
\[
	\{(x,y,z) : z = x^2 + y^2 \, \& \, z \le 4\} 
\]
Let's set $f(x,y,z) = z$ and $\iint_S f \, \mathrm{d} S$, let's set
\[
u = r \quad \text{and} \quad v = \theta 
\]
\[
x = u \cos v
\]
\[
y = u \sin v
\]
\[
z = x^2 + y^2 = u^2 
\]
\[
T(u,v) = ( u \cos v, u \sin v, u^2)
\]
\[
0 \le u \le 2
\]
\[
0 \le v \le 2 \pi
\]

\[
	\iint f \, \mathrm{d} S = \int_{[0,2]\times [0, 2\pi ] } 
	u^2 \sqrt{2 u ^{4} + u^2} \, \mathrm{d} u \, \mathrm{d} v 
\]

Another type of parametrization, 
\[
T(u,v) = (u,v, u^2 + v^2)
\]
Then this becomes, 
\[
\iint_D (u^2 + v^2) \left| \frac{\partial T}{\partial u} \times  \frac{\partial T}{ \partial v} \right| \, \mathrm{d} u \, \mathrm{d} v
\]
Here $D$ is a disk of radius $2$.
\[
\iint_D (u^2 + v^2) \sqrt{4 u^2 + 4 v^2 + 1}  \, \mathrm{d} u \,  \mathrm{d} v
\]

\section*{Vector Field} 
Every point in space is a vector. 
\[
\vec{F} : \mathbb{R}^{n} \to \mathbb{R}^{n} 
\] 
Inputs are thought of as points and outputs are thought of as arrows. 
An easy example, 
\[
F(x,y) = \langle 1 , 3 \rangle 
\]



\end{document}
