\documentclass[letter]{article}
\usepackage[monocolor]{ahsansabit}

\title{Honors Multivariable Calculus : : Class 14}
\author{Ahmed Saad Sabit, Rice University}
\date{\today}

\begin{document}
\maketitle
For $f:\mathbb{R}^{n}\to \mathbb{R}$ the first order derivatives are 
\[
\frac{\partial f}{\partial x_1} , \frac{\partial f}{\partial x_2}
\] There are are all $\mathbb{R}^{n}$ to $\mathbb{R}$.

Differentiate $\partial f_{x_i}$ with respect to $x_j$, so 
\[
\frac{\partial}{\partial x_j} \left( \frac{\partial f}{\partial x_i}\right)
\] In more condensed form 
\[
	\frac{\partial ^2 f}{\partial x_j \partial x_i}= \left(f_{x_i} \right) _{x_j}
\]

An example can be 
\[
f(x,y) = x \sin (y-x) + y \sin (x+y) - y
\] 
\[
f_x = \sin (y-x) + x \cos (y-x) (-1) + y \cos (x+y) \cdot 1
\]
\[
f_y = x \cos (y-x) + \sin (x+y) + y \cos (x+y) \cdot  1 - 1
\] 
\[
	f_{x x}= \cos (y-x) \left(-1\right) - \cos (y-x) + x \sin(y-x) (-1) - y \sin(x+y)
\] 
\[
	f_{y y } = -x \sin (y-x) + \cos (x+y) + \cos (x+y) - y \sin(x+y)
\] 
\[
f_{xy}= \cos(y-x) + x \sin(y-x) + \cos(x+y) - y\sin(x+y)
\]
\[
	f_{yx}= \cos(y-x) + x \sin(y-x) + \cos(x+y) - y\sin(x+y)
\] 

\section*{Clairaut's Theorem}
\thm{
If $f$ is $C^{2}$, not only the partial derivatives exist but also the function is continuous. So at some point $\vec{a}$ then mixed partials are equal.
\[
\frac{\partial^2 f}{\partial x_i \partial x_j} = \frac{\partial^2 f}{\partial x_j \partial x_i}
\] 
}

\pf{
For simplicity write $f$ as $f(x,y)$. Let 
\[
\vec{a} = (x_0, y_0)
\] 
Define 
\[
S (\Delta x, \Delta y) = f(x_0 + \Delta x, y_0 + \Delta y) - f(x_0 + \Delta x, y_0	) 
- (f(x_0, y_0 + \Delta y) - f(x_0, y_0))	
\]

Define another function $g(x)$ 
\[
g(x) = f(x, y_0+\Delta y) - f(x, y_0)
\]
Then 
\[
S (\Delta x , \Delta y) = g(x_0 + \Delta x) - g(x_0)
\] 
Using the mean value theorem on $G$ because it is a single variable function, and $g$ is also differentiable because $f$ is differentiable and we have single order and this is going to be
\[
= \Delta x g'(c) \text{ for some $c$ between $x_0$ and $x_0 + \Delta x$}
\]
\[
= \Delta x 
\left(
f_x (c, y_0 + \Delta y) - 
f_x (c, y_0) 
\right) = 
\Delta x 
\left(
	\Delta y f_{xy} (c, d ) 
\right)
\text{ for some $d$ between $y_0$ and $y_0 + \Delta y$ as Mean Value Theorem}
\]
So 
\[
\frac{S (\Delta x, \Delta y)}{\Delta x \Delta y} = 
f_{xy} (c,d)
\] 
Here $c,d$ is some coordinate in the rectangle. But as $\Delta x, \Delta y \to  0$ we need $c,d \to  0$ and hence the position become $\vec{a} = (x,y)$. Using continuity 
\[
	f_{xy} (c,d) \to f_{xy} (\vec{a})
\] This is the limit as $\Delta x, \Delta y \to  0$ as $\lim_{\Delta x, \Delta y \to 0} S (\Delta x, \Delta y) / \Delta x \Delta y = f_{xy}(\vec{a}) $. We can do the exact same process in the opposite order we can simply form the $f_{yx}$
}

Partial differentiation draw a point $\vec{a}$ and extend it two both directions by $\Delta x$ and $\Delta y$, so you have a rectangle of $\Delta x$ and $\Delta y$ dimensions. 

\section*{Taylor Polynomial}
Whats the second order Taylor Polynomial for $f$ at $x=a$? 

\[ f(x) \approx 
f(a) + \frac{f'(a) }{1 ! } (x-a) + 
\frac{f''(a) }{2! }(x-a)^2
\] When $x $ is near $a$. Why is the $2 !$ there? Because we want the second derivative to match the left side of the equation.  

\end{document}
