\documentclass[letter]{article}
\usepackage[monocolor]{ahsansabit}

\title{}
\author{Ahmed Saad Sabit, Rice University}
\date{\today}

\begin{document}
\maketitle
\df{
Suppose $g:\mathbb{R}^{n}\to \mathbb{R}$ is $C^{1}$. Let 
\[
X = \text{ Level Surface } g^{-1}(\{c\} ) = 
\{\vec{x} \in \mathbb{R}^{n} : g(\vec{x}) = \vec{c}\} 
\] 
Take $f:\mathbb{R}^{n}\to \mathbb{R}$ and we say that $\vec{a} \in X$ is ``interesting" for $f$ on $X$ if any one of the following is
\begin{itemize}
	\item $f$ not differentiable $\vec{a}$ 
	\item $\nabla g(\vec{a}) = \vec{0}$ 
	\item $\nabla f(\vec{a}) = \lambda \nabla g (\vec{a})$ for some scalar $\lambda$. 
\end{itemize}
}

\thm{
If $\vec{a} \in X = g^{-1} \left(\{c\} \right)$ is uninteresting for $X$, then 
$\forall r >0$, $\exists \vec{b} \in  X$ with $| \vec{b} - \vec{a}| < r$ and $\vec{c} \in X$ with $|\vec{b_2} - \vec{a}| < r$ such that
\[
f(\vec{b_1}) > f(\vec{a}) > f(\vec{b_2})
\] 
on $X$ $\vec{a}$ isn't a local min/max for $f$ ". 
}

\section*{Example}
\[
f(x) = 4x + 3y 
\]
On $x^2 + y^2 = 1$. Is $f$ ever going to be non differentiable. Now what is $g$? It's $g(x,y)= x^2 + y^2 $ and $X = g^{-1}(\{1\} )$. $f$ always being differentiable, so 
\[
\nabla g = \langle 2x, 2y \rangle
\]
This is never $\vec{0}$ on $X$. So only interesting points are at
\[
\nabla f = \lambda \nabla g
\] 
\[
\langle 4,3 \rangle = \nabla \langle 2x , 2y \rangle
\]
\[
4 = 2 \lambda x
\] 
\[
3 = 2 \lambda y
\] 
And 
$x^2 + y^2 = 1$. Solve these for interesting points $\frac{4}{3} = \frac{x}{y}$ and 
\[
4y = 3x
\]
\[
 y= \frac{3}{4} x
\] 
\[
x^2 + \frac{9}{16} x^2 = 1
\] 
So we can have
\[
2 S x^2 \frac{1}{16} = 1
\] 
\[
x = \pm \frac{4}{5}
\] 
\[
y = \pm \frac{3}{5}
\] 
We know $f$ attains its min and max on $X$ since is a cpt and $f$ is continuous. So max at $4 / 5 , 3 / 5$ and min at other. 

\section*{Example}
\[
f(x,y,z) = x^2 - y + z
\] 
on $X$ the unit sphere $x^2 + y^2 + z^2 = 1$, 
We call that $g(x,y,z)$. So,
\[
\nabla g = \langle 2x, 2y, 2z \rangle
\] 
\[
\nabla f = \langle 2x, -1, 1 \rangle
\] 
Now what we get is, and required to solve $\lambda$, 
\[
2x = \lambda 2x
\]
Gives above either $\lambda = 1$ or $x = 0$. 
\[
-1 = \lambda 2y
\] 
\[
1 = \lambda 2 z
\] 
\[
x^2 + y^2 + z^2 = 1
\]
Case 01 : $x = 0$ from there 
\[
	{-1 \over 1} = { \lambda 2 y \over  \lambda 2 z } 
\] 
\[
z = - y 
\] 
Using this 
\[
0 ^2 + y^2 + (-y)^2 = 1 \implies y = \pm \frac{1}{\sqrt{2} } \text{ and } z = - \left(\pm \frac{1}{\sqrt{2} }\right)
\] 

Case 02: $\lambda  = 1$ then from there we can find 
\[
y = - \frac{1}{2}, z = \frac{1}{2}
\] 
\[
x^2 + \frac{1}{4} + \frac{1}{4} = 1
\] 
\[
x = \pm \frac{1}{\sqrt{2} }
\] 
$2$ more interesting points are 
\[
	\left(
\pm \frac{1}{\sqrt{2} }, - \frac{1}{2}, \frac{1}{2}
	\right)
\]

$X$ is compact then we will get min max. 

\end{document}
