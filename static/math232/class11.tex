\documentclass[letter]{article}
\usepackage[monocolor]{ahsansabit}

\title{Honors Multivariable Calculus : : Class 11}
\author{Ahmed Saad Sabit, Rice University}
\date{\today}

\begin{document}
\maketitle 

Partial derivatives of $f$ at $\vec{a}$ with respect to ${x}_i$ is 
\[
	D_{{e}_i} f(\vec{a}) 
\]
More common notation is $ \frac{\partial f}{\partial x_i} (\vec{a})$
Sometimes you will see 
\[
f(x,y) = z
\] 
So 
\[
\frac{\partial z}{\partial x} = f_x
\]
An example can be $f(x,y) = (\sin y + x^2 e^{y} , x+ 2 xy)$ where $f:\mathbb{R}^{2} \to \mathbb{R}^2$ and we want $\frac{\partial f}{\partial x}$ at $(2,0)$?

We can do this on the long way so by definition 
\[
\lim_{t \to 0} \frac{f(2,0) + t \vec{e}_1 - f(2,0)}{t}
\] 
\[
\lim_{t \to 0} \frac{f(2+t,0)   - f(2,0)}{t}
\] 
\[
\lim_{t \to 0}  
\frac{
\begin{pmatrix} 0 + (2+t)^2 \\ 2 + t + 2 (2+t) \cdot 0 \end{pmatrix} 
- 
\begin{pmatrix} 4 \\ 2 \end{pmatrix} 
}{t}
\]
We have a simple single variable derivative. 
\[
\begin{pmatrix} 
\lim_{t \to 0} \dfrac{(2+t)^2-4}{t} 
\\
\lim_{t \to 0} \dfrac{2+t+2(2+t)0 - 2}{t}
\end{pmatrix} 
\]
This just boils down into treating the individual components as individual derivatives. 
\[
\frac{\partial f}{\partial x} (2,0) = \begin{pmatrix} 4 \\ 1 \end{pmatrix} 
\]
Just doing single variable we can find 
\[
\frac{\partial f}{\partial y} (2,0) = \begin{pmatrix} 5\\4 \end{pmatrix} 
\]
These are directional derivative along $\vec{e}_1$ and $\vec{e}_2$. 
\[
	\frac{\partial f}{\partial x}(2,0) = \begin{pmatrix} 4 \\ 1 \end{pmatrix}  = D_{\vec{e}_1} f(2,0) 
\] 
\[
	\frac{\partial f}{\partial y} (2,0) = \begin{pmatrix} 5\\4 \end{pmatrix} = D_{\vec{e}_2} f(2,0)
\]
If $f$ is differentiable and $df_(2,0)$ is represented by matrix $M$ then 
 \[
	 M = \begin{pmatrix} 4 & 5 \\ 1 & 4 \end{pmatrix} 
\]
\[
d f_{(2,0)} (\vec{e}_1) = M \vec{e}_1
\]
\[
	d f_{(2,0)} (\vec{e}_2) = M \vec{e}_2
\]
If $f$ is differentiable at $\vec{a}$ then $Mx$ for $d f_{\vec{a}}$ is 
\[
\begin{pmatrix} 
	\frac{\partial f}{\partial x_1} (\vec{a}) & \ldots& \frac{ \partial f }{\partial x_n} (\vec{a})
\end{pmatrix} 
\] Consider the vertical spanning of this matrix too. We have columns here beware! 

If $f(\vec{a})$ is $( f_1(\vec{a}), f_2(\vec{a}), \ldots, f_m(\vec{a}))$ if $f$ is differentiable at $\vec{a}$ then 
\[
	d f_{\vec{a}} 
	= 
	\begin{pmatrix} \dfrac{\partial f_1}{\partial x_1}(\vec{a}) & \cdots & \dfrac{\partial f_1}{\partial x_n}(\vec{a})\\
		\vdots & \ddots & \vdots \\
	\dfrac{\partial f_m}{\partial x_1} (\vec{a}) & \cdots & 
	\dfrac{\partial f_m}{\partial x_n} (\vec{a})
	\end{pmatrix} 
\]
The $ij$ entry is $ \frac{\partial f_i}{ \partial x _ j} (\vec{a})$
So $f$ is differentiable at $\vec{a}$, then this means $f$ has directional derivatives in all directions of $\vec{a}$, then this also means $f$ has partial derivatives along all $n$ basis directions. 

\[
f(x,y) = \sqrt{|xy|} 
\] 
Does not have directional derivatives (wait how why)

\section*{Partial derivatives around a region} 
\thm{
If $ \frac{\partial f_i}{ \partial x_j}$ all exist on some neighborhood of $\vec{a}$ and are continuous there (neighborhood basically means some small open set ball containing $\vec{a}$), the $f$ is differentiable at $\vec{a}$. 
}
\pf{
Lemma: If $f:D\to \mathbb{R}^{m}$ and $f(\vec{a}) = (f_1(\vec{x}), \ldots, f_m(\vec{x}))$ then $f$ is differentiable at $\vec{a}$ if and only if all $f_i$ are differentiable at $\vec{a}$. (Left as homework)

For simplicity $n = 2$ by the lemma can just work with single coordinate function, so will take $f: \mathbb{R}^2 \to \mathbb{R}^{1}$. We will later find out how this generalizes to $\mathbb{R}^{n}$. We are assuming that $ \frac{\partial f}{\partial x}$ is continuous at $\vec{a}$. I will write $f_x$ for now, hence, 
$f_x$ is continuous at $\vec{a}$ and also $f_y$. So we are trying to show that 
\[
\lim_{\vec{h} \to \vec{0}} \frac{f(\vec{a}+\vec{h}) - f(\vec{a}) -L(\vec{h})}{|\vec{h}|} = \vec{0}
\]
$L(\vec{h})$ is what we think as the derivative. Here this $L(\vec{h})$ should be the matrix 
\[
	L = \begin{pmatrix} f_x(\vec{a}) & f_y(\vec{a}) \end{pmatrix} 
\]
\[
\text{LHS} = 
\frac{
|
f(\vec{a}+\vec{h}) -
f(a_1 + h, a_2) + 
f(a_1 + h_1, a_2) 
-
f(a_1, a_2) 
- L(h_1, h_2)
}{\sqrt{h_1^2+h_2^2} }
\]
We will consider there is a mean value theorem being applied hence a point in between
}

\begin{figure}[ht]
    \centering
    \incfig{showing-continuity-through-partial-derivatives}
    \caption{Showing continuity through partial derivatives}
    \label{fig:showing-continuity-through-partial-derivatives}
\end{figure}
\end{document}
