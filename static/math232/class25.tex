\documentclass[letter]{article}
\usepackage[monocolor]{ahsansabit}

\title{Honors Multivariable Calculus : : Class 25}
\author{Ahmed Saad Sabit, Rice University}
\date{\today}

\begin{document}
\maketitle
For all $a$ and for all $b$ there exist a $c$ is a (statement $S$ involving $a,b,c$ ). 
For all $b$ there exists $c$ such that  for all $a$
Which statement might be easier to prove here? Statement $2 $ is harder to prove. 

The difficulty in $\epsilon-\delta$ is choosing the $\delta$. What is the statement $f$ is continuous on $D$ of the form? Most likely statement 1. That's for because $\forall x$ and $\forall \epsilon>0$ there exists $\exists \delta >0$ such that - something. 

\df{
$f$ is uniformly continuous on $D$ if $\forall \epsilon>0$, $\exists \delta>0$ such that $\forall  x \in D$ if $|\vec{y}-\vec{x}| < \delta$ then $|f(\vec{y}) - f(\vec{x})| < \epsilon$. 
}

An example can be $f(x) = 4x$ to be uniform continuous. 
\pf{
Let $\epsilon > 0$ and choose $\delta = \frac{\epsilon}{4}$, then if $x,y \in  \mathbb{R}$ with $|y-x| < \delta = \frac{\epsilon}{4}$ , we have,
\[
|f(y) - f(x)| = |4y - 4x| = 4 |y-x| < \epsilon
\] 
}

Another example, $f(x) = x^2$ is not a uniformly continuous on $\mathbb{R}$. Note about the $\Delta x$ for $\Delta f$, moving to the left, we need narrower and narrower tolerance around $x$ that $\Delta x$ gets smaller. So there is not a single $\delta$ that can be universally okay. 

\pf{ We want to find one counter example which will totally negate the statement. 

Choose $\epsilon = 1$. Then we are trying to claim that there is no $\delta$ that works. So for any $\delta > 0$ we can find an $x$ and choose it such that $x $ is greater than $\frac{1}{ \delta}$. Then what happens when we take $y = x +\frac{\delta}{2}$ then we get 
\[
f(y) - f(x) =  
\left(x + \frac{\delta}{2}\right)^2 - x^2 = \delta x + \frac{\delta^2}{4} > \delta x > \epsilon 
\] 
}

\thm{
If $D$ is compact and $f$ is continuous on $D$ then $f$ is a uniformly continuous on $D$. 
}
This is a happy fact. 
\pf{Analysis}

\thm{Proposition: 
If $f$ is continuous on a box $D$ then $f$ is integrable on $D$.  
}
Here $f : D \to \mathbb{R}$ and $D \in \mathbb{R}^{n}$. 
\pf{
Let $\epsilon > 0$. Define $\epsilon ' = \frac{\epsilon}{ \text{vol of } D}$. Uniform continuity of $f$ on $D$ means there exists an $\delta >0$ that $\vec{y},\vec{x} \in D$ with $|\vec{y} - \vec{x}| < \delta$ then $|f(\vec{y}) - f(\vec{x})| < \epsilon'$. Pick a partition $P$ such that $\vec{x},\vec{y}$ are in the same piece $P$ then $|\vec{y}-\vec{x}| < \delta$. 

So on each piece the max of $f$ subtracted from $min$ value of $f$ : 
\[
U(f, P) - L(f, P)  < 
\sum_{\text{pieces}}^{} \text{(vol of piece)} \text{(max value of f on piece - min value of f on piece)} < 
\]\[
< \sum_{\text{pieces}}^{} \text{(vol of piece)} \epsilon' = \text{vol}(D) \epsilon' = e
\]  
}

Before we move to non-boxes, what about $f$ is non continuous? 

\df{
A set $X \subset \mathbb{R}^{n}$  has content $0$ or content zero if $\forall \epsilon > 0$ $\exists \text{finitely many boxes } B_1, \ldots, B_k$ such that $x \subset \cup B_i$ and 
\[
\sum_{i = 1}^{k} \text{vol}(B_i) < \varepsilon
\] 
}
\[
\subset \cup \, \cap \in 
\]
\thm{Proposition: 
If the set of the discontinuities of $f$ on the box $D$ is content zero, then $f$ is integrable on $D$. 
}
When we are trying to integrate functions it's important to remember that our functions are bounded.

\pf{
$D$ and we are not assuming $f$ is continuous. In the box $D$ imagine some line where $X$ is the set of discontinuities. Choose $P$ partition such that the pieces of $P$ that intersect $X$ that have total volumen $< (\text{fill in the blank later})$. (by $X's$ content zero.)  

$f$ is uniformly continuous outside of those boxes, choose $P$ also such that if $\vec{y}, \vec{x}$ are in a single piece of outside of these boxes then $|f(\vec{y}) - f(\vec{x})| < (\text{fill in box})$.

Then 
\[
U(f, P) - L(f, P) = \sum_{n=1}^{\infty} \text{(vol of piece)} \text{(min - max)}
\]
\[
= \sum_{\text{piece that contain X}}^{} \text{(vol)  } |\text{min - max} |  + 
\sum_{\text{others}}^{} \text{(vol)}|\text{min - max}| 
\] 
Now the boxes around the discontinuous part can be taken really small though the min - max would not be ssmall. 
\[
< \sum_{\text{pieces containing $X$}}^{} \text{(vol)}\text{(overall max - over min of f on D)} + 
\sum_{\text{other pieces}}^{} \text{(vol) } \epsilon' \]\[ < 
\text{(overall max - overall min)  } \epsilon'' + (\text{vol} D) e' < \frac{\epsilon}{2} + \frac{\epsilon}{2}
\] 
} 
\end{document}
