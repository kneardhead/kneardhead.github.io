\documentclass[letter]{article}
\usepackage[monocolor]{ahsansabit}

\title{Honors Multivariable Calculus : : Class 27}
\author{Ahmed Saad Sabit, Rice University}
\date{\today}

\begin{document}
\maketitle
	
Continuous function inside a compact set is uniformly continuous. 
\thm{
If $f $ is continuous on a box 
\[
	B= [a_1, b_1] \times [a_2,b_2] \times  [a_3,b_3]
\] then setting $S(x,y) = \int_{a_3} ^{b_3} f(x,y,z) \mathrm{d} z$ we have 
\[
	\int_B f = \int_{[a_1, b_1] \times [a_2, b_2] } S(x,y)
\] 
This is called Fubini's Theorem.}
Sketch of a proof. First thing is we need to say $S$ is integrable ($S(x,y)$). We will show that $S$ is continuous then it will be integrable on a rectangle. 
Why is $S$ continuous? We say, 
$(x_1, y_1) = (x + \delta x , y + \delta y)$ here $\delta x  , \delta y$ are small. We want to compare $f(x,y,z)$ to $f(x_1,y_1, z)$. Let's claim these functions are close. 
This is required to be universal closeness. We need ``uniform continuity" of $f$ on $B$ if this distance between $\delta x, \delta y$ are small enough then $f(x_1,y_1,z)$ and $f(x,y,z)$ are close within some $\epsilon'$. So we have $f(x_1, y_1, z) - \epsilon' < f(x_1,y_1,z) < f(x,y,z) + \epsilon'$ for all $z$. Thus here the 
\[
	\int_{a_3}^{b_3} f(x_1 y_1 , z) - \epsilon' \mathrm{d} z < 
	\int f(x_1, y_1 , z) \mathrm{d} z  
	< 
	\int f(x_1, y_1, z) + \epsilon ' 
\]
So we have 
\[
S(x,y) - \epsilon' (b_3 - a_3) < S(x_1,y_1) < S(x_1, y) + \epsilon'(b_3, -a_3)
\]
So $S$ is continuous hence integrable. From there, now. To prove. 

So if division of the the ``mesh" is fine enough, then $\int_{B'} S(x,y)$ is close to any Riemann Sum 
\[
\sum_{n}^{} s(x_n , y_n ) \Delta x \Delta y = 
\sum_{}^{} 
\left(\int f dz\right) \Delta x \Delta y
\]
Remember
\[
\int_B f  \text{ is close to } \sum_{i}^{} \left(
\sum_{j}^{} f(x_i , y _i , z_j) \Delta x \Delta y \Delta z
\right) 
= 
\sum_{i}^{} \left(
\sum_{j} f(x_i, y_i, z_j) \Delta z
\right) \Delta x \Delta y
\]
We have a similarity to the equation we had built, 

\[
\sum_{n}^{} s(x_n , y_n ) \Delta x \Delta y = 
\sum_{}^{} 
\left(\int f dz\right) \Delta x \Delta y
\]
Fixed the notation to be exact what professor wrote (couldn't type properly in class).
\[
\sum_{i}^{} s(x_i , y_i ) \Delta x \Delta y = 
\sum_{i}^{} 
\left(\int_{a_3}^{b_3} f(x_i, y_i, z) dz\right) \Delta x \Delta y
\]

Here's an idea
\[
	\int_{a_3}^{b_3} f(x_i, y_i, z) \, \mathrm{d} z = 
	\sum_{j}^{} 
	\left(
		\int_{s_{j-1}}^{s_j} f(x_i, y_i, z)\, \mathrm{d} z
	\right)
\]
There is a mean value theorem for integration, 
\[
	\int_a^b g(x) \mathrm{d}  x = (b-a) \text{ arg value of } g  \text{ on } [a,b] = (b-a) g(c) 
\] for some $c$ and $g$ is continuous. 

Not in a box? Then 
\[
\int_R f = \int_B f \cdot \xi_R
\]
Here $\xi = 1$ if $ \vec{x} \in  R$ and $\xi = 0$ if not.

\end{document}
