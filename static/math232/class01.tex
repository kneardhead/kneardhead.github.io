\documentclass[letter]{article}
\usepackage[monocolor]{ahsansabit}

\title{Honors Multivariable Calculus : : Class 01}
\author{Ahmed Saad Sabit, Rice University}
\date{\today}

\renewcommand{\vec}{\mathbf}
\begin{document}
\maketitle

\nt{
This text is being edited a lot later than the first class. Well, I didn't have any respectable system running to take notes in class that day. 
}
Rigorous definition of a Limit is necessary 

\section*{Single Variable Limit} 
\df{
Let there be a function $f:\mathbb{R} \to \mathbb{R} $ and let there exist a real number $a$. The number $L$ is defined to be the limit $L$ 
\[
\lim_{x \to a} f(x) = L
\] 
meaning that for all $\epsilon > 0$, there exists some $\delta > 0$ such that if 
\[
\left| x- a \right| < \delta 
\] 
then
\[
\left| f(x) -L \right| < \epsilon
\]
A requirement is that $x \neq a$. Hence we say $L$ is the limit of $f(x)$ if $x$ ,,approaches" $L$.  
}
We will later see if $x = a$ and $f(x) = L$ then it's definition of being continuous. We have defined closeness with a tricky way, we can pick $\epsilon$ to be anything above $0$. This can be infinitely large, this can be infinitely close to $0$. That's the beauty of this idea, it just nicely defines what's small without having to create weird abstractions. 

I was thinking hard about limits when I was pretty young trying to define smallest limit. This brings the whole thing to perspective.

\newpage \section*{Multi Variable Limit}
\subsection*{Multivariable Function} 
We can have single variable function like 
\[
f(2) = 3
\] 
But what is a vector function? Well this one takes input vectors and spits out another vector, not necessarily the same family of vector. Of course this is a linear map, or linear transformation, from either one space to another, or to itself. 

\[
f \left(
\begin{bmatrix} 2 \\ 1\\ 4\\ 0 \\1 \end{bmatrix} 
\right)= 
\begin{pmatrix} 7 \\ 1 \end{pmatrix} 
\]
This specific function takes members of $\mathbb{R}^{5}$ and gives out $\mathbb{R}^{2}$. 
\[
f : \mathbb{R}^{5}\to \mathbb{R}^{2}
\]
We can break it down like this
\[
f(\vec{x}) = \begin{pmatrix} f_1(\vec{x}) \\ f_2(\vec{x}) \end{pmatrix} 
\]
The good thing about $f_1, f_2$ is
\[
f_1, f_2 : \mathbb{R}^{5} \to \mathbb{R}^{1}
\] 
They take vectors and spit out simple numbers (scalars). The $f_1$ and $f_2$ can be something like 
\[f_1(\vec{x}) = 
f_1(x_1,x_2,x_3,x_4,x_5) = 2x_1 - x_5 + x_3 = 7
\]
Similarly, taking a list $f_2$ behaves
\[
f_2(\vec{x}) = 
f_2(x_1,x_2,x_3,x_4,x_5) = x_1 + x_2 + x_3 + x_4 - 6 x_5  = 1
\] 


Multivariable limit is basically instead of dealing with $\mathbb{R}$, we deal with $\mathbb{R}^{n}$ where the inputs are vectors $\vec{v}$ and outputs are $\mathbb{R}^{m}$ members like $\vec{u}$. Concept of limit is basically broken down into components of the vectors. 

\df{
Let there be a function $f:\mathbb{R}^{n} \to \mathbb{R}^{m}$ and points $\vec{a} \in \mathbb{R}^{n}$ and $\vec{L} \in \mathbb{R}^{m}$. The statement 
\[
\lim_{\vec{x} \to \vec{a}}  f(\vec{x}) = \vec{L}
\] meaning that for all $\epsilon  > 0$, there exists $\delta > 0$ such that if $|\vec{x}-\vec{a}| < \delta $ and $|f(\vec{x}) - \vec{L}| < \epsilon$ with the requirement $\vec{x}\neq \vec{a}$. 
}

\thm{ 
Suppose $f: \mathbb{D} \to \mathbb{R}$ is a vector function of $m$ dimension. This means, 
\[
f = (f_1, \ldots, f_m) \in \mathbb{R}^{m}
\]
Let a member of $\mathbb{D}$ be $\vec{a}$. This $\vec{a}$ itself might be a vector if $D \subset \mathbb{R}^{n}$ with $n$ dimensions. Now if $\vec{L} \in \mathbb{R}^{m}$, then 
\[
\lim_{\vec{x} \to \vec{a}} f(\vec{x}) = \vec{L}
\] 
If and Only If 
\[
\lim_{\vec{x} \to \vec{a}} f_i(\vec{x}) = L_i
\] 
Where $i = 1, \ldots, m$. 
}

\nt{The funny thing is we can also treat that like a theorem and prove it using single variable idea and introducing the idea of a norm $|\vec{v}|$. For this theorem, additionally it's important to consider $\vec{a}$ is a limit point. That means $\vec{a}$ isn't a boundary point (at the edge)or something and has infinite points around a small circle around. I will go indepth on this later using the idea of a Ball. For now consider $\vec{a}$ is well behaved. 
	}
\pf{
	Consider this 
	\[
	\lim_{\vec{x} \to \vec{a}} f(\vec{x}) = \vec{L}
	\] 
	This means that there is a $\epsilon>0$ such that 
	\[
	| f(\vec{x}) - \vec{L} | < \epsilon
	\] 
	From the definition of limit we say in the last section, well, the limit implies there must be a $\delta$ such that 
	\[
	| \vec{x} - \vec{a} | < \delta
	\] 
So having $\epsilon$ we also must have $\delta$ follow the rule. Because it's ,,if and only if", it also implies if $\delta$ exists, then so as the bound $\epsilon$. 

\textbf{Vector function limit implies Component Limit}: let's say $| \vec{x} - \vec{a}| < \delta$. Then if component limit exists, we need a bound like. 
\[
|f_i(\vec{x}) - L_i | < \epsilon
\] We are going to prove this bound $\epsilon$ is existent. 

Well we know that a component must be either equal or smaller than the vector it forms
\[
|\alpha_i| \le | \vec{A}| 
\]
Using this, and using the fact that the \textbf{Vector Function} already is under the bound $\epsilon$, 
\[
|f_i(\vec{x})  - L_i| \le | f(\vec{x}) - \vec{L}| < \epsilon
\]
So we just showed 
\[
|f_i (\vec{x}) - L_i | \le \epsilon
\] 
\textbf{Component Limit implies Vector Limit}: Given $|f_i(\vec{x}) - L_i| < \beta$ (some bound), then we want to show $|f(\vec{x}) - \vec{L}| < \epsilon$. 

Now note that 
\[
|f(x) - \vec{L}| = \sqrt{ \left[ 
\sum_{i = 1}^{m} |f_i(\vec{x}) - b_i| ^2 
\right]} 
\]
This looks scary but this is just Pythagoras theorem lol. 

Let me further break down the logic for what we are about to do now, so every $|f_i(\vec{x}) - L_i| < \beta$ is unique for every $i$-th value, so it's important we had called it $\beta_i$ instead of just $\beta$. But I will assume out of all $\{\beta_i\} $, we pick the largest $\beta$ so that it's bigger than every bound. Hence $\beta = \text{max}(\{\beta_i\} )$

Hence $\beta$ being bigger than every term

\[
|f(x) - \vec{L}| = \sqrt{ \left[ 
\sum_{i = 1}^{m} |f_i(\vec{x}) - b_i| ^2 
\right]}  < 
\left[
m \beta^2
\right]^{\frac{1}{2}}  = \varepsilon 
\]
Choosing $\varepsilon = \sqrt{m \beta^2}$
}

\end{document}
