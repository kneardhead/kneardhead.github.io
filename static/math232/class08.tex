\documentclass[letter]{article}
\usepackage[monocolor]{ahsansabit}

\title{Honors Multivariable Calculus : : Class 08}
\author{Ahmed Saad Sabit, Rice University}
\date{\today}

\begin{document}
\maketitle 
Speed is defined to be \[
s ^2 = \vec{v} \cdot \vec{v}
\] 
\[
\frac{\mathrm{d} }{\mathrm{d} t} s^2 = v \cdot v' + v' \cdot  v = 2va
\] 
$s$ is increasing if and only if $v\cdot a > 0$, $s$ is decreasing if $v\cdot a< 0$. If it's is constantly zero then we have no motion $s = \text{const}$. 

Let's take pure circular motion,
\[
p(t) = (\cos t, \sin t)
\] So we get \[
v(t) = (-\sin t, \cos t)
\] 
\[
a(t) = (- \cos t, -\sin t)
\] 
Talking about $\vec{a}\cdot \vec{v}$ then 
\[
\vec{v} \cdot \vec{w} = |\vec{v}| |\vec{w}| \cos \theta
\] 
Dot product being acute angle is the most olympiad shit out there haha. In general, one should try to something coordinate free as long as you can.

\section{Thinking about Tangents} 
\begin{itemize}
	\item If you keep zooming into a curve it becomes a line. 
	\item Think about a water slide and you suddenly switch off gravity at some point. 
	\item I am thinking $\lim_{\vec{x} \to \vec{a}} $ then tangent line is a line parallel to the vector $\vec{x}-\vec{a}$. 
\end{itemize}

\end{document}
