\documentclass[letter]{article}
\usepackage[monocolor]{ahsansabit}

\title{Honors Multivariable Calculus : : Class 22}
\author{Ahmed Saad Sabit, Rice University}
\date{\today}

\begin{document}
\maketitle
\df{
Given $m$ functions $f_1 ,\ldots, f_m : \mathbb{R}^{n} \to \mathbb{R}$ where 
\[
F = (f_1, \ldots, f_m) : \mathbb{R}^{n} \to \mathbb{R}^{m}
\]
$\vec{a} \in  \mathbb{R}^{n}$, where $F(\vec{a}) = \vec{c}$. Relabel $x_{n-m+1} \ldots x_n$ as $z_1 ,\ldots, z_m$. Then near $\vec{a}$, the constraint 
\[
F(\vec{x}) = \vec{c}
\] defines $z_1, \ldots, z_m$ as implicit functions of $x_1, \ldots, x_{n - m} $ if 
\[
	\begin{pmatrix} \frac{\partial f_1}{\partial z_1} & \frac{\partial f_2}{\partial z_1} & \cdots \\
	\vdots & & \\ 
	       & & \frac{\partial f_ a}{\partial z_b }
\end{pmatrix}
\] 
}

\section*{Example} 
Intersection of $x^2 + y^2 + z^2 = 3$ and $x + 2y + 3z = 6$ near $\vec{a} = (1,1,1)$. Can we get $y,z$ as implicit function of $x$ near $\vec{a}$? 
\[
f_1 = \text{first one}
\] 
\[
f_2 = \text{second one}
\]
\[
F = (f_1, f_2)
\] 
\[
F(\vec{a}) = (3,6) = \vec{c}
\]
\[
	\begin{pmatrix} \partial_2f_1 & \partial_2 f_2 \\ \partial_3f_1 & \partial_3f_2 \end{pmatrix} (\vec{a}) = 
	\begin{pmatrix} 2 y & 2 \\ 2z & 3 \end{pmatrix}  (1,1,1) = \begin{pmatrix} 2 & 2 \\ 2 & 3 \end{pmatrix} 
\]
This is inververtible. So by the implicit function theorem we can treat $y,z$ as some function $h(x)$ near $1,1,1$ ($x,y,z$). 

To calculate $\frac{dy}{dx}$ and $\frac{dz}{dx}$. 
\[
y = h(x) \text{ and } z = j(x)
\]
Near the given point hte equation is going to hold, 
\[
x^2 + h(x)^2 + j(x)^2 = 3
\] 
\[
x + 2 h (x) + 3 j(x) = 6
\]
Taking a derivative, 
\[
2x + 2 h(x) h'(x) + 2 j(x) j'(x) = 0
\] 
\[
1 + 2 h'(x) + 3 j' (x) = 0
\] 
At $(x,y,z) = (1,1,1)$ that becomes, 
\[
2 + 2h' + 2j' = 0
\] 
\[
1 = 2h' + 3j' = 0
\] 
We just need to solve this system of equation for $h'$ and $j'$.
That same thing can be written as
\[
	\begin{pmatrix} 2 & 2 \\ 2 & 3 \end{pmatrix}  
	\begin{pmatrix} h' \\ j' \end{pmatrix} = 
	\begin{pmatrix} -2 \\ -1\end{pmatrix} 
\]
This is only going to work out well if you plot it. 

\section*{General Lagrange Multipliers} 
Constraint is given by $$g_1(\vec{x}) = c_1$$
\[
g_2(\vec{x}) = c_2
\] 
\[
g_m (\vec{x}) = c_m
\] 
So $G = (g_1, \ldots, g_m)$. Here 
\[
G: \mathbb{R}^{n} \to \mathbb{R}^{m}
\] 
And let's say that $X$ is $G^{-1} (\{\vec{c}\} )$. We want to optimize $f: \mathbb{R}^{n}\to \mathbb{R}$ constrained to $X$. The idea again is that you know there are going to be certain special, $f $ is not going to be maximized, unless something interesting happens. What's the interesting thing? 

Given $\vec{a} \in X$, if $f$ is differentiable at $\vec{a}$. 
\[
\{\nabla g_i (\vec{a})\} \text{ is linearly independent}
\]
And $\nabla f(\vec{a})$ is not in span of $\{\nabla g_i(\vec{a})\} $. Then $f$ is not max or min at $\vec{a}$ when restricted to $X$. 

\section*{INtuitive Justification} 
Draw the picture of a sphere getting intersected by a plane. Sphere is $g_1 = c_1$ and $g_2 = c_2$ is plane. Intersection is our $X$. 
\[
X = G^{-1} ( c_1, c_2)
\] 
Let's pick a point $\vec{a}$ right there on the intersection disk. If we are moving along the intersection then $g_1, g_2$ are constant. And so any tangent direction along the $X$ are $\perp$ to $\nabla g_i (\vec{a})$. Tangent directions along $X$ $\subset  \nabla g_i ^{\perp }$ $\forall i$. 
\[
	X \subset \bigcap_{i=1}^m \nabla g_i^{\perp} 
\]
Implicit function theorem says
\[
X = \cap  \nabla g_i ^{\perp}
\]
If $\nabla f$ not in span of $\{\nabla g_i\} $ then $\exists \text{some} \vec{v} \cap \nabla g_i^{\perp}$ where $\vec{v}$ is not perp to $f$. Going aong that direction will increase or decrease $f$. 

Think about $\nabla g_1$ and $\nabla g_2$ and they are perp to $\vec{t}$ tangent vector. $\nabla f$ is not in their span so it can't either be perp to $\vec{t}$. This diagram is necessary. 

\section*{Subject ot the constraints example}
\[
x^2 + y^2 + z^2 = 3
\] 
\[
x + 2y + 3z = 6
\] 
What is the maximum and minimum value of $x$? 
\[
f(x,y,z) = x
\] 
Are we guarenteed we are going to have a maximum or minimum? $f$ is continous function. Constraint is the $X$ which is compact. 
\nt{
Given $\vec{a} \in X$, if $f$ is differentiable at $\vec{a}$. 
\[
\{\nabla g_i (\vec{a})\} \text{ is linearly independent}
\]
And $\nabla f(\vec{a})$ is not in span of $\{\nabla g_i(\vec{a})\} $. Then $f$ is not max or min at $\vec{a}$ when restricted to $X$. 
}
These conditions are given in the note. 

\[
\nabla g_1 = (2x, 2y , 2z)
\]
\[
\nabla g_2 =( 1,2,3)
\] 
Are they every linearly dependent? well yes but they will be linearly dependent on points that are not on $X$. 
Linearly dependent if $(x,y,z) = \kappa (1,2,3)$
\[
\kappa^2 + (2\kappa)^2 + (3\kappa)^2 = 3
\] 
But we get $\kappa = \pm \sqrt{\frac{3}{14}} $. This point is outside of our required place of interest. 
\[
	(x,y,z) = \pm 
\left(
	\sqrt{\frac{3}{14}}  , 2 \sqrt{\frac{3}{14}} , 3 \sqrt{\frac{3}{14}} 
	\right)
\]
We don't have $\kappa$ range within $X$ (this sentence makes no sense lol). So, where is $\nabla f$ in span of $\nabla g_1, \nabla g_2$ are? 
\[
\nabla f = \lambda_1 \nabla g_1 + \lambda_2 \nabla g_2
\]
\[
	(1,0,0) = \lambda_1 (2x, 2y, 2z) + \lambda_2 (1,2,3)
\] 
\begin{align*}
	1 &= \lambda_1 2x + \lambda_2 \\
	0 &= \lambda_1 2y + 2\lambda_2 \\
	0 &= \lambda_1 2z + 3 \lambda_2 \\
	3 &= x^2 + y^2 + z^2 \qquad \text{(constraint 1)}\\
	6 &= x+ 2y + 3z \qquad \text{(constraint 2)} \\
\end{align*}
\end{document}
