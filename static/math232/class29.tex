\documentclass[letter]{article}
\usepackage[monocolor]{ahsansabit}

\title{Honors Multivariable Calculus : : Class 29}
\author{Ahmed Saad Sabit, Rice University}
\date{\today}

\renewcommand{\d}{\,\mathrm{d}} 
\begin{document}
\maketitle
Say $f$ is integrable on some compact domain $D \subset \mathbb{R}^{n}$. So $T$ is going to take $D'$ to $D$, one domain to another. Hence, $D' \subset \mathbb{R}^{n}$, and if 
\[
T : D' \to D
\]  is $C^1$, the function is ``onto". And $T$ is into except -> <fudge this?> 

Example. 

$D$ a annulus ring. Inner circle $1$ radius, outer is $3$. We don't want to integrate using $x,y$ but we do want to use $r, \theta$ here. So, 
\[
T(r, \theta) = r \cos \theta, r \sin \theta
\] 
So, $x = r \cos \theta$ and $y = r \sin \theta$. What is our $D'$ in the $r, \theta$ universe? The $r, \theta$ graph lower horizontal line and upper horizontal line is contant zero. So although not injective we don't care. $r = 1$  to $r= 3$ and $\theta = 0$ to $\theta = 2\pi $. But, $\theta = 2 \pi  = 0$, but this is content zero so we don't quite care. 

<fudge> $T$ is into (injective) except at possibly content $0$ portion of $D'$. 

Then, 
\[
	\int_D f = \int_{D'} f \cdot T | \det  T | 
\]


If we are trying this, 
\[
\int_D x^2 = \int_D' (r \cos \theta)^2  \cdot r (= | \det T| ) = \int_0^{2\pi } 
\int_{r = 1}^{r=3} r^3 \cos ^2 \theta \, \mathrm{d} r \, \mathrm{d}  \theta  
\]

Average distance of all points in a circle, 
\[
\frac{1}{\text{area }D} \int_D \sqrt{x^2 + y^2}  = \frac{1}{36 \pi } \int_D \sqrt{x^2 + y^2}  
\]
After change of coordinates, $\sqrt{x^2 + y^2 }  = r$
\[
\frac{1}{36 \pi } \int_{\theta = 0}^{2 \pi } \int_{r = 0}^{6} r | \det \d T| \d r \d \theta  
=\frac{1}{36 \pi } \int_{\theta = 0}^{2 \pi } \int_{r = 0}^{6} r \cdot  r \d r \d \theta  
\] 

Why do we want to change coordinates? 
\begin{itemize}
	\item Turns the problem boundary relatively easier to do.
	\item Makes the function (integrand) easier to deal with. 
	\item 
\end{itemize}

Let's try, 
\[
\int_D (x-y) ^3 (x^2 - y^2) ^{\frac{1}{3}}
\] 

\begin{figure}[ht]
    \centering
    \incfig{the-region-for-the-integration-(class-29)}
    \caption{The region for the integration (Class 29)}
    \label{fig:the-region-for-the-integration-(class-29)}
\end{figure}
Changing coordinates, 
\[
T(x,y) = (x-y, x+y)
\]
\[
x = \frac{u+v}{2} \quad y = \frac{v - u}{2}
\]
\[
\frac{
\partial 
\left( u, v\right)
}{\partial
\left(x , y\right)}  = 
\begin{pmatrix} 1 & -1 \\ 1 & 1 \end{pmatrix} 
\] 
\[ 
\det 	\frac{
\partial 
\left( u, v\right)
}{\partial
\left(x , y\right)}  = 
\det \begin{pmatrix} 1 & -1 \\ 1 & 1 \end{pmatrix}  = 2
\] 
\[
S (u , v) = (x,y)
\] 
We can just take inverse of the determinant of the $\frac{\partial (u,v)}{\partial (x,y)}$ for the integral.
\[
\det 
\frac{\partial (x,y)}{\partial (u,v)} = \det \frac{\partial (u,v)}{\partial (x,y)}^{-1}= \frac{1}{2}
\] 
The integral is now,
\[
	\int_{D'} u^{\frac{10}{3}} v^{\frac{1}{3}} \frac{1}{2} \d u \d v
= \int_{0}^{1} \int_{-v}^{v} \frac{1}{2} u^{\frac{10}{3}} v^{1 / 3} \d u \d v  
\]

\[
\int_{-\infty}^{\infty} e^{-x ^2 } \d x = \sqrt{ \pi }  
\] 
\end{document}
