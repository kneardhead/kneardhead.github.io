\documentclass[letter]{article}
\usepackage[monocolor]{ahsansabit}

\title{Honors Multivariable Calculus : : Homework 01}
\author{Ahmed Saad Sabit, Rice University}
\date{\today}

\begin{document}
	\pr{
	What is the name by which you would like me to call you?
	}
	\solu{
		I am happy with anything, but usually people just call me by my last name \emph{Sabit}. Professor Frank Jones calls me \emph{Mr. Bangladesh.} My mom calls me \emph{Idiot}.
	}

	\pr{
		What year at Rice are you?
	}
	\solu{
		Freshman, Class of 2027
	}

	\pr{
		What is your major?
	}
	\solu{
	I am a Theoretical Physics major. Because Rice doesn't has a by name "Theoretical" Physics, I want to take a lot of higher level maths and call it a Theoretical Physics degree. Other than that I might minor in German Studies/Philosophy/Computer Science given how I navigate through college later on.}

	\pr{
		When and where did you take Linear Algebra? How do you feel about what you remember?
	}
	\solu{
		I have not officially taken Linear Algebra but I had a 4 year long exposure to Competitive Physics because I regularly competed in Physics Olympiads. At one point of learning Tensors and Tensor Calculus I had picked up some basic street-fighting level Linear Algebra. I have later studied bit and pieces of Linear Algebra now and then and I am confident to pick up linear algebra when it becomes necessary during this course. 
	}

	\pr{When and where did you take Calculus 1?}
	\solu{I never had taken Calculus 1 but, again, because of Physics Olympiads I have a very strong experience in using Calculus.}

	\pr{Have you seen any multivariable calculus before? How much and where?}
	\solu{I studied the Physics that used Multivariable calculus extensively in 2019, it was core to solving problems. This was not a part of my school education (fun fact I didn't have a much strong formal education back in Bangladesh). It has always came up in Electrodynamics now and then, and I went through Multivariable the most before appearing at Asian Physics Olympiad this year. I absolutely love Olympiads and I can't stop talking about it, but to be fair, I know how to use multivariable to solve physics problems, but I don't know how it was derived.}
	
	\pr{What courses have you taken before?}
	\solu{I have taken \textbf{MATH 211} last semester. This semester I am taking \textbf{MATH 354, MATH 382} alongside this one. }

	\pr{Is there anything particular you want me to know about why are you taking this class?}
	\solu{I don't properly understand the question but I am taking this class because last semester I felt in 211 I learned no mathematics at all because nothing was being derived. I didn't feel challenged and by the end I gave up studying for that course. I have always been challenged before when I had practiced for Olympiads and MATH232 seems to be like that healthy nostalgia where I feel I am learning maths in general.}

	\pr{Have you read the Homework policies in the syllabus?}
	\solu{Yes! I usually work alone and talk about homework after submission so it's fine by me.}

	\pr{Go to Piazza page and respond to introduce yourself post}
	\solu{Okay!}


	\pr{Let $m: \mathbb{R}^{2} \to \mathbb{R}$ be a multiplication function. That is,
		\[
		m(x,y) = xy
		\] 
		Formally show that,
		\[
		\lim_{(x,y) \to (a,b)} m(x,y) = ab
\] Where both $a,b$ are positive. }
\solu{
	I like to imagine the ball around $(x,y)$ with radius $\delta$, when linear mapped to the function we get the answer between the region $(f(x,y))$ inside an $\epsilon$ radius ball. 	
}



\end{document}
