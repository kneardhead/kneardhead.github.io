\documentclass[letter]{article}
\usepackage[monocolor]{math232/ahsansabit}

\title{FWIS 198 Brainstorming}
\author{Ahmed Saad Sabit, Rice University}
\date{\today}

\begin{document}
\maketitle

\textbf{Essays that I think I can write something from:}
\begin{itemize}
	\item The Earthworm Essay
	\item The Weasel Essay
	\item The Essay where Montaigne writes about almost dying. 
	\item My own essay on my notepad I wrote about dealing with Conformity and Protocols and how soothing they are. 
\end{itemize}

Expanding on the essay I wrote myself; reading through the assignments I have been thinking about "Thinking". My psychiatrist just had this incredibly correct guess from listening to me that I really like following rules and protocols; and recently I have began dealing with life by crafting my skills on writing methods. 

One very successful thing I did recently was I could get my room cleaned in 30 minutes without feeling extremely afraid of it. You know, I get very anxious trying to do basic life things unless I am very near the edge of deadlines (if it exists. For cleaning, my room gets really messy and I really hate untidy rooms).

I was struggling very hard last night to sit down to start working on my quick assignments because I had "irrational fears" with assignments. I procrastinated by writing a very clear set of algorithms that I would follow to clean everything up. 

I started from a corner of the bed, fixed mattress, then pillows, then the comforter, then moved to the thing right beside the bed, then moved right to the closet and fixed my clothes and followed a very specific and I finished the job without a single discomfort. 

I am a serious fond of this idea of making rules out of everything. I definitely want to find this conformity in these essays. This FWIS course has helped me tremendously in conquering these stupid fears. 

I am inspired to write about things that give the reader a stroll through my mind. Because all my life, I have written things only to appease a reader. Not that a reader exist, but it's the readers perspective where I have always written from. Having an author-centered essay is a wholly flipped perspective I was not looking through. 

\emph{
I think I can write something like Conquering ourselves through Essays.}

Let me explain what I am saying through some points. 

\begin{itemize}
	\item  Special pressure of individual thinking has been given in the Earthworms essay. It inspires us to be our own thing. 
	\item The Migraine essay, although I haven't listed it above, is an excellent example of being able to write about one's feelings and troubles. 
	\item The Weasel essay inspires me to use euphemisms and somewhat casual linguistic jewelry to make writings vivid. And other than that, this essay has a very strong presence of the author's "aura". I think it's important when it comes to writing an essay that the point is not just to get your thoughts out but also not hesitating to drop those subtle personalistic-traits that make it possible for a familiar acquintance to read that essay and point out, ``Oh I know who wrote this, this sounds like something Annie Dillard would write!". 
	\item Self-reviewing like Montaigne, especially in his "On Discussion" essay. 
	\item From Montaigne I have realized that having a future self doesn't always imply that future self is always better than who he was. Yes, idea's change, but I've always thought an individual always becomes better over a period of time. Montaigne has broke that perception of mine through his essays. 
\end{itemize}



\textbf{I can write something along the line: Solving Problems and Growing through Essays:}

\begin{itemize}
	\item I would stress that one person is required to write an essay about something. 
	\item He/She would follow some rules (I am yet to invoke these rules) to write that essay. 
	\item After writing that essay, he/she will let it ferment it for a while. 
	\item After some time of writing that essay, he or she will come back and start updating his/her own thoughts. 
	\item Not just essays, the individual can write rules of doing things or ways to get around life. 
	\item He/She lets it sit and come back to it in the future. The stroll through one's mind can serve as a ground that can be further improved. But from Montaigne I know that future self isn't always correct. I want to commit a research on this. 
\end{itemize}

\textbf{A similar experience I have}

I got myself a 4 dollar composition note book that is kind of clumsy to write because of the weird type of binding it has. But I have written a few sections on that notebook with each section receiving two free sides of the paper real-estate. 

\begin{itemize}
	\item On Work [essays on my overall productivity as a student because I deal with irrational anxiety related to coursework and I deal with them there, sometimes improvising methods]
	\item On appearance [essays on types of clothing I like (I happen to like whatever grandfather's wear these days)
\item On management of thoughts [metaphysical, the physics of physics, thoughts about thoughts, managing how to manage things, every effect that requires that same effect on itself, I write them there. For instance one entry of a three line ``essay" there is an attempt for me to write why I got myself a clumsy composition notepad. 
\item On fear [I delayed working on a 4 line math problem for a week until the very last moment because I had done two long and wrong calculations before and I was extremely afraid to work on that again. I wrote about being afraid of working on a harmless math problem, inspired somewhat from Montaigne's memories of almost dying.]
\item On year long delays [has information about things I have delayed because I didn't do them for a long time out of anxiety. One being application of study abroad program, and another being joining a physics research lab.] 
\end{itemize}

\textbf{What can I do to make this academic article worth it?}
\begin{itemize}
	\item I propose a method to write Essays in a journal in a way to solve life problems. \emph{The method is what I need to study. But this isn't a "method"-"method". Rather I believe it is going to be a reason why writing an Essay, even a short one could be impactful, deriving inspiration from the essays that I've read. } 
	\item I am starting to think this Essay would serve as a method of notetaking through which ``\emph{one person can be able to make a notebook his second brain - or an extension of his brain where thoughts are commited not in the head but through hand"}. 
	\item A possible title can be \textsf{Essays: attempts to a Second Brain.}
\end{itemize}

\end{document}
