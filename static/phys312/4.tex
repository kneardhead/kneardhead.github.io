\documentclass[letter, 10pts]{article}
\usepackage[monocolor]{../math232/ahsansabit}
\usepackage[]{float}
\usepackage{tikz}
\usepackage{tikz-3dplot}
\usepackage[outline]{contour} % glow around text
\usepackage{xcolor}
\usepackage{pdfpages}
\usepackage{physics}
\usepackage{multicol}
\title{Quantum Mechanics : : Homework 0X}
\author{Ahmed Saad Sabit, Rice University}
\date{\today}
\newcommand{\hb}{\hbar}
\newcommand{\U}{\uparrow}
\newcommand{\D}{\downarrow}
\usepackage[]{braket}
\begin{document}
\maketitle


\section*{Problem 4} 
\subsection*{(a)} 
\begin{align*}
\sigma_B &= \vec{P} \cdot \vec{n} 
\\
&= (k \vec{r}) \cdot  \vec{n}  \\
&= kr \\
\end{align*} 
At the surface the charge is $\sigma_B = k R$. Total charge at the surface, 
\[
Q_\text{surface} = 4 \pi R^3
\] 
\begin{align*}
\rho_B &= - \nabla \cdot  \vec{P} \\
&= -\frac{1}{r^2} \frac{\partial }{\partial r}(r^2 k r)  \\
&= -\frac{1}{r^2} \frac{\partial}{\partial r}(r^3 k)  \\
&= -\frac{k}{r^2} (3 r^2 )  \\
&= -3k \\
\end{align*}
Total charge in the volume is 
\[
Q_\text{volume} = \frac{4}{3} \pi R^3 (- 3 k) = - 4 \pi R^3 
\] 

\textbf{Answers:}
Surface charge density \[
\boxed{
\sigma_B = kr
}
\] 
Volume charge density 
\[
\boxed{
\rho_B = - 3k
}
\] 

\subsection*{(b)} 
For the homogeneous charge distribution inside the surface, hence using Gauss's law we know that the Electric field is just going to be 
\[
	E_r (4 \pi r^2) = (-3k) (4 \pi r^3 / 3\epsilon_0) 
	\implies 
	E_r = - \frac{k}{\epsilon_0}\tag{$r < R$}
\] 

Outside, using Gauss's law
\[
	E_r (4 \pi \mathrm{r}   ) = \frac{Q_\text{surface} + Q_\text{volume} }{\epsilon_0} = 0   
	\tag{$\mathrm{r} > R $}
\]  
The field outside the ball is zero. 

\textbf{Answers:} Electric Field \[
\boxed{E_r =
\begin{cases}
 - k / \epsilon_0 & r < R\\
	 0 & r > R 
\end{cases}} 
\] 




\section*{Problem 5} 
The uniform polarization along $\vec{P} = P \hat{z}$. 
Bound charge 
\[
\rho_B = - \nabla \cdot  \vec{P} = \frac{\mathrm{d} P}{\mathrm{d}z } = 0
\] 
Surface charge (on flat ends) because curved surface would be zero. Top face
\[
\sigma^{\text{top}}_B = \vec{P} \cdot \hat{n} = P
\]
Bottom face
\[
\sigma^{\text{lower}}_B = \vec{P} \cdot  \hat{n} = -P 
\] 

















\section*{Problem 6} 
Compute the total charge
\[
	Q = \int \mathrm{d} q_{\text{volume}}  + \int \mathrm{d} q_{\text{surface}} 
\]
\begin{align*}
Q &= 
\int \mathrm{d} V (\rho_b) + 
\int \mathrm{d} A (\sigma_b) \\
  &=
- \int \mathrm{d} V (\nabla \cdot \vec{P}) + 
\int \mathrm{d} A (\vec{P} \cdot \vec{n})\\
&= -  \int \mathrm{d} \vec{S} \cdot  \vec{P} + 
\int \mathrm{d} A (\vec{P} \cdot \vec{n})\\
&= - \int \mathrm{d} A  \vec{P} \cdot \hat{n} + 
\int \mathrm{d} A (\vec{P} \cdot \vec{n})\\ 
&= 0 \\
\end{align*}

\end{document}
