\documentclass[10pt]{article}
\usepackage{NotesTeXV3,lipsum}
%\usepackage{showframe}

\begin{document}
	\title{{Quantum Mechanics : PHYS 311}\\{\normalsize{\itshape 
Attempting to write a study guide for midterm
	}}}
	\author{Ahmed Saad Sabit}
	\affiliation{
	Sophomore at Rice University\\
	\href{https://kneardhead.github.io/}{Website}\\
	}
	\emailAdd{ahmedsaadsabit@rice.edu}
	\maketitle
	\newpage
	\pagestyle{fancynotes}

\part{Methodology}
\section{Reading this handout} 
What I am going to do is that
\begin{itemize}
	\item Read the lectures from the first day. 
	\item Read the example solved problems in problem session. 
	\item Find out some relevant problems from outside sources that are in context, and then type their solution up. 
\end{itemize}
\begin{prob}
	This is a problem statement.
\end{prob}
\begin{solu}
	This is a solution statement. 
\end{solu}
\begin{definition}
	This is a definition. 
\end{definition}
\begin{fact}
This is a fact. By fact I mean it's a derivative of Definition.
\end{fact}



\begin{align*}
	\cos \theta &= 1 - \frac{1}{2!} x^2 + \frac{1}{4!} x^{4} - \cdots  \\ 
	\sin \theta &= x - \frac{1}{3!} x^3 + \frac{1}{5!} x^{5} - \cdots \\
\end{align*}


\begin{align*}
	\begin{bmatrix} 
		\cos \theta & - \sin \theta & 0 \\
		\sin \theta & \cos \theta & 0 \\ 
		0 & 0 & 1
	\end{bmatrix} \\ &=
	\begin{bmatrix} 
	\cos \theta = 1 - \frac{1}{2!} x^2 + \frac{1}{4!} x^{4} - \cdots   
	& 
-	\sin \theta = - x + \frac{1}{3!} x^3 - \frac{1}{5!} x^{5} + \cdots 
	&
	0
	\\
		\sin \theta = x - \frac{1}{3!} x^3 + \frac{1}{5!} x^{5} - \cdots & 
		\cos \theta = 1 - \frac{1}{2!} x^2 + \frac{1}{4!} x^{4} - \cdots  & 
0 \\ 
		0 & 0 & 1  
	\end{bmatrix}  \\  
	& = \text{ roughly speaking } 
	\sum x^{n}/n!
	\begin{bmatrix} 
		1,0,-1,0,1, \ldots & 
		0, -1, 0 , 1, 0 \ldots & 0 
		\\
		0, 1, 0, -1, 0, 1 \ldots  & 
		1, 0, -1, 0, 1 \ldots \\ 
		0 & 0 & 1 
	\end{bmatrix}  \\ 
	&= 
	\begin{bmatrix} 1&0&0\\0&1&0\\0&0&1 \end{bmatrix}  + 
	x 
	\begin{bmatrix} 0&-1&0\\1&0&0\\0&0&0 \end{bmatrix}  + 
	\frac{x^2}{2!} 
	\begin{bmatrix} -1&0&0\\0&-1&0\\0&0&0 \end{bmatrix} + 
	\frac{x^3}{3!} 
	\begin{bmatrix} 0&1&0\\-1&0&0\\0&0&0 \end{bmatrix}  + \cdots
	\\ &= 
\hat{I} + x G_3 + \frac{x^2}{2!} G_3^2 + \frac{x^3}{3!} G_3^3 + \cdots
= e^{\theta G_3} 
\end{align*}


\end{document}
